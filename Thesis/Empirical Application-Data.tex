\documentclass[12pt]{article}
\usepackage{amsmath}
\usepackage{mathptmx}
\usepackage{setspace}
\usepackage{amssymb}
\usepackage{float}
\usepackage{graphicx}
\usepackage{booktabs}
\usepackage{subfigure}
\usepackage[margin=2.5cm]{geometry}
\usepackage[title]{appendix}
\usepackage{bm}
\usepackage{tcolorbox}
\usepackage{apacite}
\onehalfspacing
\usepackage{lineno}
\linenumbers
\usepackage{diagbox}

\DeclareMathOperator*{\argmin}{arg\,min}
\newtheorem{experiment}{Experiment}



\usepackage{fontspec}
\setmainfont{Times New Roman}
\addtolength{\jot}{0.5em}
\linespread{1.5}

\begin{document}
	\section{Empirical Application}
	\subsection{Data}
	
In the empirical section, for the return of assets,  we use the monthly excess returns from Standard Poor (S\&P) 500 index component companies.\footnote{The data was obtained from the Global Finance Data, Osiris, and Yahoo Finance}
We prepared three data sets for different time spams: 10 years (January 2008 to December 2017), 20 years (January 1998 to December 2017), and 30 years (January 1989 to December 2017).
Because of the components companies of the index are constantly changing, for each of the datasets, the companies amount (n) is different, the dimensions of the data set is showing in the table (\ref{Data_set}).


\begin{table}[h]
		\caption{Data Set Dimensions}
			\label{Data_set}
	\begin{tabular}{c|ccc}
		\hline
		& Time Spam                    & Companies Amount (n) & Observations Amount (T) \\ \hline
		10 Years & January 2008 - December 2017 & 419                  & 120                     \\
		20 Years & January 1998 - December 2017 & 342                  & 240                     \\
		30 Years & January 1988 - December 2017 & 242                  & 360                     \\ \hline
	\end{tabular}
\end{table}
The one-month U.S. treasury bill return rate was set as the risk free return $r_{ft}$.
For company i, we calculates the companies return at month t ($r_{it}$) use the following formula:
\begin{align*}
r_{it} = \frac{p_{i t} - p_{i t-1}}{p_{i t-1}}
\end{align*}
and calculate the access return $x_{it} = r_{it} - r_{ft}$.
Here the $p_{it}$ and $p_{i t-1}$ are the company's close stock price at the first day of month t and t-1.
The price is adjusted for the dividends and splits.\footnote{The data is adjusted base on the Central for Research in Security Price (CRSP) method.}

	With regard of the factors, we use 146 different risk factors, including the market factors as market return minus risk free rate form \citeA{Feng2020}.

\newpage
\bibliographystyle{apacite}
\bibliography{thesis.bib}
\end{document}