\chapter{Factor Strength}\label{strength}
The concept of factor strength employed in this project comes from \citeA{Bailey2020}, and it was first introduced by \citeA{Bailey2016}.
They defined the strength of factor from the prospect of the cross-section dependences of a large panel and connect it to the pervasiveness of the factor, which is captured by the factor loadings.
In a separate paper, \citeA{Bailey2019} extended the method by loosening some restrictions and proved that their estimation can also be applied on the residuals of regression result.
Here, we focus on the case of observed factor, and use the method of \citeA{Bailey2020} in this project.

\section{Definition}\label{strength_definiton}
Consider the following multi-factor model for n different cross-section units and T time-series observations with k  factors.

\[  x_{it} = a_{i}+  \sum_{j=1}^{k}\beta_{ij}f_{jt} + \varepsilon_{it} \tag{1}\label{definition_model} \]

In the left-hand side, we have $x_{it}$ denotes the cross-section unit i at time t, where $i = 1, 2, \cdots, n$ and $t = 1,2, \cdots T$.  
In the other hand side, $a_{i}$ is the constant term.
We use $f_{jt}$ of $j = 1, 2, \cdots k$ to represent the factors included in the model, and $\beta_{ij}$ to represent the corresponding factor loadings.
For the error term, we use $\varepsilon_{it}$.

The concept of factor strength relates to how many non-zero loadings correspond to a factor.
More precisely, for a factor $f_{jt}$ with n different factor loading $\beta_{ij}$ and strength $\alpha_j$, we assume that:

\begin{align*}
|\beta_{ij}| &> 0\quad i = 1, 2,  \dots, [n^{\alpha_j}]\\
|\beta_{ij}| &= 0 \quad i = [n^{\alpha_j}] + 1, [n^{\alpha_j}] +2 ,\dots, n
\end{align*}

%The $\alpha_j$ represents the strength of factor $f_{jt}$ and $\alpha_j \in [0,1]$.
All factor strength, by definition, should fall in the interval of $[0,1]$.
In this case, the strength $\alpha_j$ represents that the first $[n^{\alpha_j}]$ loadings are all significantly different from zero.
Here $[\cdot]$ is defined as the integer operator, which will only take the integer part of the inner value.
%If a factor has strength $\alpha_j$, we will assume that the first $[n^{\alpha_j}]$ loadings are all different from zero, and here $[\cdot] $  is defined as the integer operator, which will only take the integer part of the inside value.
The remaining $n - [n^{\alpha_j}]$ terms are all equal to zero. 
Assume for a factor which has strength $\alpha = 1$, the factor's loadings will be non-zero for all cross-section units.
We will refer such factor as a strong factor.
The financial theory suggests that the market factor, denotes by the difference between average market return and market risk free return, should have strength equals to one.
And if we have factor strength $\alpha = 0$, it means that the factor has all factor loadings equal to zero, and we will describe such factor as a weak factor \cite{Bailey2016}.
For any factor with strength in [0.5, 1), we will refer such factor as semi-strong factor.
In general term, the more non-zero loading a factor has, the stronger the factor's strength is. 

\section{Estimation Under Single-factor Setting}\label{strength_one_factor_estimation}
To estimate the strength $\alpha_j$, \citeA{Bailey2020} suggest the following procedure.

We first present the procedure in the case a single factor model with only factor $f_t$ and corresponding loading $\beta_{i}$.
The index i indicates $i^{th}$ unit, and t represents the $t^{th}$ time observation.
Here $v_{it}$ is the stochastic error term.

\[  x_{it} = a_{i} +  \beta_{i}f_{t} + v_{it} \tag{2} \label{estimation_model}\]

Assume we have n different units and T observations for each unit: $i = 1, 2,  \cdots, n$ and $t = 1,2, \cdots T$.
Running the OLS time-regression for each $i = 1,2, \cdots, n$, we obtain:
\[   x_{it} = \hat{a}_{iT} +  \hat{\beta}_{iT}f_{t} + \hat{v}_{it}  \]

For every estimated factor loading $\hat{\beta}_{iT}$, we can construct a t-test to examine the significance under the null hypothesis of the loading is zero.
The t-test statistic will be $t_{iT} = \frac{\hat{\beta}_{iT} - 0}{\hat{\sigma}_{iT}}$.  
Empirically, we calculate the t-statistic of $\hat{\beta}_i$ using:

\[t_{i T}=\frac{\left(\bm{f}^{\prime} \bm{M}_{\tau} \bm{f}\right)^{1 / 2} \hat{\beta}_{i T}}{\hat{\sigma}_{i T}}=\frac{\left(\bm{f}^{\prime} \bm{M}_{\tau} \bm{f}\right)^{-1 / 2}\left(\bm{f}^{\prime} \bm{M}_{\tau} \bm{x}_{i}\right)}{\hat{\sigma}_{i T}} \tag{3} \label{test_statistic} \]

Here, the $\bm{M}_{\tau} = \bm{I}_T - T^{-1}\bm{\tau}\bm{\tau^\prime}$, and the $\bm{\tau}$ is a $T\times 1$ vector with every element equals to one, $\bm{f}$ and $\bm{x_i}$ are two vectors with: $\bm{f} = (f_1, f_2 \cdots, f_T)^{\prime}$   $\bm{x_i} = (x_{i1}, x_{i2}, \cdots, x_{iT})$.
The denominator $\hat{\sigma}_{iT} = \frac{\sum_{i=1}^{T} \hat{v}^2_{it} }{T}$.

Using this test statistic, we can then define an indicator function as: $\ell_{i,n} := {\bf1}[|\beta_i|>0]$.
If the factor loading is non-zero , $\ell_{i,n} = 1$.
In practice, we use the $\hat{\ell}_{i,nT} := {\bf1}[|t_{it}|>c_p(n)]$.
Here, if the t-statistic $t_{iT}$ is greater than critical value $c_p(n)$, i.e. if the test result reject the null hypothesis, we yield $\hat{\ell}_{i,n} = 1$, otherwise we have $\hat{\ell}_{i,n} = 0$.
In other words, we are counting how many $\hat{\beta}_{iT}$ are significantly different from zero.
With the indicator function, we then define $\hat{\pi}_{nT}$ as the fraction of significant factor loading amount to the total factor loading amount:

\[  \hat{\pi}_{nT} = \frac{\sum_{i=1}^n \hat{\ell}_{i,nT}}{n} \tag{4} \label{pi_function} \]


In term of the critical value $c_p(n)$, we use the following critical value function.

\[   c_p(n) = \Phi^{-1}(1 - \frac{p}{2n^\delta})   \tag{5} \label{critical_value_function} \]

Suggested by \citeA{Bailey2019}, here, $\Phi^{-1}(\cdot)$ is the inverse cumulative distribution function of a standard normal distribution, p is the size of the test, and $\delta$ is a non-negative value represent the critical value exponent. 
Adopting this adjusted value helps to tackle the problem of multiple-testing.


After obtaining the $\hat{\pi}_{nT}$, we can use the following formula provided by \citeA{Bailey2020} to estimate the strength of corresponding factor:
\[ \hat{\alpha} = \begin{cases}
1+\frac{\ln(\hat{\pi}_{nT})}{\ln n} & \text{if}\; \hat{\pi}_{nT} > 0,\\
0, & \text{if}\; \hat{\pi}_{nT} = 0.
	\end{cases} \tag{6} \label{estimation_method} \]

When we have the $\hat{\pi}_{nT} = 0$, it means that none of the factor loadings are significantly different from zero, therefore the estimated $\hat{\alpha}$ will be equal to zero. 
From the estimation, we can find out that $\hat{\alpha} \in [0,1]$.

\section{Estimation Under Multi-Factor Setting}\label{strength_multi_estimation}

This estimation can also be extended into a multi-factor set up.
Consider the following multi-factor model:

\[x_{it} = a_i +\sum_{j = 1}^k\beta_{ij}f_{jt} +v_{it} = a_i + \bm{\beta}^{\prime}_{i}\bm{f}_{t} +v_{it} \tag{7} \label{multi_factor_model} \]

In this set up, we have $i = 1, 2, \cdots, n$ units, $t = 1, 2, \cdots, T$ time-series observations, and specially, $j = 1, 2,\cdots, k$ different factors.
Here $\bm{\beta}_{i} = (\beta_{i1}, \beta_{i2}, \cdots, \beta_{ij})^{\prime} $ and $\bm{f}_t = (f_{1t}, f_{2t}\cdots, f_{jt})^{\prime}$.
We employed the same strategy as above, after running OLS and obtain the:

\[ x_{it} =\hat{a}_{iT} + \bm{\hat{\beta}}^{\prime}_{i}\bm{f}_{t} + \hat{v}_{it}    \]

We use the significance test to exam how many factor loadings are non-zero.
To conduct the significance test, we calculate the t-statistic: $t_{ijT} = \frac{\hat{\beta}_{ijT}-0}{\hat{\sigma}_{ijT}}$. Empirically, the test statistic can be calculated using:
\[ t_{i j T}=\frac{\left(\bm{f}_{j t}^{\prime} \bm{M}_{F_{-j}} \bm{f}_{j t}\right)^{-1 / 2}\left(\bm{f}_{j t}^{\prime} \bm{M}_{F_{-j}} \bm{x}_{i}\right)}{\hat{\sigma}_{i T}} \]

Here, $\bm{f}_{j t}=\left(f_{j 1}, f_{j 2}, \ldots, f_{j T}\right)^{\prime}, \bm{x}_{i}=\left(x_{i 1}, x_{i 2}, \ldots, x_{i T}\right)^{\prime}, \bm{M}_{F_{-j}}=\bm{I}-\bm{F}_{-j}\left(\bm{F}_{-j}^{\prime} \bm{F}_{-j}\right)^{-1} \bm{F}_{-j}^{\prime}$, and $\bm{F}_{-j}=\left(\bm{f}_{1 t}, \ldots, \bm{f}_{j-1 t}, \bm{f}_{j+1 t}, \ldots, \bm{f}_{m t}\right)^{\prime}$
For the denominator's $\hat{\sigma}_{iT}$, it was from $\hat{\sigma}_{i T}^{2}=T^{-1} \sum_{t=1}^{T} \hat{u}_{i t}^{2}$, the $\hat{u}_{it}$ is the residuals of the model.
Then, we can use the same critical value from (\ref{critical_value_function}).
Obtaining the corresponding ratio $\hat{\pi}_{nTj}$  from (\ref{pi_function}), and use the function:
\begin{equation*}
\hat{\alpha}_{j}=\left\{\begin{array}{l}
1+\frac{\ln \hat{\pi}_{n T, j}}{\ln n}, \text { if } \hat{\pi}_{n T, j}>0 \\
0, \text { if } \hat{\pi}_{n T, j}=0
\end{array}\right.
\end{equation*}
to estimate the factor strength under the multi-factor scenario.
