%\documentclass[12pt]{article}
%\usepackage{amsmath}
%\usepackage{mathptmx}
%\usepackage{setspace}
%\usepackage{amssymb}
%\usepackage{float}
%\usepackage{graphicx}
%\usepackage{booktabs}
%\usepackage{subfigure}
%\usepackage[margin=2.5cm]{geometry}
%\usepackage[title]{appendix}
%\usepackage{bm}
%\usepackage{tcolorbox}
%\usepackage{apacite}
%\onehalfspacing
%\usepackage{lineno}
%\linenumbers
%\usepackage{diagbox}

%\DeclareMathOperator*{\argmin}{arg\,min}
%\newtheorem{experiment}{Experiment}



%\usepackage{fontspec}
%\setmainfont{Times New Roman}
%\addtolength{\jot}{0.5em}
%\linespread{1.5}

%\begin{document}
	\section{Empirical Application}\label{Empirical}

	In this section, we introduced the data prepared for the empirical application, discuss the results of factor strength estimations. Then we apply the elastic net method introduced before, 
	\subsection{Data}\label{data}
	
In the empirical application part, we use the monthly U.S. stock return as the assets.
The companies are selected from Standard Poor (S\&P) 500 index component companies.\footnote{The data was obtained from the Global Finance Data, Osiris, and Yahoo Finance.}
We prepared three data sets for different time spams: 10 years (January 2008 to December 2017), 20 years (January 1998 to December 2017), and 30 years (January 1989 to December 2017).
Because of the components companies of the index are constantly changing, bankrupt companies will be moved out, and new companies will be added in.
Also, some companies does not have enough observation.
Therefore, for each of the datasets, the companies amount (n) are different, the dimensions of the data set is showing in the table (\ref{Data_set}) below.

\begin{table}[h]
		\caption{Data Set Dimensions}
			\label{Data_set}
	\begin{tabular}{c|ccc}
		\hline
		& Time Spam                    & Companies Amount (n) & Observations Amount (T) \\ \hline
		10 Years & January 2008 - December 2017 & 419                  & 120                     \\
		20 Years & January 1998 - December 2017 & 342                  & 240                     \\
		30 Years & January 1988 - December 2017 & 242                  & 360                     \\ \hline
	\end{tabular}
\end{table}
For the risk free rate, we use the one-month U.S. treasury bill return.\footnote{ The data was fetched from the Kenneth R. French website: http://mba.tuck.dartmouth.edu/pages/faculty/ken.french/}
For company i, we calculates the companies return at month t ($r_{it}$) use the following formula:
\begin{align*}
r_{it} = \frac{p_{i t} - p_{i t-1}}{p_{i t-1}}\times 100
\end{align*}
and calculate the excess return $x_{it} = r_{it} - r_{ft}$.
Here the $p_{it}$ and $p_{i t-1}$ are the company's close stock price at the first day of month t and t-1.
The price is adjusted for the dividends and splits.\footnote{The data is adjusted base on the Central for Research in Security Price (CRSP) method.}

With regard of the factors, we use 145 different risk factors from \citeA{Feng2020}.
The factor set also includes one market factor, represented by the difference between the average market return and risk free return.
The average market return is a weighted average return of all stocks in U.S. market, incorporated by CSRP.
Each factors contains monthly value from the January 1988 to December 2017.

\subsection{Factor Strength Analysis}

\subsubsection{Regression model setting}
For the first part of the empirical application, we estimates the factor strength using the method discussed in the section \ref{strength}.
Precisely, we set the regression model base on the discussion of section \ref{strength_multi_estimation}.

\[  x_{it} = a_i + \beta_{im}(r_{mt} - r_{ft}) + \beta_{ij}f_{jt} + v_{it}  \]

where $x_{it}$ is the excess return of asset i at time t, which is pre-defined in the section \ref{data}.
$r_{mt} - r_{ft}$ represents the market factor, calculated by the difference between average market return and risk free return at the same time t.
$f_{jt}$ is the value of $j^{th}$ risk factor  at time t. 
Here $j = 1, 2, 3,\cdots 145$ . 
$\beta_{mt}$ and $\beta_{ij}$ are the factor loadings for market factor and risk factor, respectively.

	\subsubsection{Factor Strength Finding}
The complete results of factor strength estimation is present in the appendix \ref{strength_table} and \ref{strength_figures}.
We estimates the factors' strength using three different data set discussed in the section \ref{data}, and list those strength from strong to weak, alongside the market factor strength, in the table (\ref{table:three_compare}).
For different data sets, we obtains inconsistent factor strength estimations.
In general, the ten-year data set provides a significantly weaker result, compares with the other two data sets results.
Except the market factor, no factors shows strength above 0.8 from the ten-years result.
The strongest factor beside the market factor is the beta factor which has strength around 0.75.
The strongest risk factor in the twenty-year data set is the ndp (net debt-to-price),which has strength 0.904.
In the thirty-year scenario, the salecash (sales to cash) is the strongest with strength 0.857.
When comparing the proportion (see table (\ref{table:proportion})), we can find that nearly 40\% factors from the ten year dataset shows strength less than 0.5, which is almost three times higher than the twenty and thirty years proportion.
If we use 0.8 as a threshold, we can see that there are over forty percent factors in the twenty-year results excess this threshold, and the percentage for the thirty year results is 31\%.

\begin{table}[hb]
	\caption{Proportion of Strength (Excluded Market Factor) }\label{table:proportion}
	\begin{tabular}{lccc}
		\hline
		\hline
		Strength Level & \multicolumn{1}{l}{10 Year Data Proportion} & \multicolumn{1}{l}{20 Year Data Proportion} & \multicolumn{1}{l}{30 Year Data Proportion} \\ \hline
		{[}0.9, 1{]}   & 0\%                                         & 2.07\%                                      & 0\%                                         \\
		{[}0.85, 0.9)  & 0\%                                         & 24.1\%                                      & 4.14\%                                      \\
		{[}0.8, 0.85)  & 0\%                                         & 16.6\%                                      & 27.6\%                                      \\
		{[}0.75, 0.8)  & 0\%                                         & 8.28\%                                      & 12.4\%                                      \\
		{[}0.7, 0.75)  & 7.59\%                                      & 11.7\%                                      & 9.66\%                                      \\
		{[}0.65, 07)   & 15.9\%                                      & 5.52\%                                      & 15.9\%                                      \\
		{[}0.6, 0.65)  & 17.9\%                                      & 8.28\%                                      & 5.52\%                                      \\
		{[}0.55, 0.6)  & 13.1\%                                      & 8.97\%                                      & 5.52\%                                      \\
		{[}0.5, 0.55)  & 8.97\%                                      & 2.76\%                                      & 4.83\%                                      \\
		{[}0, 0.5)     & 36.6\%                                      & 11.7\%                                      & 14.5\%                                      \\ \hline\hline
	\end{tabular}
\end{table}
Another important finding is that from the twenty year data set, we obtained three factors: ndq (Net debt-to-price, $\hat{\alpha}$ = 0.904 ), salecash (sales to cash, $\hat{\alpha}$ = 0.902), and quick (quick ratio, $\hat{\alpha}$ = 0.901) has strength greater than 0.9.
We would expected when applying the elastic net method with the twenty-year data set, those three factors with the market factors would be selected.

\begin{table}[]
	\centering
	\caption{Selected Risk Factor with Strength}
	\begin{tabular}{llc|llc|llc}
		\hline
		\multicolumn{3}{c|}{Ten Year} & \multicolumn{3}{c|}{Twenty Yera} & \multicolumn{3}{c}{Thirty Year} \\ \hline
		Rank & Factor     & Strength & Rank   & Factor     & Strength   & Rank   & Factor     & Strength   \\ \hline
		1    & beta       & 0.749    & 1      & ndp        & 0.904      & 1      & salecash   & 0.857      \\
		2    & baspread   & 0.730    & 2      & salecash   & 0.902      & 2      & ndp        & 0.852      \\
		3    & turn       & 0.728    & 3      & quick      & 0.901      & 3      & quick      & 0.851      \\
		4    & zerotrade  & 0.725    & 4      & dy         & 0.897      & 4      & age        & 0.851      \\
		5    & idiovol    & 0.723    & 5      & lev        & 0.897      & 5      & roavol     & 0.850      \\
		6    & retvol     & 0.721    & 6      & cash       & 0.897      & 6      & ep         & 0.849      \\
		7    & std\_turn  & 0.719    & 7      & zs         & 0.896      & 7      & depr       & 0.848      \\
		8    & HML\_Devil & 0.719    & 8      & cp         & 0.894      & 8      & cash       & 0.847      \\
		9    & maret      & 0.715    & 9      & roavol     & 0.894      & 9      & rds        & 0.843      \\
		10   & roavol     & 0.713    & 10     & age        & 0.894      & 10     & currat     & 0.840      \\
		20   & UMD        & 0.678    & 28     & HML        & 0.874      & 38     & HML        & 0.811      \\
		24   & HML        & 0.672    & 76     & SMB        & 0.745      & 69     & SMB        & 0.721      \\
		87   & SMB        & 0.512    & 88     & UMD        & 0.703      & 95     & UMD        & 0.672      \\ \hline
	\end{tabular}
\end{table}
%But it worth notice that the market factor is consistently strong.
%The strength of market factor is close to unity all the time in the twenty year result, but decrease gradually when the t is growing %and n is decreasing.
%However, we also find that the market factor strength will decrease with the increase of observation and decrease of units.
%For instance, the market factor in the twenty year results, in general, it decrease from 0.98 to around 0.96, and it decreased %further to 0.9 in the thirty year estimation result.
%But not matter in which data set, the market factor is always the strongest factor.

%\begin{table}[]
%	\centering
%		\caption{Fama and French value and size factor, with Momentum Factor strength}
%	\begin{tabular}{l|ccc}
%		\hline
%		\hline
%		Factor & Ten Year Estimation & Twenty Year Estimation & Thirty Year Estimation \\ \hline
%				SMB (Size)    & 0.512               & 0.745                  & 0.721                  \\
%		HML (Value)    & 0.672               & 0.873                  & 0.811                  \\
%		UMD  (Momentum)  & 0.678               & 0.703                  & 0.672                  \\ \hline \hline
%	\end{tabular}
%\end{table}

We also pay attention to some famous factors, precisely the Farma-French size factor (Small Minus Big SMB), Fama-French Value factor (High Minus Low: HML) \cite{Fama1992} and the Momentum factor (UMD) \cite{Carhart1997}.
It is surprise that none of these three factors enter the top ten list for each data sets.
Except the HML factor from the twenty and thirty year data set has strength above 0.8, none of the other factors in any data set shows strength higher than 0.75.
The value factor SMB from the ten year data set only has strength 0.512, ranked no.87.

In order to see how factor strength evolve through the time, we decompose the thirty year data set into three small subsamples.
For each subsamples, it contains 242 companies (n = 242). 
And for each companies, we obtained 120 observations (t = 120). 
The results is present in the table (\ref{table:thirty_decompose}) and figure (\ref{figure:thirty_decompose}).

In general, we can conclude that for most of the factors, their strength gradually increased from the first decade (January 1988 to December 1997) to the second decade (January 1998 to December 2007), and then decreased in the third decade (January 2008 to December 2017).
This pattern can also be seen in the figure (\ref{figure:thirty_decompose}).
The drop of factor strength in the third decades has been regards as the main reasons why the ten-years data results shows a significantly weaker results than the twenty and thirty years data set.
%The strength has been dragged down by the data from 2007 to 2017.
%One possible explanation is that the Great Financial Crisis happened during the 2007 to 2008 has distorted the risk pricing mechanism of some factors.

%Also, from the simulation results (see appendix \ref{simulationtable}), we find that when the n < t, in other words, when the units is less than the observation, the overall bias will be negative.
%This indicates that the strength from the thirty-year data is underestimated.
%After taking this correction into consideration, we may have a more close results between the twenty and thirty years dataset estimations.
%For some factors does not follow this rule, we find that they usually have very weak strength in all three periods, for instance the std\_dolvol (volume of liquidity) factor, we can see that in all three periods, the strength has never excess the 0.5 threshold.

\subsubsection{Conclusion and Explanation}
From the factor strength prospect, we would expect that for different time period, we will have different candidate factors for the CAPM model.
For the ten-year data set, we would expected that only the market factor be useful, and therefore the elastic net method applied latter may only select the market factor.
If we use the twenty and thirty year data, we will have a longer list for potential factors, 62 factors from the twenty-year estimation and 45 from the thirty years has strength greater than 0.8.
Hence, we would expect the elastic net to select a less parsimonious model. 

In terms of this discrepancy, there are several potential explanation.
First if we consider the structure of our data set, we will find that the longer the time span, the less companies are included.
This is because the S\&P index will adjust the component, remove companies with inadequate behaviours, and add in new companies to reflect the market situation.
Hence, those 242 companies in the thirty year data set can be viewed as survivals after series of financial and economic crisis.\footnote{For instance, the dot-com bubble in the early 20th century, the 911 attack, and the 2008 Global Financial Crisis}
We would expect those companies will have above average performances, such as better profitability and administration, comparing with other companies. 

{\bf Notes: (But for how will those merits influence the factor's risk pricing ability is unclear for now)}

Another possible explanation is the Global Financial Crisis in 2008.
The financial market has been disturbed by this crisis, so therefore some mechanism may no longer working properly during that period.

We also need to notice that for some factors, their strength will decrease with the time.
For instance, the gma (gross profitability) factor and convind (convertible debt indicator) factor (see figure \ref{figure:thirty_decompose}) has consecutive strength decrease from the 1987-1997 period to 2007-2017 period.
And for most of the factors, their strength will decrease significantly from the 1997-2007 period to 2007-2017 period.




%Beside, from the simulation results (see appendix \ref{simulationtable}), we find that when we have a unbalanced panel (when n < t, the unit is smaller than the observation), we will have negative bias results.
%This indicates that the strength is underestimated in that scenario.
%Therefore, we could inference that the factor strength will be stronger than what we estimated if we use the thirty-year data.
%After taking this correction into consideration, we may have a more closer results between the twenty and thirty years estimation.

%From the estimation, we see the factor strength will change through time.
%And from the factor selection prospect, we would expect to have selecting different factors when using different dataset, there is no unanimity factors with regard of risk pricing.
%The decomposition of thirty years data shows that most factors' strength increases from the first decade (January 1988 - December 1997) to the second decade (January 1998 to December 2007), and then drops in the third decade (January 2008 to December 2017).
%But from the simulation result (see section \ref{MC} and appendix \ref{simulationtable}), we can see that when the observation amount is larger than the unit amount, i.e. when the t > n, we have negative bias, which indicates that the estimated strength $\hat{\alpha}$ is smaller than the true strength $\alpha$.
%In the thirty year data set, we have t = 360, and n = 242.
%Therefore, we can reasonable assume that the factor strength is under estimates.
%Consider such correction, the true strength of factor using data from the thirty year data to estimates may have similar strength as using twenty year data.

%\begin{table}[]
%	\centering
%	\caption{RMSE between different data set estimation result}\label{table:RMSE_strength}
%	\begin{tabular}{rl|r}
	%	\hline
%		\hline
%		& Pair           & Difference \\ \hline
%		1 & Ten : Twenty    & 0.254 \\
%		2 & Ten : Thirty    & 0.220 \\
%		3 & Twenty : Thirty & 0.06  \\ \hline \hline
%	\end{tabular}
%\end{table}

%We also use the formula:
%\[ Differnce  = \sqrt{\sum_{i = 1}^{145}(\widehat{\alpha_{xi}} - \widehat{\alpha_{yi}})^2}  \]
%to calculates the Root Mean Square Difference between any two different strength results (see table(\ref{table:RMSE_strength})).
%Here the $\widehat{\alpha_{xi}}$ and $\widehat{\alpha_{yi}}$ each represents the estimates factor strength of $i^{th}$ factor from different data sets. 
%The difference between twenty and thirty year data sets results is significantly smaller than the differences between ten-twenty and ten-thirty.




%\newpage
%\bibliographystyle{apacite}
%\bibliography{thesis.bib}
%\end{document}