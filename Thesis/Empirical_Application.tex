%\documentclass[12pt]{article}
%\usepackage{amsmath}
%\usepackage{mathptmx}
%\usepackage{setspace}
%\usepackage{amssymb}
%\usepackage{float}
%\usepackage{graphicx}
%\usepackage{booktabs}
%\usepackage{subfigure}
%\usepackage[margin=2.5cm]{geometry}
%\usepackage[title]{appendix}
%\usepackage{bm}
%\usepackage{tcolorbox}
%\usepackage{apacite}
%\onehalfspacing
%\usepackage{lineno}
%\linenumbers
%\usepackage{diagbox}

%\DeclareMathOperator*{\argmin}{arg\,min}
%\newtheorem{experiment}{Experiment}



%\usepackage{fontspec}
%\setmainfont{Times New Roman}
%\addtolength{\jot}{0.5em}
%\linespread{1.5}

%\begin{document}
	\section{Empirical Application}

	In this section, we introduced the data prepared for the empirical application, discuss the results of factor strength estimations. Then we apply the elastic net method introduced before, 
	\subsection{Data}\label{data}
	
In the empirical application part, we use the monthly U.S. stock return as the assets.
The companies are selected from Standard Poor (S\&P) 500 index component companies.\footnote{The data was obtained from the Global Finance Data, Osiris, and Yahoo Finance.}
We prepared three data sets for different time spams: 10 years (January 2008 to December 2017), 20 years (January 1998 to December 2017), and 30 years (January 1989 to December 2017).
Because of the components companies of the index are constantly changing, bankrupt companies will be moved out, and new companies will be added in.
Also, some companies does not have enough observation.
Therefore, for each of the datasets, the companies amount (n) are different, the dimensions of the data set is showing in the table (\ref{Data_set}) below.

\begin{table}[h]
		\caption{Data Set Dimensions}
			\label{Data_set}
	\begin{tabular}{c|ccc}
		\hline
		& Time Spam                    & Companies Amount (n) & Observations Amount (T) \\ \hline
		10 Years & January 2008 - December 2017 & 419                  & 120                     \\
		20 Years & January 1998 - December 2017 & 342                  & 240                     \\
		30 Years & January 1988 - December 2017 & 242                  & 360                     \\ \hline
	\end{tabular}
\end{table}
For the risk free rate, we use the one-month U.S. treasury bill return.\footnote{ The data was fetched from the Kenneth R. French website: http://mba.tuck.dartmouth.edu/pages/faculty/ken.french/}
For company i, we calculates the companies return at month t ($r_{it}$) use the following formula:
\begin{align*}
r_{it} = \frac{p_{i t} - p_{i t-1}}{p_{i t-1}}\times 100
\end{align*}
and calculate the excess return $x_{it} = r_{it} - r_{ft}$.
Here the $p_{it}$ and $p_{i t-1}$ are the company's close stock price at the first day of month t and t-1.
The price is adjusted for the dividends and splits.\footnote{The data is adjusted base on the Central for Research in Security Price (CRSP) method.}

With regard of the factors, we use 145 different risk factors from \citeA{Feng2020}.
The factor set also includes one market factor, represented by the difference between the average market return and risk free return.
The average market return is a weighted average return of all stocks in U.S. market, incorporated by CSRP.

\subsection{Factor Strength Analysis}

\subsubsection{Regression model for single security and two facotrs}
For the first part of the empirical application, we estimates the factor strength using the method discussed in the section \ref{strength}.
Precisely, we set the regression model base on the discussion of section \ref{strength_multi_estimation}.

\[  x_{it} = a_i + \beta_{im}(r_{mt} - r_{ft}) + \beta_{ij}f_{jt} + v_{it}  \]

where $x_{it}$ is the excess return of asset i at time t, which is pre-defined in the section \ref{data}.
$r_{mt} - r_{ft}$ represents the market factor, calculated by the difference between average market return and risk free return at the same time t.
$f_{jt}$ is the value of $j^{th}$ risk factor  at time t. 
Here $j = 1, 2, 3,\cdots 145$ . 
$\beta_{mt}$ and $\beta_{ij}$ are the factor loadings for market factor and risk factor, respectively.








%\newpage
%\bibliographystyle{apacite}
%\bibliography{thesis.bib}
%\end{document}