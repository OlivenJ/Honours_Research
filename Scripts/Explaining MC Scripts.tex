\documentclass[12pt]{article}
\usepackage{amsmath}
\usepackage{mathptmx}
\usepackage{setspace}
\usepackage{amssymb}
\usepackage{float}
\usepackage{graphicx}
\usepackage{booktabs}
\usepackage{subfigure}
\usepackage[margin=2.5cm]{geometry}
\usepackage{appendix}
\usepackage{bm}
\usepackage{tcolorbox}
\usepackage{apacite}
\onehalfspacing
\usepackage{lineno}
\linenumbers

\usepackage{fontspec}
\setmainfont{Times New Roman}
\addtolength{\jot}{0.5em}

\title{Explanation of the simulation scripts}
\author{Zhiyuan Jiang\\I.D:28710967}
\date{\today}

\begin{document}
\maketitle
The scripts follow the same idea as discuss before, I will explain how the scripts implied those ideas into ground.

The whole package used three packages beside base R
\begin{enumerate}
\item {\bf tidyverse}: for the data manipulation and store the data in tibble.
\item {\bf broom}: present the regression result in tidy way.
\item {\bf mvtnorm}: generate a multinomial distribution matrix.
\end{enumerate}

The first three variables t, n, and k (from line 8 to 15) represent the time length (mnth), the assets amount, and the factor amounts respectively. 

First we generate the constant term from a uniform distribution, with upper and lower bound been 0.5 and -0.5.  (line 17)

For simplicity, the error term is assumed to follow standard normal distribution.(line 20)

Then set the alpha, denotes the strength of factors beside market factor. 
Repeat the alpha k times, to match our k different factors, and for the last term of the strength vector, we append value 1 to reflect the fact that market factor has strength 1. (line 22-26)

Sigma here is a variance-covariance matrix of factors, we assume Sigma been a diagonal matrix with 1 in diagonal and 0 elsewhere.
This because the first simulation as a start point we assume each factor are mutual independent, so their covariances are 0 and variance are 1. 
Factor follows a multinomial distribution, with mean zero and Sigma variance. Line 30 use the function rmvnorm() from mvtnorm package to generate those vector.

I take out the market factor separately and pack it into the tibble format (line 35-39)

Since the factor strength will influence the amount of significant loading, line 41 calculate the number by apply $[n^{\alpha}]$, and store the result into a vector. This result will used when generating factor loading.

Line 45 calculate the adjusted critical value for the t-test.

The next big step is to generate the factor loading. (line 48 - 60)
Since all factor loadings are assumed to follow a uniform distribution, we generate a n$\times$ k+1 matrix with each element follows uniform distribution. However, since for each factor, the strength is different, and therefore the number of significant loading is different. So, base on the calculation result from previous lines, the function will randomly select some loading and assign them to zero. Only keep some non-zero loadings to reflect the factor strength.

From line 62 to line 74, I calculate the return and pack it with facto together into the tibble format, for the upcoming regression.

For the regression, first I run the model with one market factor. Here I use the map() function from purrr to run multiple regression in a efficient way. Same idea applied later when running the regression with two factors. 
.


\end{document}
