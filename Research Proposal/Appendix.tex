%\documentclass[12pt]{article}
%\usepackage{amsmath}
%\usepackage{mathptmx}
%\usepackage{setspace}
%\usepackage{amssymb}
%\usepackage{float}
%\usepackage{graphicx}
%\usepackage{booktabs}
%\usepackage{subfigure}
%\usepackage[margin=2.5cm]{geometry}
%\usepackage[title]{appendix}
%\usepackage{bm}
%\usepackage{tcolorbox}
%\usepackage{apacite}
%\onehalfspacing

%

%\begin{document}

		\section{Monte Carlo Design}\label{MC}
	In this section, I will introduce the baseline design setting of the Monte Carlo Simulation and provides a preliminary result of the simulation.
	
		\subsection{Monte Carlo Design}
	Before start using the real data, we want to study the property of $\alpha$ by running Monte Carlo simulation and in this section, I will introduce the basic simulation design.
	
	Consider the following model with stochastic error:
	
	\[  r_{it} = f_1(\bar{r_{t}} - r_f) + f_2( \theta_{i}x_{t}) +\epsilon_{it} \tag{2}\label{baselineequation} \]
	
	In this Monte Carlo simulation, we consider a dataset has $i = 1, 2,\dots, n$ different assets, with $t= 1, 2,\dots, T$ different observations. 
	$j = 1, 2, \dots, k$ different factors and one market factors are included in the simulation. 
	
	
	$f_1(\cdot)$ and $f_2(\cdot)$ are two different functions represent the unknown mechanism of market factor and other factors in pricing asset risk.
	$(\bar{r_{t}}- r_f) $ is the market return, calculated from market or index return $\bar{r_{t}}$ minus risk free return $r_f$. 
	$r_{it}$ is the stock return, $\theta_{jt}$ denotes factors other than market factors and $\beta_{ij}$is the corresponding factor loading. 
	$\epsilon_{it}$ is random error with structure can be defined in different designs.
	Notice that the $\beta_{ij}$ will be influenced by each factor's strength $\alpha_j$, where we have $\alpha$ as defined in section $\ref{strength}$. 
	And for each factor, we assume they follow a multinomial distribution with mean zero and a $k\times k$ variance-covariance matrix $\Sigma$. 
	The diagonal of matrix $\Sigma$ indicates the variance of each factor, and the rest represent the correlation among all $k$ factors.
	In this model, we can control several parts to investigates different scenarios of the simulation:
	\subsection{Baseline Design}\label{base}
	Follow the model (\ref{baselineequation}), we assume both $f_1(a)$ and $f_2(a)$ are linear function:
	\begin{align*}
	f_1(a ) &= c_{i} +\beta_{} a\\
	f_2(a) &=a
	\end{align*}
	Therefore, the model with single factor can be write as:
	\[   r_{it} = c_i +   \theta_{i}x_{t} +\epsilon_{it}   \]
	
	The constant$c_i$ is generated from a uniform distribution $\mathnormal{U}[-0.5, 0.5]$.
	$\theta_{i}$ is the factor loading, and $x_t$ is factor with strength $\alpha_{x}$.
	To generate factors loading, we employed a two steps strategy.
	First we generate a whole factor loadings vector $\mathbf{\theta_i} = (\theta_{i1}, \theta_{i2} \cdots, \theta_{in})$,
	All elements of the vector follows $\mathnormal{IIDU}(\mu_{\theta} -0.2, \mu_{\theta} + 0.2)$. 
	The $\mu_{\theta}$ has been equalled to 0.71 to ensure all values apart from zero. 
	After generating the vector, we randomly selected $[n^{\alpha_{x}}]$ elements from $\mathbf{\theta_i}$ to keep their value and set the other elements to zero. 
	This step ensures the loading reflects the strength of each factor. 
	For the stochastic error term, in this baseline design, we assume it follows a Standard Gaussian distribution, but we can easily extend it into a more complex form.
	
	Follow the same idea, we also construct a two factor model:
	\[   r_{it} = c_i + \lambda x_m  + \theta_{i}x_{t} +\epsilon_{it}   \]
	Here the $x_m$ is the market factor which assumably  has strength $\alpha_{m} = 1$. 
	$\mathbf{\lambda}$ is the market factor loading as a vector with all elements different from zero. 
	
	For each of the those different models, we consider the $T = \{120, 240, 360\}$, $n =\{100, 300, 500\} $.
	The market factor will have strength $\alpha_m = 1$ all the time, and the strength of the other factor in two factor model will be $\alpha_{x} = \{0.5, 0.7, 0.9,1\}$. For every setting, we will replicate 500 times independently, all the constant $c_i$ and loading $\theta_i$ will be re-generated for each replication.
\newpage

	\section{Simulation Result Table}\label{simulationtable}
\begin{table}[!hbt]
		\caption{Simulation result of single factor model}\label{simutable1}
	\label{onefactortable}
	\centering
	\begin{tabular}{l|ccc|ccc}
		\hline
		\hline
		& \multicolumn{6}{c}{Single Factor}                                  \\
		\hline
		& \multicolumn{3}{c}{Biass}   \vline    & \multicolumn{3}{c}{MSE}  \\
		\hline 
		\multicolumn{7}{c}{$\alpha = 0.5$}         \\
		\hline
		\diagbox{n}{T}       & 120   & 240   & 360                  & 120   & 240   & 360      \\
		\hline
		100                  & 0.194 & 0.188 & 0.199                & 0.050 & 0.047 & 0.053    \\
		300                  & 0.224 & 0.224 & 0.226                & 0.062 & 0.062 & 0.062    \\
		500                  & 0.229 & 0.237 & 0.225                & 0.064 & 0.067 & 0.062    \\
		\hline
		\multicolumn{7}{c}{$\alpha = 0.7$}         \\
		\hline
		100                  & 0.093 & 0.090 & 0.092                & 0.013 & 0.012 & 0.013    \\
		300                  & 0.101 & 0.098 & 0.101                & 0.014 & 0.008 & 0.014    \\
		500                  & 0.101 & 0.107 & 0.100                & 0.015 & 0.015 & 0.014    \\
		\hline
		\multicolumn{7}{c}{$\alpha = 0.9$}         \\
		\hline
		100                  & 0.023 & 0.022 &0.023                 & 0.001 &0.001&0.001     \\
		300                  & 0.023 &0.023&0.024                 & 0.001 &0.001&0.001     \\
		500                  & 0.023 &0.023  &0.024                &  0.001&0.001  &0.001     \\
		\hline
		\multicolumn{7}{c}{$\alpha = 1.0$}         \\
		\hline
		100                  & 0.000 & 0.000 & 0.000                & 0.000 & 0.000 & 0.000    \\
		300                  & 0.000 & 0.000 & 0.000                & 0.000 & 0.000 & 0.000    \\
		500                  & 0.000 & 0.000 & 0.000                & 0.000 & 0.000 & 0.000    \\
		\hline 
		\hline
	\end{tabular}
\end{table}


\begin{table}[!hbt]
		\caption{Simulation result of two factor model}\label{simutable2}
	\label{twofactor label}
	\centering
	\begin{tabular}{l|ccc|ccc}
		\hline
		\hline
		& \multicolumn{6}{c}{Two Factor}                                  \\
		\hline
		& \multicolumn{3}{c}{Biass}   \vline    & \multicolumn{3}{c}{MSE}  \\
		\hline 
		\multicolumn{7}{c}{$\alpha_x = 0.5$, $\alpha_m = 1.0$}         \\
		\hline
		\diagbox{n}{T}       & 120   & 240   & 360                  & 120   & 240   & 360      \\
		\hline
		100                  & 0.221 & 0.219 & 0.221                & 0.050 & 0.049 & 0.050    \\
		300                  & 0.253 & 0.253 & 0.253                & 0.042 & 0.064 & 0.065    \\
		500                  & 0.268 & 0.266 & 0.269                & 0.072 & 0.071 & 0.071    \\
		\hline
		\multicolumn{7}{c}{$\alpha_x = 0.7$, $\alpha_m = 1.0$}         \\
		\hline
		100                  & 0.100 & 0.101 & 0.100                & 0.010 & 0.010 & 0.010    \\
		300                  & 0.113 & 0.113 & 0.112                & 0.013 & 0.013 & 0.013    \\
		500                  & 0.118 & 0.118 & 0.119                & 0.014 & 0.014 & 0.014    \\
		\hline
		\multicolumn{7}{c}{$\alpha_x = 0.9$, $\alpha_m = 1.0$}         \\
		\hline
		100                  & 0.024 & 0.023 & 0.024                & 0.001 & 0.001 & 0.001    \\
		300                  & 0.025 & 0.025 & 0.025                & 0.001 & 0.001 & 0.001    \\
		500                  & 0.026 & 0.025 & 0.025                & 0.001 & 0.001 & 0.001    \\
		\hline
		\multicolumn{7}{c}{$\alpha_ = 1.0$, $\alpha_m = 1.0$}         \\
		\hline
		100                  & 0.000 & 0.000 & 0.000                & 0.000 & 0.000 & 0.000    \\
		300                  & 0.000 & 0.000 & 0.000                & 0.000 & 0.000 & 0.000    \\
		500                  & 0.000 & 0.000 & 0.000                & 0.000 & 0.000 & 0.000    \\
		\hline 
		\hline
	\end{tabular}
\end{table}

%\end{document}