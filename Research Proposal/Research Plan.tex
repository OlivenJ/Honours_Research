\documentclass[12pt]{article}
\usepackage{amsmath}
\usepackage{mathptmx}
\usepackage{setspace}
\usepackage{amssymb}
\usepackage{float}
\usepackage{graphicx}
\usepackage{booktabs}
\usepackage{subfigure}
\usepackage[margin=2.5cm]{geometry}
\usepackage[title]{appendix}
\usepackage{bm}
\usepackage{tcolorbox}
\usepackage{apacite}
\onehalfspacing
\usepackage{lineno}
\linenumbers
\usepackage{diagbox}


\usepackage{fontspec}
\setmainfont{Times New Roman}
\addtolength{\jot}{0.5em}
\linespread{1.5}

\begin{document}
\begin{titlepage}
\begin{center}
\vspace*{1cm}
\Huge
\textbf{Factor Selection and Factor Strength}

\vspace{0.5cm}
\LARGE
An Application to U.S. Stock Market Return\\
\Large
Research Plan 

\vspace{1.5 cm}
\textbf{Zhiyuan Jiang\\I.D:28710967}
\vfill
 Supervisor : Dr Natalia Bailey\\\hspace{30mm} Dr David Frazier
 \vspace{0.8cm}
 
\Large
\today
\end{center}
\end{titlepage}
\newpage	
\tableofcontents
\newpage
	\section{Introduction and Motivation}
Capital Asset Pricing Model (CAPM) (\citeA{Sharpe1964}, \citeA{Lintner1965}, and \citeA{Black1972})  has greatly changed the way people measure the relationship of asset's risk and return.
Because the original model only contains one market factors, scholars after them are always trying to add new factors to create a multi-factor extension CAPM.
The most famous example would be the three-factor model  introduced by \citeA{Fama1992}, they added the size and value factors into the original model.
\citeA{Carhart1997} add the momentum factor to the three-factor models and thereby introduce the four-factor model. 
Researchers nowadays are  still trying to find a sparse factor model. 
For instance, \citeA{Kelly2019} using the IPCA method construct a six-factor model and claim that their model outperforms all other existing sparse factor models.  

In the other hand, the factor is abundant. 
\citeA{Harvey2019} had collected over 500 factors published in the top financial and economic journal,  and the growth of new factors speed up since 2008. 
In his 2011 presidential address, \citeauthor{Cochrane2011} coined the term "factor zoo" to describe factor model is facing: researchers and practitioners are having too many options to help them pricing the risk.

The abundance of factors does not provides a rosy prospect to factor model.
\citeA{McLean2016} found that once a new factor is published, the probability of using this factor to successfully predict the stock result will drop significantly.
In recent research, \citeA{Hou2018} fail to replicate over 60\% result relates to the factor model, which had been published in the financial and economic journal. 
The reliability of those factors has been doubt. 

What's more, \citeA{Harvey2015} argue that the current threshold for the test statistic is too low, and this helps some factor yield a significant result purely out of luck. 

The existence of some seemly strong, but de factor weak or even useless factor jeopardized the result of many research.
\citeA{Kan1999} found that if the Fama-MacBeth two-stage regression\cite{Fama1973} want to  yield meaningful and correct result, the structure of the model must be correctly identified. 
But involving some factor, which has no correlation with the cross-section return, will cause miss-identification and thereby mislead the researcher to have the wrong conclusion. 
\citeauthor{Kleibergen2009}'s 2009 paper also argues that for FM regression, when the factor's loading is small, or the size of cross-section assets is large, the result of regressions will be incorrect.
Similar results with regards to weak or useless factor were also mentioned by other scholars. 
(see \citeA{Kleibergen2015}, \citeA{Gospodinov2017}, and \citeA{Anatolyev2018})
Due to all those reasons, \cite{Cochrane2011} post the question: "Which characteristic really provide independent information about average return. Which are subsumed by others ?"

To answer this question, lot's of scholars and research had applied various methods to pick the best factors from the factor zoo.
Such as \citeA{Harvey2017} introduced a bootstrap method.  
\citeA{Pukthuanthong2019} developed a protocol, trying to capture the real priced rick facto.
Some other scholars are trying to use machine learning methods to reduce the potential candidates of useful factors, one stream of them are using a shrinkage and subset selection method like Lasso \cite{Tibshirani1996} and its deviation to find the suitable factors (See, \citeA{Rapach2013}, \citeA{Feng2019}, \citeA{Freyberger2020}).

But for factors, especially in the high-dimension, correlation is common. 
\citeA{CoChrane2005} points out that the correlation between factors will drag the ability of using risk premium to infer factors.
ne main drawback of Lasso regression is that, when Lasso is facing a group of variables which are correlated with each other, Lasso will only select on among all and does not care about which one it picks up. 
Therefore, a new technique called Elastic Net is created \cite{Zou2005}, trying to deal with the selection of correlated variables. 


For this project, we are trying to deal with the classical problem: what factors can have significant contributions to explain the asset risk and return relationship under the CAPM framework.
And we go a step forward, take the correlation among all factors into the account. 
But before applying the Elastic Net method, we want to first investigate all those factor's strength.
Our interest is focused on first determine which factors have enough power to help us solve the risk pricing problem. 
Then, from this pre-determined relatively small factor group, we applied the Elastic Net method, trying to find out the most appropriate factors to help us form the multi-factor CAPM model.














	\section{Related Literature}

	This project is built on a series of  literature about trying to use different factors to price risk. 
	The Capital Asset Pricing Model developed by \citeA{Sharpe1964}, \citeA{Lintner1965}, and \citeA{Black1972} completely change the way people measure the relationship between risk and return. 
	For the initial CAPM models, only market factor been included, \citeA{Fama1992} develop the model into three-factors, \citeA{Carhart1997} added the momentum factors and created the four-factor models.
	Base on their three-factor model, \citeA{Fama2015} extend the model to five-factors, and recent research created a six-factors model \cite{Kelly2019}.

\citeA{Kan1999} first illustrate the negative influence of including weak factor when applying FM two-stage regression \cite{Fama1973}. \citeA{Kleibergen2009}, and \citeA{Kleibergen2015} point out how a weak factor inside the multi-factor CAPM models would disguise the fact that some structure does not exist. 
\citeA{Gospodinov2017} show how the involving of a spurious factor will distort statistical inference of parameters.
\citeA{Anatolyev2018} studied the behaviours of the model with the presence of weak factors under asymptotic settings.
These findings provide us with the motivation to investigates the method of selecting strong factors. 
	
	
	This project also relates to some researches effort to identify useful factors from the factor zoo. 
\citeA{Harvey2015} exam over 300 factors published on journals, presents that the traditional threshold for a significant test is too low for newly proposed factor, and they suggest to adjust the p-value threshold to around 3. 
\citeA{McLean2016} exams 97 different factors, find their ability of predict out of sample return declined, and thereby cast doubt on the reliability of some factors' pricing ability. 
Some other method like bootstrap \cite{Harvey2017} or Bayesian procedure \cite{Barillas2018} were tried to find out useful factors from a large group of homogenous competitors. 
\citeA{Pukthuanthong2019} discovered this problem from another prospect, they developed a protocol to identify whether a factor is indeed a priced factor or not.

This project also benefits from and will contribute to emerging literature concerning applying machine learning on selecting characteristics on predicting cross-section stock returns. 
\citeA{Gu2020} elaborate on the benefit and advantages of using emerging machine learning algorithms in asset pricing. 
The advantages including more accurate predict result, and better efficiency.
Various machine learning algorithms have been adopted on selecting factors for the factor model, especially in recent years.
\citeA{Lettau2020} applying Principle Components Analysis on investigating the latent factor of model. 
Lasso method, since it's ability to select features, is popular in the field of the factor selection.
\citeA{Rapach2013} applying the lasso regression, trying to find some characteristics from a large group to predict the global stock market's return.
\citeA{Feng2019} used the double-selected Lasso method \cite{Belloni2014}, a grouped lasso method \cite{Huang2010} is used by \citeA{Freyberger2020} on picking factors from a group of candidates. 
\citeA{Kozak2020} used a Bayesian-based method, combing with both Ridge and Lasso regression, argues that the factor sparse model is ultimately futile. 
The correlation underneath the factors, however, inevitably affect the choice of model. 
Due to the problem that Lasso model can not select from correlated factors, this project will apply the method called elastic net \cite{Zou2005} to select factors from a group of pre-choice factor candidates. 

	\section{Methodology}
			\subsection{Factor Strength}\label{strength}
Capital Asset Pricing Model (CAPM), especially the model‘s multi-factor extension, has become the benchmark when studying the relationship between risk and return. 
Consider the following multi-factor  models with stochastic error term $\varepsilon_{it}$:
\[   r_{it} = c + \beta_0 x_m + \mathbf{\beta_j}^{\prime}\mathbf{x_j} + \varepsilon_{it}  \tag{1}\label{2CAPM} \]
In the left-hand side, we have $r_{it}$ denotes the return of  security i at time t.
In the other side, two vectors $\mathbf{\beta_j }= \{\beta_1, \beta_2, \cdots, \beta_J\}$ and $\mathbf{x_j} = \{x_1, x_2, \cdots, x_J\}$,  represents J different factor loadings and J different factors respectively.
$x_m$ is the market factor, which usually represents by the difference between market expected return and the risk -free return.
$\beta_0$ denotes the market factor loading.
In general, CAPM and it's multi-factor extension divided an asset risk into two parts, the systematic part which can be captured by the market factor $x_m$, and the asset-specific idiosyncratic part which is demonstrated by different characteristics from the factor vector $\mathbf{x_j}$. 

In order to reduce the dimension of the factors when using the Elastic Net, we use factor strength as a criterion to select some  factors in ahead.
The factor strength of factor $x_j$ as $\alpha_j$ from \citeA{Pesaran2019}, and \citeA{Bailey2020} is defined as the pervasiveness of a factor.
The stronger a factor is, the more loading that factor generates against different assets will be significantly different from zero. 
For instance, if a factor has strength $\alpha$, and base on the Fama-MacBeth two-stage regressions' first-stage regression, use this factor against N different assets, we will have $[N^\alpha]$ factor loading statistically significantly different from zero, here $[\cdot]$ operator will take the integer from the result:

\begin{align*}
\beta_j \neq 0, j &= 1, 2, 3, \cdots, [N^\alpha]\\
\beta_j = 0, j &= [N^\alpha ]+1 ,[N^\alpha]  +2, [N^\alpha] +3, \cdots, N
\end{align*}


\subsection{Elastic Net}

Elastic net  is a factor selection model introduced by \citeA{Zou2005}.
The primary feature of the elastic net is that it has two penalty terms, combined the advantages of both ridge regression and lasso regression.
Consider the model (\ref{2CAPM}) with only one random factor $x_j$
Assume that we are using the OLS to estimates the factor loading $\beta_j$
OLS will trying to find $\beta_j$which has the smallest residual sum of squares. 
\[  \hat{\beta}_j  = argmin\{\sum_{i = 1}^{N}(r_{it} - \mathbf{\beta_j }\mathbf{x_j})^2\} \]
The elastic net, however, implies two different penalty terms when estimating the loading $\beta_j$.
	\[   \hat{\beta}_j  = argmin\{\sum_{i = 1}^{N}(r_{it} - \mathbf{\beta_j }\mathbf{x_j})^2 + \lambda_1\mathbf{\beta_j}^2  + \lambda_2|\mathbf{\beta_j}|  \label{ENcriterion} \tag{2}   \}    \]

The main advantage of elastic net, comparing with other factor selection method such as lasso, in this project is that elastic net can handle the problem of factor's correlation.
Unlike lasso, the elastic net can choice factor properly from a group of related candidates. 






\section{Preliminary Result}
In current stage, we have only studied the property of factor strength $\alpha$ under finite sample scenario.
In purpose of this, we have designed and applied a Monte Carlo Simulation.
The design details and result table can be seen at the Appendix \ref{MC} and Appendix \ref{simulationtable} 

To measure the goodness of simulation, we calculate the difference between the estimated factor strength and assigned factor strength as bias.
Base on the bias,, we also calculated the Mean Squared Error (MSE) for each setting.

From the result, we can easily find out that the error converge to zero when the strength $\alpha$ increases. 
When the $\alpha _x= 1$, we obtain the unbiased $\hat{\alpha}_{x}$

In the other hand, when the $\alpha$ is at a relatively low level, the estimation result will tend to overestimate the strength, and the level of overestimation decrease with the actual strength increase.

			\section{Further Plan}
For the next step of this project, we will start the empirical analyse. 

We will use companies return from Standard \& Poors (S\&P) 500 index as assets, to exam factors from \citeA{Harvey2017}'s factor list. 
The time span will be 30 years. 

For the purpose of evaluation the selecting factors, we are planing using Out of Sample (OOS) method to predict the future return, and therefore exam how good those selected factors are. 


\newpage
\bibliographystyle{apacite}
\bibliography{proposal.bib}

\newpage
\appendix

%\documentclass[12pt]{article}
%\usepackage{amsmath}
%\usepackage{mathptmx}
%\usepackage{setspace}
%\usepackage{amssymb}
%\usepackage{float}\documentclass[12pt]{article}
\usepackage{amsmath}
\usepackage{mathptmx}
\usepackage{setspace}
\usepackage{amssymb}
\usepackage{float}
\usepackage{graphicx}
\usepackage{booktabs}
\usepackage{subfigure}
\usepackage[margin=2.5cm]{geometry}
\usepackage[title]{appendix}
\usepackage{bm}
\usepackage{tcolorbox}
\usepackage{apacite}
\onehalfspacing
%\usepackage{lineno}
%\linenumbers
\usepackage{diagbox}
\usepackage{longtable}
\usepackage{pdflscape}
\usepackage{rotating}
%\usepackage{subcaption}
%\usepackage{floatrow}
\usepackage{fontspec}


\DeclareMathOperator*{\argmin}{arg\,min}
\newtheorem{experiment}{Experiment}



\usepackage{fontspec}
\setmainfont{Times New Roman}
\addtolength{\jot}{0.5em}
\linespread{1.5}



\begin{document}
	\begin{titlepage}
\begin{center}
\vspace*{1cm}
\Huge
\textbf{Factor Selection and Factor Strength}

\vspace{0.5cm}
\LARGE
An Application to U.S. Stock Market Return\\
\Large
Research Plan 

\vspace{1.5 cm}
\textbf{Zhiyuan Jiang\\I.D:28710967}
\vfill
 Supervisor : Dr Natalia Bailey\\\hspace{30mm} Dr David Frazier
 \vspace{0.8cm}
 
\Large
\today
\end{center}
\end{titlepage}
	
	\tableofcontents
	\newpage
	\listoffigures
	\listoftables
	\newpage

\chapter{Introduction and Motivation}
Capital Asset Pricing Model (CAPM) \cite{Sharpe1964, Lintner1965, Black1972} introduces a risk pricing paradigm.
By incorporating factors, the model divides the uncertainty of an asset's return into two parts: systematic risk part and asset specified idiosyncratic risk part.
The systematic risk is captured by the market factor, which is represented by the difference between the average market return and the risk-free return of the market.
Different risk factors contributed to price the idiosyncratic risk of different assets.
Researchers (see \citeNP{Fama1992, Carhart1997, Kelly2019}) have shown that by adding different risk factors into the model, CAPM can more precisely price the idiosyncratic risk.
Because of this, identifying unique risk factors has become an important topic in finance.
Numerous researchers have contributed to this field, and the direct result is an explosive growth of different risk factors.
 \citeA{Harvey2019} have documented and categorised over 500 factors from papers published in the top financial and economic journals, and they find the growth of new factors has sped up since 2008. 
 
The high dimension factor group makes it hard for practitioners to find suitable factors when constructing the multi-factor CAPM model.
And because of the dimensionality and computational burden, some traditional variable selection method like stepwise selection will not be applicable.
When looking inside the factor group, we will find that some factors will only have a very weak or even no correlation with the asset's return.

This is because of the multiple-testing/data mining problem researchers will encounter when they trying to identify new factors.
The problem of multiple-testing will cause a false positive significant test results for factors, and mislead we to believe they can explain asset's risk.
\citeA{Hou2018} argues that because researchers did not take the multiple testing into consideration when discovering new factors, a large portion of published factor results can not been replicated.
Some evidences (see \citeNP{Kan1999, Kleibergen2015}) have pointed out that, if the CAPM contains factor that has weak or even no correlation with the asset's return, the CAPM estimation will be distorted.
Therefore, how to identify risk factors that can provide independent information about average return and risk become a crucial question in the field of finance and asset pricing \cite{Cochrane2011}.

In order to answer this problem, many scholars applied various methods to identify factors that can independently provide fresh information to explain risk-return relationship from a large existed factor pool.
For instance, \citeA{Harvey2017} provided a bootstrap method to adjust the threshold of factor loading's significant test, trying to exclude some falsely significant factor caused by multiple-test problem.
In recent years, some other scholars are using machine learning methods to identify factors and eliminate redundant factors from a group of candidates.
One stream of them has used a shrinkage and subset selection method called Lasso \cite{Tibshirani1996} and the variations of it to find suitable factors.
An example of such application is made by \citeA{Rapach2013}.
They applied the Lasso regression, trying to find some characteristics from a large group, and then constructed a correspond CAPM model to predict the global stock market's return.

However, there is an additional challenge.
Factors, especially in high-dimensions, are usually highly correlated.
\citeA{Kozak2020} point out that when facing a group of correlated factors, Lasso will only pick several highly correlated factors, seemly at random, and then ignore the other and shrink them to zero. 
In other words, Lasso fails to handle the issue of correlated factors appropriately, because it can not distinct factors with strong correlation.

This leads to the main empirical question in this paper: how to select useful factors from a large group of possibly highly correlated candidates.
We address this problem from two prospectives.

On the one hand, we employ a new idea called factor strength, trying to reduce the dimension of candidate factors by allocating them into smaller subgroups base on their estimated strength.
The concept of factor strength is introduced by \citeA{Pesaran2019}.
They developed the idea to assess the significance of each factor.
By measuring how many non-zero coefficient, or refer it as loadings in financial literature, a factor can generate for different assets, we can obtain the strength of every risk factor.
With the factor strength, we can break down the high dimension group of candidate factors into small subgroups.
It also provides us with a standard to evaluate the risk pricing performance of factor, enable us compare the risk-pricing ability across factors.
So, we can use the strength as reference to evaluates how the factor selection method performed.

On the other hand, we use another variable selection method called Elastic Net \cite{Zou2005} to select factors from each subgroups.
With regard of the first approach, \citeA{Bailey2020} provide a consistent estimator for the factor strength, and we will use this method to examine the strength of each candidate factor.
Under the second approach, unlike Lasso, elastic net contains an extra penalty term, which enables it to avoid the problem of handling correlated features.
This trait makes Elastic net fit for our purpose.
We will assess and compare the methods in their selection of risk factors.

The rest of the thesis is organized as follows.
In chapter \ref{Literature}, we go through some literatures relate with the CAPM model and methods about factor selection.
Then in chapter \ref{strength}, we will provide a detailed description of the concept of factor strength and the estimation method.
In chapter \ref{MC}, we set up a simple Monte Carlo simulation experiment to examine the finite sample properties of the factor strength estimator.
Chapter \ref{Empirical:factor_strength} includes the empirical application regarding the factor strength, where we estimate the strength of each risk factors.
We introduce and apply the Elastic net approach, alongside Lasso to select factors in chapter \ref{Empirical:Elastic_net}.
Finally, we provides the conclusion and further discussion in chapter \ref{Conclusion}.


	\chapter{Related Literature}\label{Literature}
This project is built on contributions to the field of asset pricing.
First formulated by \citeA{Sharpe1964}, \citeA{Lintner1965}, and \citeA{Black1972}, the Capital Asset Pricing Model (CAPM) builds up the connection between expected asset return and the risk.
The original CAPM model only contains the market factor, which is denoted by the difference between average market return and risk-free return.
Then\citeA{Fama1992} extend the model to contain size factor (SMB) and the value factor (HML).
This three-factor models became popular in the finance industry.
\citeA{Carhart1997}, base on the Fama-French three factors model, added the momentum factor and makes it a new standard of factor pricing model.
Some recent researches are attempting to extend the factor model even further.
For instance, \citeA{Fama2015} create a five factors model base on their 1995 works by adding an investment factor and a profitability factor.
They also created a six-factor model \cite{Fama2018}, by adding a momentum factor of their on version base on the five-factor models.
\citeA{Kelly2019} proposed a new method named Instrumented Principle Component Analysis (IPCA) which can identify latent factor structure.
They applied the IPCA and constructed a six-factor model, claimed their six-factor model outperforms most of the sparse factor models, such as the five-factor models published by Fama and French in 2015.

In terms of assessing the strength of risk factors, this thesis also relates to papers discussing factors that have no or weak correlation with assets' return under the paradigm of the CAPM model.
The Fama-MacBeth two-stage regression (FM two-stage regression) introduced by \citeA{Fama1973} is a standard method when trying to estimates the CAPM and its multi-factor extension. 
\citeA{Kan1999} found that the test-statistic of FM two-stage regression will inflate when incorporating factors which are independent of the cross-section return.
Therefore, when factors with no pricing power were added into the model, those factors may have the chance to pass the significant test falsely.
\citeA{Kleibergen2015} found out that even when some factor-return relationship does not exist, the r-square and the t-statistic of the FM two-stage regression would become in favour of the conclusion of such structure presence. 
\citeA{Gospodinov2017} showed how the addition of a spurious factor will distort the statistical inference of parameters, and misleads the researchers to believe that they correctly specified the factor structure, even when the degree of misspecification is arbitrarily large.
Besides, \citeA{Anatolyev2018} studied the behaviours of the model with the presence of weak factors under asymptotic settings, and they find the regression will lead to an inconsistent risk premia estimation result.
To address the problem of misspecified factor-return relationship, \citeA{Gospodinov2014} proposed a factor selection procedure, which bases on their statement, can eliminates the falsely presented factor robustly, and restores the inference. 
	
Finally, of interest in this thesis is the large dimension of potential factors.
\citeA{Harvey2019} documented over 500 published risk factors, and they indicated that more factors are discovered every year.
Among all those risk factors, \citeA{Hou2018} tried to replicates 452 of them, and they find only 18\% to 35\% factors are reproducible.
For the reasons and findings above, this thesis also borrows from researchers that identify useful factors from a group of potential factors.
\citeA{Harvey2015} examine over 300 factors published in journals, presents a new multi testing framework to exam the significance of factors.
And they claim that a higher hurdle for the t-statistic is necessary when examining the significance of newly proposed factors.
Some other methods are also developed to selecting factors.
\citeA{Barillas2018} introduces a Bayes test procedures.
It enables researchers to compare the probabilities of a collection of potential models, which can be constructed after giving a group of factors.
In order to identify factors risk-pricing ability, \citeA{Pukthuanthong2019} defined several criteria for "genuine risk factor", and based on those criteria introduced a protocol to examine whether a factor is associated with the risk premium.

Once the factor strength is identified, the thesis will attempt to reconcile empirically the factor selection under machine learning techniques and the factor strength implied by the selection.
\citeA{Gu2020} elaborate on the advantages of using emerging machine learning algorithms in measuring equity risk premiums.
They obtained a higher predictive accuracy in measuring risk premium and demonstrated large economics gains using investment strategy base on the machine learning forecast results.
In recent years, machine learning algorithms have become popular in the finance studies, and various methods are adopted when selecting factors for the factor model.
\citeA{Lettau2020} apply Principle Components Analysis (PCA) when investigating the latent factor of the model. 
Lasso, been innovated by \citeA{Tibshirani1996}, is a popular algorithm which can eliminate redundant features. 
The derivation of Lasso has become increasingly popular in the factor selection.
For example, \citeA{Feng2019} used the double-selected Lasso method \cite{Belloni2014}, and \citeA{Freyberger2020} used a grouped lasso method \cite{Huang2010} when picking factors from a group of candidates. 
\citeA{Kozak2020} arguing that the sparse factor model is ultimately futile by using a Bayesian-based method. 
They constructed their estimator similar to the ridge regressor, but instead of putting the penalty on the sum of squared of factor coefficients, they impose the penalty base on the maximum squared Sharpe ration implied by the factor model.
They also augmented their Bayesian based estimator with extra $L^1$, created a method,  similar but different to the elastic net algorithm which will be employed by our project. 

\chapter{Factor Strength}\label{strength}
The concept of factor strength employed in this project comes from \citeA{Bailey2020}, and it was first introduced by \citeA{Bailey2016}.
They defined the strength of factor from the prospect of the cross-section dependences of a large panel and connect it to the pervasiveness of the factor, which is captured by the factor loadings.
In a separate paper, \citeA{Bailey2019} extended the method by loosening some restrictions and proved that their estimation can also be applied on the residuals of regression result.
Here, we focus on the case of observed factor, and use the method of \citeA{Bailey2020} in this project.

\section{Definition}\label{strength_definiton}
Consider the following multi-factor model for n different cross-section units and T time-series observations with k  factors.

\[  x_{it} = a_{i}+  \sum_{j=1}^{k}\beta_{ij}f_{jt} + \varepsilon_{it} \tag{1}\label{definition_model} \]

In the left-hand side, we have $x_{it}$ denotes the cross-section unit i at time t, where $i = 1, 2, \cdots, n$ and $t = 1,2, \cdots T$.  
In the other hand side, $a_{i}$ is the constant term.
We use $f_{jt}$ of $j = 1, 2, \cdots k$ to represent the factors included in the model, and $\beta_{ij}$ to represent the corresponding factor loadings.
For the error term, we use $\varepsilon_{it}$.

The concept of factor strength relates to how many non-zero loadings correspond to a factor.
More precisely, for a factor $f_{jt}$ with n different factor loading $\beta_{ij}$ and strength $\alpha_j$, we assume that:

\begin{align*}
|\beta_{ij}| &> 0\quad i = 1, 2,  \dots, [n^{\alpha_j}]\\
|\beta_{ij}| &= 0 \quad i = [n^{\alpha_j}] + 1, [n^{\alpha_j}] +2 ,\dots, n
\end{align*}

%The $\alpha_j$ represents the strength of factor $f_{jt}$ and $\alpha_j \in [0,1]$.
All factor strength, by definition, should fall in the interval of $[0,1]$.
In this case, the strength $\alpha_j$ represents that the first $[n^{\alpha_j}]$ loadings are all significantly different from zero.
Here $[\cdot]$ is defined as the integer operator, which will only take the integer part of the inner value.
%If a factor has strength $\alpha_j$, we will assume that the first $[n^{\alpha_j}]$ loadings are all different from zero, and here $[\cdot] $  is defined as the integer operator, which will only take the integer part of the inside value.
The remaining $n - [n^{\alpha_j}]$ terms are all equal to zero. 
Assume for a factor which has strength $\alpha = 1$, the factor's loadings will be non-zero for all cross-section units.
We will refer such factor as a strong factor.
The financial theory suggests that the market factor, denotes by the difference between average market return and market risk free return, should have strength equals to one.
And if we have factor strength $\alpha = 0$, it means that the factor has all factor loadings equal to zero, and we will describe such factor as a weak factor \cite{Bailey2016}.
For any factor with strength in [0.5, 1), we will refer such factor as semi-strong factor.
In general term, the more non-zero loading a factor has, the stronger the factor's strength is. 

\section{Estimation Under Single-factor Setting}\label{strength_one_factor_estimation}
To estimate the strength $\alpha_j$, \citeA{Bailey2020} suggest the following procedure.

We first present the procedure in the case a single factor model with only factor $f_t$ and corresponding loading $\beta_{i}$.
The index i indicates $i^{th}$ unit, and t represents the $t^{th}$ time observation.
Here $v_{it}$ is the stochastic error term.

\[  x_{it} = a_{i} +  \beta_{i}f_{t} + v_{it} \tag{2} \label{estimation_model}\]

Assume we have n different units and T observations for each unit: $i = 1, 2,  \cdots, n$ and $t = 1,2, \cdots T$.
Running the OLS time-regression for each $i = 1,2, \cdots, n$, we obtain:
\[   x_{it} = \hat{a}_{iT} +  \hat{\beta}_{iT}f_{t} + \hat{v}_{it}  \]

For every estimated factor loading $\hat{\beta}_{iT}$, we can construct a t-test to examine the significance under the null hypothesis of the loading is zero.
The t-test statistic will be $t_{iT} = \frac{\hat{\beta}_{iT} - 0}{\hat{\sigma}_{iT}}$.  
Empirically, we calculate the t-statistic of $\hat{\beta}_i$ using:

\[t_{i T}=\frac{\left(\bm{f}^{\prime} \bm{M}_{\tau} \bm{f}\right)^{1 / 2} \hat{\beta}_{i T}}{\hat{\sigma}_{i T}}=\frac{\left(\bm{f}^{\prime} \bm{M}_{\tau} \bm{f}\right)^{-1 / 2}\left(\bm{f}^{\prime} \bm{M}_{\tau} \bm{x}_{i}\right)}{\hat{\sigma}_{i T}} \tag{3} \label{test_statistic} \]

Here, the $\bm{M}_{\tau} = \bm{I}_T - T^{-1}\bm{\tau}\bm{\tau^\prime}$, and the $\bm{\tau}$ is a $T\times 1$ vector with every element equals to one, $\bm{f}$ and $\bm{x_i}$ are two vectors with: $\bm{f} = (f_1, f_2 \cdots, f_T)^{\prime}$   $\bm{x_i} = (x_{i1}, x_{i2}, \cdots, x_{iT})$.
The denominator $\hat{\sigma}_{iT} = \frac{\sum_{i=1}^{T} \hat{v}^2_{it} }{T}$.

Using this test statistic, we can then define an indicator function as: $\ell_{i,n} := {\bf1}[|\beta_i|>0]$.
If the factor loading is non-zero , $\ell_{i,n} = 1$.
In practice, we use the $\hat{\ell}_{i,nT} := {\bf1}[|t_{it}|>c_p(n)]$.
Here, if the t-statistic $t_{iT}$ is greater than critical value $c_p(n)$, i.e. if the test result reject the null hypothesis, we yield $\hat{\ell}_{i,n} = 1$, otherwise we have $\hat{\ell}_{i,n} = 0$.
In other words, we are counting how many $\hat{\beta}_{iT}$ are significantly different from zero.
With the indicator function, we then define $\hat{\pi}_{nT}$ as the fraction of significant factor loading amount to the total factor loading amount:

\[  \hat{\pi}_{nT} = \frac{\sum_{i=1}^n \hat{\ell}_{i,nT}}{n} \tag{4} \label{pi_function} \]


In term of the critical value $c_p(n)$, we use the following critical value function.

\[   c_p(n) = \Phi^{-1}(1 - \frac{p}{2n^\delta})   \tag{5} \label{critical_value_function} \]

Suggested by \citeA{Bailey2019}, here, $\Phi^{-1}(\cdot)$ is the inverse cumulative distribution function of a standard normal distribution, p is the size of the test, and $\delta$ is a non-negative value represent the critical value exponent. 
Adopting this adjusted value helps to tackle the problem of multiple-testing.


After obtaining the $\hat{\pi}_{nT}$, we can use the following formula provided by \citeA{Bailey2020} to estimate the strength of corresponding factor:
\[ \hat{\alpha} = \begin{cases}
1+\frac{\ln(\hat{\pi}_{nT})}{\ln n} & \text{if}\; \hat{\pi}_{nT} > 0,\\
0, & \text{if}\; \hat{\pi}_{nT} = 0.
	\end{cases} \tag{6} \label{estimation_method} \]

When we have the $\hat{\pi}_{nT} = 0$, it means that none of the factor loadings are significantly different from zero, therefore the estimated $\hat{\alpha}$ will be equal to zero. 
From the estimation, we can find out that $\hat{\alpha} \in [0,1]$.

\section{Estimation Under Multi-Factor Setting}\label{strength_multi_estimation}

This estimation can also be extended into a multi-factor set up.
Consider the following multi-factor model:

\[x_{it} = a_i +\sum_{j = 1}^k\beta_{ij}f_{jt} +v_{it} = a_i + \bm{\beta}^{\prime}_{i}\bm{f}_{t} +v_{it} \tag{7} \label{multi_factor_model} \]

In this set up, we have $i = 1, 2, \cdots, n$ units, $t = 1, 2, \cdots, T$ time-series observations, and specially, $j = 1, 2,\cdots, k$ different factors.
Here $\bm{\beta}_{i} = (\beta_{i1}, \beta_{i2}, \cdots, \beta_{ij})^{\prime} $ and $\bm{f}_t = (f_{1t}, f_{2t}\cdots, f_{jt})^{\prime}$.
We employed the same strategy as above, after running OLS and obtain the:

\[ x_{it} =\hat{a}_{iT} + \bm{\hat{\beta}}^{\prime}_{i}\bm{f}_{t} + \hat{v}_{it}    \]

We use the significance test to exam how many factor loadings are non-zero.
To conduct the significance test, we calculate the t-statistic: $t_{ijT} = \frac{\hat{\beta}_{ijT}-0}{\hat{\sigma}_{ijT}}$. Empirically, the test statistic can be calculated using:
\[ t_{i j T}=\frac{\left(\bm{f}_{j t}^{\prime} \bm{M}_{F_{-j}} \bm{f}_{j t}\right)^{-1 / 2}\left(\bm{f}_{j t}^{\prime} \bm{M}_{F_{-j}} \bm{x}_{i}\right)}{\hat{\sigma}_{i T}} \]

Here, $\bm{f}_{j t}=\left(f_{j 1}, f_{j 2}, \ldots, f_{j T}\right)^{\prime}, \bm{x}_{i}=\left(x_{i 1}, x_{i 2}, \ldots, x_{i T}\right)^{\prime}, \bm{M}_{F_{-j}}=\bm{I}-\bm{F}_{-j}\left(\bm{F}_{-j}^{\prime} \bm{F}_{-j}\right)^{-1} \bm{F}_{-j}^{\prime}$, and $\bm{F}_{-j}=\left(\bm{f}_{1 t}, \ldots, \bm{f}_{j-1 t}, \bm{f}_{j+1 t}, \ldots, \bm{f}_{m t}\right)^{\prime}$
For the denominator's $\hat{\sigma}_{iT}$, it was from $\hat{\sigma}_{i T}^{2}=T^{-1} \sum_{t=1}^{T} \hat{u}_{i t}^{2}$, the $\hat{u}_{it}$ is the residuals of the model.
Then, we can use the same critical value from (\ref{critical_value_function}).
Obtaining the corresponding ratio $\hat{\pi}_{nTj}$  from (\ref{pi_function}), and use the function:
\begin{equation*}
\hat{\alpha}_{j}=\left\{\begin{array}{l}
1+\frac{\ln \hat{\pi}_{n T, j}}{\ln n}, \text { if } \hat{\pi}_{n T, j}>0 \\
0, \text { if } \hat{\pi}_{n T, j}=0
\end{array}\right.
\end{equation*}
to estimate the factor strength under the multi-factor scenario.

\chapter{Monte Carlo Simulation}\label{MC}
	In this chapter, we set up several simple Monte Carlo simulation experiments to study the finite sample properties of factor strength $\hat{\alpha}_j$.
	Through the simulation, we compare the property of the factor strength in different settings.
	
\section{Simulation Design}
The experiments is designed to reflect the CAPM model and its extension.
For simplicity, we first define $x_{it} := r_{it}- rf_{t}$.
$r_{it}$ is the unit's return, and $rf_{t}$ represent the risk-free rate at time t, therefore, the $x_{it}$ is the excess return of unit i at time t.
We use $f_{mt}:=r_{mt} - rf_{t}$ to denote the market factor, and $\beta_{im}$ to denote the market factor loading.
The market factor is defined as the difference between the average market return and the risk free return.
Here $r_{mt}$ is the average market return of hypothetically all assets in the universe.
%Additionally, we set $q_1(\cdot)$ and $q_2(\cdot)$ as two different functions that represent the unknown mechanism of market factor and other risk factors in pricing asset risk.
%In the classical CAPM model and it's multi-factor extensions, for example, the three-factor model introduced by \citeA{Fama1992}, both $q_1$ and $q_2$ are linear.
Now consider the following data generating process (DGP):
	
		\[ x_{it} = \beta_{im}f_{mt} +  \sum_{j=1}^k\beta_{ij}f_{jt} +\epsilon_{it}  \]
	
%	\[ x_{it} = q_1(f_{mt}) + q_2( \sum_{j=1}^k\beta_{ij}f_{jt}) +\epsilon_{it}  \]


In the simulation, we consider a dataset has $i = 1, 2,\dots, n$ different cross-section units, with $t= 1, 2,\dots, T$ different time-series observations. 
Different risk factors are represented by the $f_{jt}$, and the corresponding $\beta_{ij}$ are the factor loadings.
We expect the market factor will have strength equal to one all the time, so we consider the market factor has strength $\alpha_m = 1$, and this will be reflected in the setting.
$\varepsilon_{it}$ is the stochastic error term.

% For each factor, we assume they follow a multivariate normal distribution with mean zero and a $k\times k$ variance-covariance matrix $\Sigma$. 
%\begin{align*}
%\bm{f_t} = \begin{pmatrix}
%f_{i,t}\\f_{2,t}\\\vdots\\f_{k,t}
%\end{pmatrix} \sim MVN(\bm{0}, \Sigma) \quad
 %\Sigma := 
%\begin{pmatrix}
%\sigma^2_{f_1}, & \rho_{12}\sigma_{f1}\sigma_{f2} &\cdots  & \rho_{1k}\sigma_{f1}\sigma_{fk}\\
%\rho_{12}\sigma_{f2}\sigma_{f1}, & \sigma^2_{f2} &\cdots  & \rho_{2k}\sigma_{f2}\sigma_{fk}\\
%\vdots & \vdots & \ddots & \vdots \\
%\rho_{1k}\sigma_{fk}\sigma_{f1}, & \rho_{k2}\sigma_{fk}\sigma_{f2} &\cdots  & \sigma^2_{fk}\\
%\end{pmatrix}
%\end{align*}
%The diagonal of matrix $\Sigma$ indicates the variance of each factor, and the rest represent the covariance among all $k$ factors.


	\section{Experiment Setting}\label{exp_set}
%Follow the general model above, we assume both $q_1(\cdot)$ and $q_2(\cdot)$ are linear function:
%\begin{align*}
%q_1({f_{mt}}) &= a_{i} +\beta_{im} f_{mt}\\
%q_2(\sum_{j = 1}^{k}\beta_{ij}f_{jt}) &=\sum_{j = %1}^{k}\beta_{ij}f_{jt}
%\end{align*}
To start the simulation, we consider a two-factor model:
\[    x_{it} = a_{i} + \beta_{i1}f_{1t} + \beta_{i2}f_{2t}+\epsilon_{it} \tag{8} \label{two_factor}   \]
The constant term $a_{i}$ is generated from a uniform distribution, $a_{it} \sim \mathnormal{U}[-0.5,0.5]$.
For the factor loading $\beta_{i1}$ and $\beta_{i2}$, we first use a uniform distribution $IIDU(\mu_{\beta} - 0.2, \mu_{\beta}+0.2)$ to produce the values.
Here we set $\mu_{\beta}=0.71$ to make sure every generated loading value is sufficiently larger than 0.
Then we randomly assign $n - [n^{\alpha_{1}}]$ and $n - [n^{\alpha_{2}}]$ factor loadings as zero.
$\alpha_1$ and $\alpha_2$ are the true factor strength of $f_1$ and $f_2$. 
In this simulation, we will start the factor strength from 0.7 and increase it with increment 0.05 till unity, say $(\alpha_{1}, \alpha_{2}) = \{0.7, 0.75,0.8,\cdots,1\}$.
 $[\cdot]$ is the integer operator defined at chapter \ref{strength_one_factor_estimation}.
This step reflects the fact that only $[n^{\alpha_1}]$ or $[n^{\alpha_2}]$ factor loadings are non-zero.
In terms of the factors, they come from a multinomial distribution $MVN(\bm{0}, \Sigma) $, 
\begin{align*}
 \begin{pmatrix}
		f_{i,t}\\f_{2,t}
	\end{pmatrix} \sim MVN(\bm{0}, \Sigma) \quad
	\Sigma := 
	\begin{pmatrix}
		\sigma^2_{f_1}, & \rho_{12}\sigma_{f1}\sigma_{f2} \\
		\rho_{12}\sigma_{f2}\sigma_{f1}, & \sigma^2_{f2} \\
	\end{pmatrix}
\end{align*}
The diagonal of matrix $\Sigma$ indicates the variance of each factor, and the rest represent the covariances.

Currently, we consider four different experiments set up:

\begin{experiment}[single factor, normal error, no correlation]
Set $\beta_{i2}$ from (\ref{two_factor}) as 0, the error term $\varepsilon_{it}$ and the factor $f_{1t}$ are both standard normal.
\end{experiment}

\begin{experiment}[two factors, normal error, no correlation]
Both $\beta_{i1}$ and $\beta_{i2}$ are non-zero. Error term and both factors are standard normal. The correlation $\rho_{12}$ between $f_{1t}$ and $f_{2t}$ is zero. 
The factor strength for the first factor $\alpha_1 = 1$ all the time, and $\alpha_2$ varies.
\end{experiment}

\begin{experiment}[two factors, normal error, weak correlation]
Both $\beta_{i1}$ and $\beta_{i2}$ are non-zero. Error term  and both factors are standard normal. The correlation $\rho_{12}$ between $f_{1t}$ and $f_{2t}$ is 0.3.
The factor strength for the first factor $\alpha_1 = 1$ all the time, and $\alpha_2$ varies.
\end{experiment}

\begin{experiment}[two factors, normal error, strong correlation]
	Both $\beta_{i1}$ and $\beta_{i2}$ are non-zero. Error term and both factors are standard normal. The correlation $\rho_{12}$ between $f_{1t}$ and $f_{2t}$ is 0.7.
	The factor strength for the first factor $\alpha_1 = 1$ all the time, and $\alpha_2$ varies.
\end{experiment}

The factor strength in experiment one is estimated using the method discussed in chapter (\ref{strength_one_factor_estimation}), and for the rest of experiments, we use the method from chapter \ref{strength_multi_estimation}.
The size of the significance test is $p = 0.05$, and the critical value exponent $\sigma$ has been set as 0.5.
For each experiment, we calculate the bias, the RMSE and the size of the test to assess the estimation performances.
The bias is calculated as the average of difference between the true factor strength $\alpha$ and the estimated factor strength $\hat{\alpha}$.
\[bias = \frac{1}{R}\sum_{r = 1}^R(\alpha - \hat{\alpha}_r)\]
The Root Square Mean Error (RMSE) comes from:
\[ RMSE =[\frac{1}{R}\sum_{r=1}^{R}(bias_r)^2 ]^{1/2}\]
Where the R represents the total number of replication.
The size of the test is under the hypothesis that $H_0: \hat{\alpha_j} = \alpha_j,\;j =1, 2$ against the alternative hypothesis $H_1:\hat{\alpha_j} \neq \alpha_j,\; j=1,2$.
Here we employed the following test statistic from \citeA{Bailey2020}.

	\[  z_{\hat{\alpha_j}:\alpha_j} =\frac{(\ln n)\left(\hat{\alpha_j}-\alpha_{j}\right)-p\left(n-n^{\hat{\alpha_j}}\right) n^{-\delta-\hat{\alpha_j}}}{\left[p\left(n-n^{\hat{\alpha}_j}\right) n^{-\delta-2 \hat{\alpha}_j}\left(1-\frac{p}{n^{\delta}}\right)\right]^{1 / 2}}\quad j=1,2 \tag{9}  \label{z_indicator}\]

Define a indicator function $\bm{1}(|z_{\hat{\alpha_j}:\alpha_j} |>c|H_0)$.
For each replication, if this test statistic is greater than the critical value of standard normal distribution: $c = 1.96$, the indicator function will return value 1, and 0 otherwise.
Therefore, we calculate the size of the test base on:
	\[ size = \frac{\sum_{r=1}^{R} \bm{1}(|z_{\hat{\alpha_j}:\alpha_j} |>1.96|H_0)}{R} \quad j =1,2 \tag{10}, \label{size_calculator}\]


For this Monte Carlo Simulation, we consider the different combinations of T and n with $T = \{120, 240, 360\}$, $n =\{100, 300, 500\} $.
The market factor is designed to have strength $\alpha_m = 1$ all the time, and the strength of the other factor will be $\alpha_{x} = \{0.7, 0.75, 0.8,0.85, 0.9,0.95, 1\}$. For every setting, we will repeat the experiment 2000 times independently, all the constant and variables will be re-generate for each replication.


 
	\section{Monte Carlo Findings}\label{MC_findings}
We report the results in Table (\ref{table:exp1}), (\ref{table:exp2}), (\ref{table:exp3}), and (\ref{table:exp4}) in Appendix \ref{simulationtable}.

Table (\ref{table:exp1}) provides the results under the experiment 1.
The estimation method we applied tends to over-estimate the strength slightly most of the time when the true strength is relatively weak under the single factor set up.
With the true strength increasing, the bias will turn to negative, represents an under-estimated results.
Such bias, however, vanishes quickly while observation t, unit amount n, and true strength $\alpha$ increase.
When we increase the time span by including more data from the time dimension, the bias, as well as the RMSE decrease significantly.
Also, when including more cross-section unit n into the simulation, the performance of the estimation improves, as shown by the decreased bias and RMSE values.
An impressive result is that the gap between estimation and true strength will go to zero when we have $\alpha = 1$, the strongest strength we can have.
With the strength approaching unity, both bias and RMSE will converge to zero.
We also present the size of the test in the table.
The size of the test will not vary too much when the strength increases, so as the unit increases.
But we can observe that when observation amounts for each unit increase, in other words, when t increases, the size will shrink dramatically.
The size will become lower than the 0.05 threshold after we extend the t to 240, or empirically speaking, when we included 20 years monthly return data into the estimation.
Notice that, from the equation (\ref{z_indicator}), when $\hat{\alpha} = \alpha = 1$, the nominator becomes zero.
Therefore, the size will collapse to zero in all settings, so we do not report the size for $\hat{\alpha} = \alpha = 1$

%For the two factors scenarios, we obtain similar conclusions in no correlation setting, weak correlation setting, and the strong correlation setting.
For the two factors scenarios, we obtain similar conclusions across all three different correlation settings.
The result of no correlation settings is shown in the table (\ref{table:exp2}), table (\ref{table:exp3}) shows the result when the correlation between two factors is 0.3, and the table (\ref{table:exp4}) presents the result of 0.7 correlation setting.
Same as the single factor scenario, the estimation results improve when increasing either the observations amount t, or the cross-section units amount n.
We also have the same unbiased estimation when true factor strength is unity under all unit-time combinations.
In some cases, even when the factor strength is relatively weak, we can have unbiased estimation if the n and t are big enough. (see table (\ref{table:exp3})).
However, we should also notice that when t > n, the results of the size of the test in two factors setting are performing similar to the single factor result. 
The size will shrink with the observation amount t increasing, and when we have time-series amounts t greater than 240, the size will be smaller than 0.05 threshold in all situations.
However, it is worth notice that in the strong correlation setting, the size of the test is extremely big when the time span is relatively short, and the test size will not improve even we increase the sample size.
Once we increase the t, the size of the test will reduce dramatically, and when we have the thirty year time period, the size of the test are almost all below the 0.05 threshold.


	\section{Elastic Net}\label{Elastic_Net}

Elastic net,introduced by \citeA{Zou2005},  is a penalised linear regression method which developed from the Lasso regression \cite{Tibshirani1996} and ridge regression.
To illustrate the application of elastic net method, first recall the multi-factor model (\ref{multi_factor_model}) we discussed in section \ref{strength_multi_estimation}.
When applying the OLS on estimating the factor loading $\bm{\beta_{i}}$, we targeting minimise the Residual Sum of Squares (RSS):
\[  \bm{\hat{\beta}}_i =   \argmin\{  (x_{it} - \hat{a}_{iT} - \bm{\hat{\beta}}_i^{\prime}\bm{f}_t )^2 \}    \]
In the method of elastic net, base on the RSS minimisation, we impose two extra penalty terms on the estimated loadings:
	\[   \bm{\hat{\beta}_{i}}  = \argmin_{\beta_{ij}}\{ (x_{it} - \hat{a}_{iT} - \bm{\hat{\beta}}_{i} ^{\prime}\bm{f}_{t})^2 + \lambda_2\sum_{j = 1}^k\hat{\beta}_{ij}^2  + \lambda_1\sum_{j=1}^k|\hat{\beta}_{ij}|  \label{ENcriterion} \tag{11}   \}    \]
Here, the estimated $\bm{\beta}_i$ value is subject to two penalty terms: the $L^1$ norm $\lambda_1\sum_{j=1}^k|\hat{\beta}_{ij}|$ and the $L^2$ norm: $\lambda_2\sum_{j = 1}^k\hat{\beta}_{ij}^2$.
Both term 
{\bf To be complete}
\chapter{Empirical Application: Factor Strength}\label{Empirical:factor_strength}
	
Researchers and practitioners have been using the CAPM model \cite{Sharpe1964, Lintner1965, Black1972} and its multi-factor extension like the Fama-French three factors model \cite{Fama1992} when they are trying to capture the uncertainty of asset's return.
The surging amount of new factors \cite{Harvey2019} provides numerous option to construct the CAPM model, but it also requires users to pick the factors wisely.
In this chapter, we will use the method introduced in the chapter \ref{strength} to estimates the factor strength of 146 candidate factors.
We try to reduce the dimension of the factor group by grouping factors base on their strength.

First, we introduce the data set used in this empirical chapter, and the setting for the estimation process.
Then, we discuss the findings from the estimation results.
The estimated factors strength provides us a starting point when applied the Elastic net method in the chapter \ref{Empirical:Elastic_net}.

	\section{Description of Data for Factor Strength Estimation}\label{data}
	

In the empirical application part, we use the monthly securities returns of company listed on the U.S. market.
The companies are selected from Standard Poor (S\&P) 500 index component companies.\footnote{The companies return data was obtained from the Global Finance Data: http://www.globalfinancialdata.com/,\\ Osiris: https://www.bvdinfo.com/en-gb/our-products/data/international/osiris, \\and Yahoo Finance: https://finance.yahoo.com/.}
We prepared three data sets for different time spans: 10 years (January 2008 to December 2017, T = 120), 20 years (January 1998 to December 2017, T  = 240), and 30 years (January 1989 to December 2017, T = 360).
The initial data set contains 505 companies, because of the component companies of the index are constantly changing, bankrupt companies will be moved out, and new companies will be added in.
Also, some companies do not have enough observations.
Therefore, for each of the datasets, the number of companies (n) is different, the dimensions of the data set are showing in the table (\ref{Data_set}).

\begin{table}[H]
	\centering
		\caption{Data Set Dimensions}
			\label{Data_set}
	\begin{tabular}{c|ccc}
		\hline
		& Time Span                    & Number of Companies (n) & Observations Amount (T) \\ \hline
		10 Years & January 2008 - December 2017 & 419                  & 120                     \\
		20 Years & January 1998 - December 2017 & 342                  & 240                     \\
		30 Years & January 1988 - December 2017 & 242                  & 360                     \\ \hline
	\end{tabular}
\end{table}
For the risk-free rate, we use the one-month U.S. treasury bill return.\footnote{ The risk free rate was from the Kenneth R. French website: http://mba.tuck.dartmouth.edu/pages/faculty/ken.french/}
For every company i, we calculate the companies return at month t ($r_{it}$) using the following formula:
\begin{align*}
r_{it} = \frac{p_{i t} - p_{i t-1}}{p_{i t-1}}\times 100
\end{align*}
and then we calculate the excess return $x_{it} = r_{it} - r_{ft}$.
Here the $p_{it}$ and $p_{i t-1}$ are the company's close stock price on the first trading day of month t and t-1.
The price is adjusted for the dividends and splits.\footnote{The data is adjusted base on the Central for Research in Security Price (CRSP) method.}

Concerning the factors, we use 145 different risk factors from \citeA{Feng2020}.
The details of the factors are discussed in chapter \ref{EN:risk_factor}.
The factor set also includes the market factor, represented by the difference between the average market return and risk-free return.
The average market return is a weighted average return of all stocks in the U.S. market, calculated by CSRP.
Each factor contains observations from January 1988 to December 2017.

\section{Setting for Factor Strength Estimation}
For the first part of the empirical application, we estimate the factor strength using the method discussed in chapter \ref{strength}.
More precisely, we set the regression models based on chapter \ref{strength_multi_estimation}.
\begin{align*}
%  r_{it} - rf_{t} &= a_i + \beta_{im}(r_{mt} - rf_{t}) + v_{it} \\
r_{it} - rf_{t} &= a_i + \beta_{im}(r_{mt} - rf_{t}) + \beta_{ij}f_{jt} + v_{it} 
\end{align*}

Here $r_{it}$ is the return of asset i at time t, and $rf_t$ is the risk free return.
Therefore, $r_{it} - rf_t$ is the excess return of the asset i.
We calculate the market factor using $r_{mt} - rf_{t}$, the difference between average market return and risk-free return at the same time t.
The $j^{th}$ risk factor at time is represented by $f_{jt}$.
Here $j = 1, 2, \cdots 145$ . 
Market factor loading and risk factor loadings are denotes by $\beta_{mt}$ and $\beta_{ij}$ respectively.

%When conducting the estimation, we will re-estimates the market factor strength when estimating the strength for different risk factors.
%This means that we will have 145 different market factor estimated strength $\alpha_{mj}$ for $j = 1,2, \cdots , 145$ 
%Therefore, we calculates the average market factor strength:
%	\[ \bar{\alpha}_m = \frac{1}{145}\sum_{j = 1}^{145}\alpha_{mj}   \]

%We use two different regressions in the purpose of estimating the strength under the single factor setting and the two factors setting.
%However, due to the potential correlations among factors, we will only focus the market factor strength when using the first single factor regression.


	\section{Factor Strength Estimation and Discussion}
The complete set of results of factor strength estimation is presented in the appendix \ref{strength_table} and \ref{strength_figures}.
We estimated the factors' strength using three different data sets discussed in the chapter \ref{data}, and rank those strength from strong to weak, alongside the market factor strength, in the table \ref{table:three_ranked_compare}.

We first look at the market factor strength. (see table \ref{table:market})

\begin{table}[H]
	\centering
			\caption{Market factor strength estimation} \label{table:market}
	\begin{tabular}{l|ccc}
		\hline
\hline
                                                                                                  & Ten Year Data & Twenty Year Data & Thirty Year Data \\ \hline
%\begin{tabular}[c]{@{}l@{}}Market Factor Strength\\ (Single Factor Setting)\end{tabular}          & 0.988         & 0.990            & 0.995            \\
\begin{tabular}[c]{@{}l@{}}Average Market Factor Strength \end{tabular} & 0.987         & 0.991            & 0.996            \\    \hline \hline
	\end{tabular}
\end{table}
%As we expected, the estimated strength of market factor under all three scenarios shows consistently strong results.
%The market factor strength across the three periods is close to unity, which indicates that the market factor can generate significant factor loading almost all time for every asset.
%Although the value is close to unity, we still notice that the strength will increase slightly with the time span extended.
%This might indicate that for the security returns, from the long run, it will more closely mimic the behaviours of the market than the short run. 
%Then, we turn to the double factor CAPM setting.
Because of the model we used in estimating strength contains both market factor and risk factor, we will re-estimate the market factor strength every time when we estimating the strength for different risk factors.
This means that we will have 145 different market factor strength estimation results: $\alpha_{mj}$ for $j = 1,2, \cdots , 145$. 
So, we calculate the average market factor strength here:
\[ \bar{\alpha}_{m} = \frac{1}{145}\sum_{j= 1}^{145}\alpha_{jm}   \]
%$\alpha_{im}$ is the estimated market factor strength for the $i^{th}$ risk factor setting.
As the financial theory suggests, the estimated strengths of market factor under all three scenarios show consistently strong strength.
The market factor strength across the three periods is close to unity, which indicates that the market factor can generate significant factor loading almost all time for every asset.
Although the value is close to unity, we still notice that the strength will increase slightly with the time span extended.
This might indicate that the security return will more closely mimic the behaviours of the market from the long-run prospective.

%Table\ref{table:market} shows that the estimated strength is consistent with the single factor settings.
%All three data sets provides extremely strong results, strengths are close to one.
%Overall, the market factor results indicates that the market factor can generate significant loadings for almost every companies' return at any time.

When looking at different risk factors (see table \ref{table:top15}),  the ten-year data set in general provides a significantly weaker result, compares with the other two data sets results.
Except for the market factor, no other factors from the ten-years result show strength above 0.8.
The strongest factor besides the market factor is the beta factor which has strength around 0.75.
In contrast, the strongest risk factor (factor other than market factor) in the twenty-year data set is the ndp (net debt-to-price), which has strength 0.937.
In the thirty-year scenario, the salecash (sales to cash) is the strongest with strength 0.948.

\begin{table}[hbt!]
	\centering
	\caption{ Proportion of factor within certain strength range }\label{table:proportion}
	\begin{tabular}{lccc}
		\hline
		\hline
		Strength Level & \multicolumn{1}{l}{10 Year Data Proportion} & \multicolumn{1}{l}{20 Year Data Proportion} & \multicolumn{1}{l}{30 Year Data Proportion} \\ \hline
		{[}0.9, 1{]}   & 0\%                                         & 21.4\%                                      & 23.4\%                                         \\
		{[}0.85, 0.9)  & 0\%                                         & 17.9\%                                      & 17.9\%                                      \\
		{[}0.8, 0.85)  & 0\%                                         & 7.59\%                                      & 6.21\%                                      \\
		{[}0.75, 0.8)  & 0\%                                         & 11.7\%                                      & 17.9\%                                      \\
		{[}0.7, 0.75)  & 7.59\%                                      & 8.28\%                                      & 7.59\%                                      \\
		{[}0.65, 07)   & 15.9\%                                      & 8.28\%                                      & 2.76\%                                      \\
		{[}0.6, 0.65)  & 17.9\%                                      & 5.52\%                                      & 8.97\%                                      \\
		{[}0.55, 0.6)  & 13.1\%                                      & 6.21\%                                      & 2.76\%                                      \\
		{[}0.5, 0.55)  & 8.97\%                                      & 4.83\%                                      & 4.14\%                                      \\
		{[}0, 0.5)     & 36.6\%                                      & 8.28\%                                      & 8.28\%                                      \\ \hline\hline
	\end{tabular}
\end{table}

When comparing the proportion of factors with strengths falling in different intervals between 0 and 1 (see table \ref{table:proportion}), we can find that when using 0.8 as a threshold, over forty-five per cent factors in the twenty-year and thirty-year result exceeds this threshold.
In ten year results, the number is zero.
We also find that nearly 40\% of factors from the ten-year dataset show strength less than 0.5, which is almost four times higher than the twenty and thirty years proportion.
When look at the ranking, we found that the top three factors are consistent for the twenty year data set result and thirty year data set result.
The top three factors: ndq (Net debt-to-price ), salecash (sales to cash), and quick (quick ratio) are presents in both twenty and thirty years results with different order.
We would expect when applying the Elastic net method with the twenty-year and thirty-year data set, those three factors with the market factors are more likely to be selected than other weaker factor.

Another interesting finding is that the roavol (Earnings volatility, 10th of ten-year result, 7th of twenty-year result, 5th of thirty-year result ), age (Years since first Compustat coverage, 11th of ten-year result, 9th of twenty-year result, 4th of thirty-year result), and ndp (net debt-to-price, 14th of ten-year result, 1st of twenty-year result, 2nd of thirty-year result) are all ranking high  in all three time periods estimated results.
This indicates a persistent risk pricing ability of these three factors exist, even with the changes of the data set's dimensions.

\begin{table}[]
	\centering
	\caption{Selected Risk Factor with Strength: top 15 factors from each data set and three well known factors.}
	\label{table:top15}
	\begin{tabular}{llc|llc|llc}
		\hline
		\multicolumn{3}{c|}{Ten Year} & \multicolumn{3}{c|}{Twenty Yera} & \multicolumn{3}{c}{Thirty Year} \\ \hline
		Rank & Factor     & Strength & Rank   & Factor     & Strength   & Rank   & Factor     & Strength   \\ \hline
%	 	 & Market & 0.988 & &Market & 0.990 & &Market & 0.995 \\
		1   & beta               & 0.749    & 1      & ndp           & 0.937      & 1      & salecash  & 0.948\\
		2   & baspread       & 0.730    & 2      & quick        & 0.934      & 2      & ndp          & 0.941\\
		3   & turn               & 0.728    & 3      & salecash   & 0.933      & 3      & quick      & 0.940\\
		4   & zerotrade      & 0.725    & 4      & lev            & 0.931      & 4      & age         & 0.940\\
		5   & idiovol           & 0.723    & 5      & cash         & 0.931      & 5      & roavol    & 0.938\\
		6   & retvol            & 0.721    & 6      & dy             & 0.929      & 6      & ep           & 0.937\\
		7   & std\_turn      & 0.719    & 7      & roavol      & 0.929      & 7      & depr       & 0.935\\
		8   & HML\_Devil & 0.719    & 8      & zs              & 0.927      & 8      & cash       & 0.934\\
		9   & maret           & 0.715    & 9        & age          & 0.927      & 9      & rds         & 0.931\\
		10 & roavol          & 0.713    & 10     & cp             & 0.926      & 10    & dy          & 0.927 \\
		11 & age               & 0.703    &11      & ebp           & 0.926      & 11     & currat   & 0.927 \\ 
		12 & sp                 & 0.699    &12      & op            & 0.925      &12       & chcsho  & 0.927 \\ 
		13 & ala                & 0.699    &13      & cfp          & 0.924       &13      & lev         & 0.926 \\ 
		14 & ndp              & 0.686    &14      & nop          & 0.924       &14     & stdacc     & 0.926 \\ 
		15 & orgcap         & 0.686    &15      & ep            & 0.923       &15     & cfp          & 0.925 \\ 
		20 & UMD            & 0.678    & 29     & HML        & 0.905      & 39     & HML       & 0.894 \\
		24 & HML            & 0.672    & 76     & SMB        & 0.770      & 68     & SMB        & 0.804 \\
		87 & SMB            & 0.512    & 89     & UMD        & 0.733      & 96     & UMD       & 0.745 \\ 
		\hline
	\end{tabular}
\end{table}

We also focus on some well-known factors, namely the Fama-French size factor (Small Minus Big SMB), Fama-French Value factor (High Minus Low: HML) \cite{Fama1992} and the Momentum factor (UMD) \cite{Carhart1997}.
It is surprising that none of these three factors enter the top fifteen list for each data sets.
Except for the HML factor from the twenty and thirty-year data set has strength closely around 0.9, none of the other factors in any data set shows strength higher than 0.85.
When using the ten-year data, both UMD and HML has strength around 0.67, and the SMB only has strength 0.512.
Result from the twenty-year data set shows that HML has strength 0.905, for SMB and UMD the strength are 0.770 and 0.733 respectively.
Comparing with the twenty-year result, the thirty-year estimated strength changes slightly, HML decreases to 0.894, SMB is 0.804 and UMD has strength 0.745.
Therefore, when using the strength as a criterion, we may only select the value factor to incorporate in the CAPM model when having twenty and thirty-year data.

In general, we found that the twenty-year and thirty-year data sets provides similar estimated strength, while, in contrast, the estimated strengths from ten-year data set are significantly weaker.
Therefore, as a second step, in order to see how factor strengths evolve through the time, we decompose the thirty yea-data set into three small subsample.
For each subsamples, it contains 242 companies (n = 242). 
And for each company, we obtained 120 observations (t = 120). 
The results are present in the table (\ref{table:thirty_decompose}) and figure (\ref{figure:thirty_decompose}).


In general, we can conclude that about 80\% factors follow a patten that the strength will increase from the first decade (January 1988 to December 1997) to the second decade (January 1998 to December 2007), and then decrease in the third decade (January 2008 to December 2017).
Such pattern can also be observed from the figure (\ref{figure:thirty_decompose}).
The drop of factor strength in the third decades can be reconciled with the ten-year data results shows a significantly weaker strength than the results from twenty and thirty years data set.

%Overall from the factor strength prospect, we would expect that for different time periods, we will have different candidate factors for the CAPM model.

Overall from the factor strength prospect, we would expect that for different time periods, we will select different factors for the multi-factor CAPM model.
%For the ten-year data set, we would expect that only the market factor be useful, and therefore the elastic net method applied latter may only select the market factor.
For the ten-year data set, it seems that only the market factor can independently provides enough information for explaining the risk-return relationship of stocks.
And if we apply the Elastic net to the ten-year data set, we will expect that only the market factor been selected.
From the twenty and thirty-year data results, we will expect to see a longer list for potential factors, 62 factors from the twenty-year estimation and 45 from the thirty years has strength greater than 0.8.
Hence, if we use factor strength as the criterion to select factors, we may need more carefully select from the list or relies on other methods.
And we may expect the Elastic net method latter to select a more generous model.

We also notice that for some factors, their strength decreases with time.
For instance, the gma (gross profitability) factor and dwc (change in net non-cash working capital) factor (see figure \ref{figure:thirty_decompose}) has consecutive strength decrease from the 1987-1997 period to 2007-2017 period.
And for most of the factors, their strength will decrease significantly from the 1997-2007 period to 2007-2017 period.
Therefore, disqualify some factors as the candidate of the CAPM model when using recent year data is inevitable.

%Hence, if we use factor strength as criterion to select factors, we may need more carefully select from the list, and we will expect the elastic net to select a more generous model.


%Hence, we would expect the elastic net to select a less parsimonious model. 

In terms of the findings we have above, there are several potential explanations.
First, if we consider the structure of our data set, we will find that the longer the time span, the fewer companies are included.
This is because the S\&P index will adjust the component, remove companies with inadequate behaviours, and add in new companies to reflect the market situation.
Hence, those 242 companies in the thirty-year data set can be viewed as survivals after a series of financial and economic crisis.
We would expect those companies will have above average performances, such as better profitability and administration, compared with other companies. 
Such traits may indicates a potential bias of the estimation for longer-time span, comparing with the shorter results.



This happening can also be contributed to a series of political and financial unease from the time of late 20 century to 2008.
Crisis like the Russia financial crisis in 1998, the bankruptcy of Long Term Capital Management (LTCM) in 2000, the dot com bubble crisis in early 21st century and the Global Financial Crisis (GFC) in 2008 creates market disturbances.
Such disturbances, however, provides extra correlations among factors.
The extra correlations enable some factors provides additional risk pricing power.
But we should also notice that the financial market has been disturbed by those crises so, therefore, some mechanism may no longer working properly during that period.
Which means that those crises will also have negative influences on factor when they are capturing the risk-return relationship.




\chapter{Empirical Application: Elastic Net}\label{Empirical:Elastic_net}
As introduced in the previous chapter, the high-dimension risk factor group posts challenge when selecting factors.
We want to identify factor that can independently provide information to explain the risk-return relationship. \cite{Cochrane2011}.
In this chapter, we employ the Elastic net algorithm to discuss this problem.
The application is build on the basis of factor strength we estimated in the chapter \ref{Empirical:factor_strength}.
Factor strength provides a criterion to reduce the dimensions of potential factor group.
The factor strength also works as a reference when evaluating the performances of algorithms.
For the rest of this chapter, we first briefly introduce the core idea of the Elastic net method, explain what's different between Elastic net and other factor selection methods, especially the Ridge regression and Lasso regression.
Then, we provide a full discussion of how to select the tuning parameter with regard to the Elastic net method when using r package \textit{glmnet}.
Finally, we compare and discuss the differences selection results of both Elastic net methods and Lasso.


\section{Brief Introduction to Elastic Net} \label{Elastic_Net}

Elastic net,introduced by \citeA{Zou2005},  is a penalised linear regression method which developed on the basis of the Lasso regression \cite{Tibshirani1996} and Ridge regression.
To illustrate the application of the Elastic net method in our research, recall the multi-factor model (\ref{multi_factor_model}) we discussed in chapter \ref{strength_multi_estimation}.
%In that setting, we have $i = 1,2,\cdots, n$ cross-section unit, $j = 1,2,\cdots, k$ risk factors, and for all unit and factor, each of them have $t = 1,2,\cdots, T$ time-series observation.
When applying the OLS to estimate the factor loading $\bm{\beta_{i}} = \{ \beta_{i1}, \beta_{i2}, \cdots, \beta_{ij}   \}$ of corresponding factor $\bm{f_{t}} = \{ f_{1t}, f_{2t}, \cdots, f_{jt} \}$ for model (\ref{multi_factor_model}), we targeting minimise the residual sum of squares (RSS):
\[  \bm{\beta}_i =   \argmin\{  (x_{it} - a_{iT} - \bm{\beta}_i^{\prime}\bm{f}_t )^2 \}    \]
The OLS method will consider all factors proposed by users when constructing the multi-factor CAPM model.
%This means that even factors with weak strength, which can not bring any new information to price the risk of assets will generate loadings under the OLS method.
This means that even factors with weak or no relationship with the assets, will generate loadings under the OLS method.
%The Elastic net method, although also focusing on minimising RSS, including two extra penalty terms inside the loss function.
The Elastic net method, although also focusing on minimising RSS, avoid this problem by adding two extra penalty terms.
When applying the Elastic net, we focusing on the following loss function:
\[   \bm{\beta}_{i}  = \argmin_{\beta_{ij}}\{ (x_{it} - a_{iT} - \bm{\beta}_{i} ^{\prime}\bm{f}_{t})^2 + \lambda_2\sum_{j = 1}^k\beta_{ij}^2  + \lambda_1\sum_{j=1}^k|\beta_{ij}|  \label{ENcriterion} \tag{11}   \}    \]
Here, the estimated $\bm{\beta}_i$ loading value is subject to two penalty terms: the Lasso penalty $\lambda_1\sum_{j=1}^k|\beta_{ij}|$ and the Ridge penalty $\lambda_2\sum_{j = 1}^k\beta_{ij}^2$.
The Elastic net estimation, in essence, is a combination method of Ridge regression and the Lasso regression.
If setting $\lambda_1 = 0$, the Elastic net will collapse into the Lasso regression, and if $\lambda_2 = 0$, we will obtain same result as the Ridge regression.
In the empirical application of this thesis, the Elastic net estimation uses the following loss function \cite{Friedman2010}:
\[		\bm{\beta}_{i} = \argmin_{\beta_{ij}}\{ \frac{1}{2N} (x_{it}-a_{iT} - \bm{\beta_{i}^{\prime}}\bm{f}_t ^2 ) +\phi P_{\theta}(\bm{\beta}_i)  \} \label{EN:empirical_formula}\tag{12} \]
\[	P_{\theta}(\bm{\beta}_i) =\sum_{j=1}^k [ (1-\theta)\beta_{ij}^2 + \theta |\beta_{ij}|] \label{EN:elastic_net_penalty} \tag{13}\]
Here \citeauthor{Zou2005} call the $P_{\theta}(\bm{\beta}_{i})$ as the Elastic net penalty.
Parameter $\theta$ acts as the turning parameter to determine how will the Elastic net penalty is combined by the Lasso penalty and Ridge penalty.
When set $\theta = 1$, we have $P_{\theta}(\bm{\beta}_i) =\sum_{j=1}^k  |\beta_{ij}|$ which is identical to the Lasso penalty.
Therefore we have the Elastic net collapse to the Lasso regression when $\theta = 1$.
Similarly, when setting $\theta = 0$, we have the Elastic net collapse to the Ridge regression. 
%and when $\theta = 0.5$, the Elastic net is the half-half combination of Lasso and Ridge.
The other tuning parameter $\phi$ decides how strong the penalty terms is.
If $\phi = 0$ the Elastic net will become the OLS estimation.

In this study, we use the data introduced in chapter \ref{data}, and the estimated factor strength from the chapter \ref{Empirical:factor_strength}.
We will briefly review risk factors with their strength in the next section.
More specifically, we allocates the 145 risk factors into six subgroups according to there thirty-year estimated strength.
For each subgroup, we want to investigate how will the Elastic net algorithm, alongside the Lasso regression, select the risk factors for company stock.
We would expect that when facing factor with strong strength, the algorithm will construct a dense factor model, and with the factor strength decrease, the density will decrease simultaneously.

To simplify implementation of the Elastic net, we first run an OLS regression between the market factor and the excess return of each company, and use the resulting residuals as the dependent variable in the Elastic net regression.
This step helps us to remove the potential influence of market factors, allowing the algorithms only focussing on the risk factors. 

Now, the main challenge of applying the Elastic net algorithm is to select the appropriate tuning parameters $\theta$ and $\phi$, and we will discuss the choice of tuning parameter in the following section.

\section{Properties of Risk Factors} \label{EN:risk_factor}

Before we start to discuss the parameter tuning, we briefly review the properties of the risk factors series.
The risk factors data set contains 145 risk factors at the monthly frequency for the period from July 1988 to December 2007.\footnote{For how those factors are constructed, please view \citeA{Feng2020}}
As a standard practice of time series data, we exam the stationarity of the risk factor series by employing Augmented Dick-fuller (ADF) test  \cite{Dickey1979},  Phillips-Perron (PP) test \cite{Phillips1988}, and the Kwiatkowski-Phillips-Schmidt-Shin (KPSS) test \cite{Kwiatkowski1992}.
Overall, the ADF test and PP test both provides the same conclusions that all 145 risk factor series are unit-root free.
The KPSS test, however, disagrees with the ADF and PP test.
If we take 0.05 as the threshold of p-value, the KPSS test concludes that there are six unit root processes across the 145 factors.
We also present the correlation heat map of all 145 risk factors in the Figure \ref{figure:correlation} at the appendix \ref{factor_correlation}.
All risk factors are sorted base on their estimated factor strength.
The dark lower-left area of the Figure \ref{figure:correlation} indicates that factor with strong estimated strength presents a high correlation with other strong factors.
With the factor strength decrease, the correlation coefficient among factors decreases significantly.
If focusing on the upper-right corner of the figure, where weak factors are clustering, we can see that the correlation coefficients are close to zero.
By observing the upper-left corner and the lower-right corer, we can observe that the correlation between weak factors and strong factors are also very low


\section{Tuning Parameter} \label{EN:parameter_tuning }
To fit the Elastic net, we use the R package \textit{glmnet} \cite{Friedman2010, Simon2011}.

From the equation (\ref{EN:empirical_formula}) and (\ref{EN:elastic_net_penalty}) we know that the estimation of Elastic net using \textit{glmnet} dependent on two tuning parameters $\theta$ and $\phi$.
The \textit{glmnet} package provides function to select the $\phi$ value by adopting the principle of minimise mean squared error (MSE) and ten-fold cross-validation.
%The \textit{glmnet} package provides function to select the $\phi$ automatically. 
%This selection is base on the minimisation of Mean Squared Error (MSE), using 10-fold cross-validation.
However, the package does not provide aid on determining which value of $\theta$ parameter is optimal.
Therefore, we adopt the following strategy to select our tuning parameters $\phi$  and $\theta$:
\begin{enumerate}
\item Prepare a sequence of $\theta$ values, from 0: ridge regression, to 1: Lasso regression with the step of 0.01
\item Randomly assign 90\% of the data set as the training set and the rest 10\% as the test set. 
\item For each of the $\theta$ value, we fit the corresponding Elastic net model using the training set, with function picked $\phi$ values.
\item Base on the $\theta$ and $\phi$ values select, we produce the predicted values using the test data, and calculate the MSE between the true values and predicted values.
\item We select the $\theta - \phi$ combination which minimises the MSE.
\end{enumerate}
We repeat the above procedures 2000 times for each factor strength group and
calculates the $\bar{\theta}$ by averaging every $\theta$ we obtained from the repetition.
% average each $\theta$ values to generate $\bar{\theta}$, as our selected tuning parameter.
Due to the computational burden, when implementing step 2, instead of using the full sample of stocks and factors, we randomly select 50 companies from 242 all companies, and 10 factors from each subgroups.
The selected tuning parameter $\theta$ values are shown in table \ref{table:optimal_theta}.
\begin{table}[H]
	\centering
	\caption{Estimated Optimal $\theta$ values for different factor groups}
	\label{table:optimal_theta}
	\begin{tabular}{l|cccc}
		\hline
		\hline
		Factor Group            & (0, 0.5{]}   & (0.5, 0.6{]} & (0.6, 0.7{]} & (0.7, 0.8{]} \\ 
		Selected $\theta$ value & 0.377        & 0.401        & 0.429        & 0.411        \\ \hline
		Factor Group            & (0.8, 0.9{]} & (0.9, 1{]}   & Mix          & Random       \\ 
		Selected $\theta$ value & 0.396        & 0.413        & 0.448        & 0.431        \\ \hline
		\hline
	\end{tabular}
\end{table}
In order to fully investigate the behaviours of parameter tuning, on the basis of the six subgroups, we create tow addition groups.
The Mix group contain five highly strong factors: factor with strength higher than 0.9, and five weak factors: factor with strength lower than 0.5.
The Random group consist of ten randomly selected factors from all 145 risk factors, and the 10 factors are re-selected every replication.
From table \ref{table:optimal_theta} we can see that the selected $\theta$ value in general increase with the factor strength increase.
For weak factor group, the selected parameter value is 0.377.
While for the strong factor group, the value is 0.413.
The mix factor group has the highest $\theta$ value of 0.448.

%Such a pattern of the $\theta$ value, however, does not follow what we expected.
%The definition of factor strength indicates that factor with strong strength is able to produce more significant loadings, in other words, can explain more assets' risk-return relationship.
%Therefore, when using the above procedure to decide tuning parameter, we would expect that for factor groups with lower strength, like groups with strength smaller than 0.5, the selected $\theta$ parameter will be close to one, or larger than other groups' selected $\theta$.
%This is because factor with weak strength will only provide limited pricing power, and therefore may be recognised by the algorithm as redundant variables.
%When $\theta$ is closer to unity, the elastic net is behaved more like a Lasso, which will eliminate variables provides limited explaining power.
%In contrast, when the group has stronger strength, the $\theta$ will approach closer to zero, leads the Elastic nets more like Ridge, which will not eliminate any variables, but only reduce the coefficient.
%So we would expect the $\theta$ value to increase with the group strength decrease.


The upward pattern of $\theta$ value can be explained by several reasons.
First, MSE may not be an ideal criterion for parameter tuning under the scenario of applying Elastic net with regard to the financial return. 
In our application, we find that the MSE for all $\theta - \phi$ combinations are very close to each other.
The results of MSE for all combination are around 64.
%Therefore, we can not tell the differences among the results of $\theta$ value by using MSE.
So the algorithm can not distinct the results 
Therefore, we can hardly tell the differences of adopting different $\theta$ just by MSE.
Second, because of the estimation method we used, the market risk has already been absorbed by the market factors.
What been left is only the idiosyncratic risk of the assets, and when using risk factors we will expected those factors can form certain linear combination to  explain those risk.
When all factors are strong, the linear combination may consist by a few of factors because even just a single strong factor, we would think it can explain most of the risk.
Then the rest of the factors may be recognised by the algorithm as redundancy and abandoned.
But when we have weak factors, the linear combination may rely on more factor's contribution.
%Then for the strong factors, we would expect that any single of them, or  a very small portion of them, can explained most of the idiosyncratic risk left by the market factor.
%Therefore, when we ask any ten strong factors to determine the risk simultaneously, it is possible that very few of the ten factors can explain most of the risk and the other risk factors will be recognised by the algorithm as redundant.
%But if all factors are weak, it is possible that there exist some linear combinations among most, if not all, weak factors provides enough explaining power for the idiosyncratic risk.
%But if all factors are weak, it is possible that there exists some linear combinations among most, if not all, weak factors, and such combination will provides enough explaining power for the idiosyncratic risk.
Therefore, those weak factors will be reserved by the algorithm, and hence, the parameter $\theta$ will close to zero, indicates a more Ridge like regression.





\section{Elastic Net Findings}
We applied the Elastic net algorithm with tuning parameters estimated in the previous chapter using the thirty-year data set.
The reasons why we use the thirty-year data instead of ten or twenty years is because the Monte Carlo simulation result (see chapter \ref{MC_findings} and Appendix \ref{simulationtable}) indicates us that the estimation accuracy of factor strength will improve significantly when the time-series observation is large.
Therefore, we use the factor strength estimated from the thirty-year data set.

We divided the 145 risk factors into six groups base on their factor strength.
We also randomly selected ten factors from  weak factor group (factor with strength less than 0.5), and ten factors from strong factor group (factor with strength above 0.9) to form a mixed factor group.
For each factor groups, we run two regression: the Elastic net regression with $\theta$ values presented in table \ref{table:optimal_theta}, and the Lasso regression with $\theta = 1$.
Instead of running a pooled regression, we run the Elastic net and Lasso for each individual company, and record the result of factor selection of every stock.
First, we focusing on the general behaviours of factor selection between our two methods.
Table \ref{table:select_prop} presents the average factor selection amount for each factor groups of two selection methods.
\begin{table}[h]
	\centering
		\caption{Average factor selection proportions and factor selection counts of Elastic Net and Lasso}
		\label{table:select_prop}
		\resizebox{\textwidth}{!}{
	\begin{tabular}{l|ccccccc}
		\hline
		\hline
		Factor Group                   & (0,0.5] & (0.5, 0.6]& (0.6, 0.7] & (0.7, 0.8] & (0.8,0.9] & (0.9,1] & Mix \\
		Factor Amount                  & 12            & 10               & 17               & 37               & 35              & 34            & 20  \\ \hline
		Avg EN selection amount        & 2.11          & 4.47             & 8.67             & 14.67            & 13.51           & 12.37         & 8.45                    \\
		Avg EN selection proportion    & 17.5\%        & 44.73\%          & 51.00\%          & 39.65\%          & 38.61\%         & 36.38\%       & 42.28\%                 \\
		Avg Lasso selection amount     & 2.06          & 3.87             & 8.43             & 13               & 12.19           & 10.46         & 7.26                    \\
		Avg Lasso selection proportion & 17.2\%        & 38.76\%          & 49.60\%          & 35.14\%          & 34.83\%         & 30.75\%       & 36.27\%                 \\ \hline\hline
	\end{tabular}
}
\end{table}
We can see that the factor model selected by the Lasso regression, on average, is more parsimonious than the model selected by the Elastic net.
Such discrepancy between our tow methods increases with factor strength increase.
For the weak factor group with strength less than 0.5, the Elastic net and Lasso provides a similar answer: only two factors are selected out of twelve candidates.
While when focusing factor with strength above 0.9, Lasso will select almost 2 fewer factors than the Elastic net.
This result is not surprising, since the tuning parameter $\theta$ of Elastic net is closer to 0 than 1, which means that the Elastic net algorithm in our application tend to keep the factors even though they can provide very limited information.
Unexpectedly, the proportion of factor selection to the strong-factor group is significantly lower than the semi-strong group.
For the factor group with strength between 0.6 and 0.7, both Elastic net and Lasso select almost half of 17 candidates factors. 
But when facing strong factors with strength above 0.9, Elastic net on average only keep about 36\% factors.


We then compare every single stock's factor selecting decision made by Lasso and Elastic net, and calculates in what degree those two methods will agree with each other.
For every single company, if both Elastic net and Lasso select the same factor (generates factor loading not equals to zero), and disregard the same factor (generate factor loadings equal to zero), we call Elastic net and Lasso made an exact agreement of factor selecting.
%We also lower our comparison standard of the agreement to 90\% level. 
%If the Elastic net and Lasso achieve agreements in 90\% of factors selecting decision, and opposite for 10\% of factors, we would call they made an agreement on 90\% level.
%This step helps us to understand their selecting decision making from a broader view.
The proportion of agreement is presented in the table \ref{table:proportion_agreement}.

\begin{table}[h]
	\centering
	\caption{Proportion of Lasso Regression and Elastic Net produces same results for 145 companies}
	\label{table:proportion_agreement}
	\begin{tabular}{c|ccccccc}
		\hline
		\hline
		\multicolumn{1}{l|}{Factor Group}                                         & \multicolumn{1}{l}{(0,0.5{]}} & \multicolumn{1}{l}{(0.5, 0.6{]}} & \multicolumn{1}{l}{(0.6, 0.7{]}} & \multicolumn{1}{l}{(0.7, 0.8{]}} & \multicolumn{1}{l}{(0.8,0.9{]}} & \multicolumn{1}{l}{(0.9,1{]}} & Mix    \\ \hline
		\begin{tabular}[c]{@{}c@{}}Proportion of Agreement\\ (Exact)\end{tabular} & 68.7\%                        & 55.9\%                           & 42.8\%                           & 20.9\%                           & 17.7\%                          & 13.9\%                        & 34.6\% \\ \hline \hline
%		\begin{tabular}[c]{@{}c@{}}Proportion of Agreement\\ (90\%)\end{tabular}  & 86.8\%                        & 72.0\%                           & 74.5\%                           & 72.0\%                           & 79.8\%                          & 74.4\%                        & 76.1\% \\ \hline\hline
	\end{tabular}
\end{table}

It is clear that the proportion of agreement, both exact agreed and 90\% agreed, shows a decreasing trend with the factor strength increase.
Nearly 70\% of factors selection results are identical in the 0 to 0.5 strength group, but this number will decrease to 55\% after we move to the 0.5-0.6 factor strength group.
Only 14\% of companies have identical factor selection results for group with strength between 0.9 and 1.
For the mixed strength group, the exact agreed proportion is 34.6\%, ranked between the 0.6 to 0.7 group and 0.7 to 0.8 group.
%If we focusing on the 90\% agreement column, we can see that under a looser condition, Elastic net and Lasso had made more similar selecting decision.
%For the weak factor group, over 85\% stocks' factor selecting decision made by Elastic net and Lasso are close to each other.

To provide an explanation of the disagreement increasing with factor strength increasing, we calculate the average absolute correlation coefficient among factors for each factor groups.
For every two factors, we calculate the Pearson correlation coefficient, taking the absolute values and average all coefficients.
The result is presented in table \ref{table:Correlation} and Figure \ref{figure:correlation}.
\begin{table}[h]
	\centering
	\caption{Correlation Coefficient among different factor groups. }
	\label{table:Correlation}
	\begin{tabular}{l|cccccc}
		\hline
		\hline
		Factor Group                                 & (0,0.5{]} & (0.5, 0.6{]} & (0.6, 0.7{]} & (0.7, 0.8{]} & (0.8,0.9{]} & (0.9,1{]} \\ \hline
		\multicolumn{1}{c|}{Correlation Coefficient} & 0.0952    & 0.157        & 0.213        & 0.229        & 0.371       & 0.724   \\
		Factor Amount &12 & 10 &  17 & 37& 35 &34  \\ \hline \hline
	\end{tabular}
\end{table}
We can clearly see an increasing trend of the correlation coefficient with the increase of factor's strength.
This correlation pattern provides a possible explanation of the discord results of Elastic net and Lasso.
When facing variables with high correlation, Lasso will randomly select several variables and discard others \cite{Kozak2020}.
While Elastic net address this problem with the help of the extra $L^2$ norm in its loss function.
Therefore, the Elastic net method will select all factors that can bring new information to explain the risk-return relationship even those factors are highly correlated.
This also provides a potential explanation to what we observed in the table \ref{table:proportion}  that Lasso selects significantly fewer factors than the Elastic net method.
Also, recall the fact that in the strong factor group, Elastic net will select two more factors than the Lasso on average.
We believe Elastic net here will pick up factors that been abandoned by the Lasso due to the correlativity.

	\chapter{Conclusion and Possible Extension}\label{Conclusion}
		\section{Conclusion}
%In this thesis, we propose the concept of factor strength and the corresponding estimation method.
In this thesis, we implement the concept of factor strength and the corresponding estimation method to the problem of evaluating potential risk factors in multi-factor specifications 
We applied the estimation to 145 different potential risk factors plus the market factor, estimated their strength and used the strength as reference to categorised each factors to reduce the dimension of potential factor group.
On the basis of dimension-reduced factor group, we applied two feature selection methods namely Lasso and Elastic net, in order to eliminate the redundancy of factor groups. 

From the factor strength estimation, we have a consistent estimation result of most of the strong factors.

In general, we found that more than 20\% factors can be called as strong factors because they have strength above 0.9, and less than 10\% factors will be considered as weak factors with strength below 0.5.
For those strong factors, we found that their strength will remain consistent across different time periods.
But we also notice that the overall factor strength is significantly lower when we use data set contains short time-span, compared with results obtained from using larger data set with more time-series observations.
This might be due to financial events that occurred at different points during our sample periods.
Also, it worth to notice that some conventionally strong factors, for instance, the size factor, value factor, and the momentum factor do not show particular high strength.

The empirical results of feature selection indicates that Elastic net method and Lasso regression are both able to correctly and effectively eliminate weak factors without the ability to generating sufficient factor loadings.
In another word, both Elastic net and Lasso can identify factor with weak strength and disregard those redundant features.
However, since the factors with strong strength are almost all highly correlated, the Elastic net and Lasso fail to make a mutual agreement when facing strong factors.
On average, Elastic net will selects two more factors than Lasso when facing factors with strength above 0.9.
We also notice that when factors with different strength are mixed with each other, both lasso and Elastic net can effectively pick up factor with strong strength and disregard weak factors.

\section{Possible Extension}
During the empirical exploration, we discovered some possible extension of this thesis, and because of the time-limit as well as other restrictions, we could not propose those extensions.
Here, we summarise those potential extensions to provide a starting point for future research.

Firstly, when discussing the factor strength, we ignore the heterogeneity of companies and factors.
Ideally, companies and factors should be categorised based on their nature.
For instance, companies from different industries may react differently to different factors.
Therefore, one possible treatment is to group companies based on their industries types, or categorised factors based on their economical or financial meanings.
Also, \citeA{Harvey2019} suggests that newly proposed factors are more likely to encounter the multiple-testing problem, and hence they are more likely to be the factors that can not independently provides information to explain risk-return relationship.
So, we could also categorised the factors base on their published time, and to investigates the relationship between factor strength and time.

Secondly, as discussed in the chapter \ref{EN:parameter_tuning }, the principle of parameter tuning when applying Elastic net is to select the parameter combination that minimise the MSE.
However, the results present in the corresponding chapter indicates that MSE can not distinction different parameter combination significantly enough.
Hence, considering use other criteria may lead to better performance of parameter tuning, and therefore improve the results of the Elastic net application.

Also, some other feature selection methods or dimension reduced techniques can be taken into consideration.
The application of those methods can be used to cross-check with the results of Elastic net and Lasso.
Some potential methods including simple stepwise selection method, dantzig selector, and tree-based method like decision tree.







\newpage
\bibliographystyle{apacite}
\bibliography{library.bib}


\newpage
\appendix

%\documentclass[12pt]{article}
%\usepackage{amsmath}
%\usepackage{mathptmx}
%\usepackage{setspace}
%\usepackage{amssymb}
%\usepackage{float}\include{thesis}
%\usepackage{graphicx}
%\usepackage{booktabs}
%\usepackage{subfigure}
%\usepackage[margin=2.5cm]{geometry}
%\usepackage[title]{appendix}
%\usepackage{bm}
%\usepackage{tcolorbox}
%\usepackage{apacite}
%\onehalfspacing

%

%\begin{document}

	
	\chapter{Simulation Results}\label{simulationtable}
\begin{table}[!hbt]
		\caption{Simulation result for single factor setting}\label{table:exp1}
	\centering

	\begin{tabular}{lccccccccc}
		\hline
		\hline
		\multicolumn{1}{l|}{}                   & \multicolumn{9}{c}{Single Factor}                                                               \\ \hline
		\multicolumn{1}{l|}{}                   & \multicolumn{3}{c|}{Bias $\times$ 100}        & \multicolumn{3}{c|}{RMSE $\times$ 100}     & \multicolumn{3}{c}{Size $\times$ 100} \\ \hline
		\multicolumn{10}{c}{$\alpha_1 = 0.7$}                                                                                                                            \\ \hline
		\multicolumn{1}{l|}{n\textbackslash{}T} & 120    & 240    & \multicolumn{1}{c|}{360}    & 120   & 240   & \multicolumn{1}{c|}{360}   & 120         & 240         & 360        \\ \hline
		\multicolumn{1}{l|}{100}                & 0.256  & 0.265  & \multicolumn{1}{c|}{0.227}  & 0.612 & 0.623 & \multicolumn{1}{c|}{0.560} & 7.85        & 7.7         & 5.55       \\
		\multicolumn{1}{l|}{300}                & 0.185  & 0.184  & \multicolumn{1}{c|}{0.184}  & 0.363 & 0.338 & \multicolumn{1}{c|}{0.335} & 8.9         & 4.45        & 4.5        \\
		\multicolumn{1}{l|}{500}                & 0.107  & 0.124  & \multicolumn{1}{c|}{0.109}  & 0.259 & 0.248 & \multicolumn{1}{c|}{0.234} & 6.9         & 2.5         & 1.6        \\ \hline
		\multicolumn{10}{c}{$\alpha_1 = 0.75$}                                                                                                                          \\ \hline
		\multicolumn{1}{c|}{100}                & -0.178 & -0.159 & \multicolumn{1}{c|}{-0.168} & 0.490 & 0.465 & \multicolumn{1}{c|}{0.450} & 2.5         & 0.85        & 0.4        \\
		\multicolumn{1}{l|}{300}                & 0.154  & 0.156  & \multicolumn{1}{c|}{0.143}  & 0.281 & 0.258 & \multicolumn{1}{c|}{0.234} & 9.4         & 3.7         & 3.35       \\
		\multicolumn{1}{l|}{500}                & 0.024  & 0.033  & \multicolumn{1}{c|}{0.263}  & 0.171 & 0.155 & \multicolumn{1}{c|}{0.148} & 7.8         & 2           & 1.25       \\ \hline
		\multicolumn{10}{c}{$\alpha_1 = 0.8$}                                                                                                                           \\ \hline
		\multicolumn{1}{l|}{100}                & -0.270 & -0.265 & \multicolumn{1}{c|}{-0.258} & 0.434 & 0.409 & \multicolumn{1}{c|}{0.411} & 71.4        & 72.05       & 71.45      \\
		\multicolumn{1}{l|}{300}                & -0.052 & -0.044 & \multicolumn{1}{c|}{-0.043} & 0.183 & 0.149 & \multicolumn{1}{c|}{0.150} & 10.15       & 2.45        & 2.9        \\
		\multicolumn{1}{l|}{500}                & 0.045  & 0.068  & \multicolumn{1}{c|}{0.067}  & 0.136 & 0.126 & \multicolumn{1}{c|}{0.121} & 16.6        & 6.4         & 5.9        \\ \hline
		\multicolumn{10}{c}{$\alpha_1 = 0.85$}                                                                                                                          \\ \hline
		\multicolumn{1}{l|}{100}                & 0.053  & 0.062  & \multicolumn{1}{c|}{0.058}  & 0.253 & 0.228 & \multicolumn{1}{c|}{0.221} & 6.05        & 2.95        & 2.5        \\
		\multicolumn{1}{l|}{300}                & -0.012 & 0.009  & \multicolumn{1}{c|}{-0.001} & 0.124 & 0.104 & \multicolumn{1}{c|}{0.095} & 10.55       & 1.8         & 1.15       \\
		\multicolumn{1}{l|}{500}                & -0.026 & -0.007 & \multicolumn{1}{c|}{-0.011} & 0.096 & 0.073 & \multicolumn{1}{c|}{0.069} & 13.25       & 0.9         & 0.7        \\ \hline
		\multicolumn{10}{c}{$\alpha_1 = 0.9$}                                                                                                                           \\ \hline
		\multicolumn{1}{l|}{100}                & 0.025  & 0.038  & \multicolumn{1}{c|}{0.360}  & 0.191 & 0.163 & \multicolumn{1}{c|}{0.157} & 6.85        & 2           & 1.65       \\
		\multicolumn{1}{l|}{300}                & -0.034 & -0.018 & \multicolumn{1}{c|}{-0.020} & 0.099 & 0.069 & \multicolumn{1}{c|}{0.068} & 13.2        & 0.8         & 0.9        \\
		\multicolumn{1}{l|}{500}                & -0.025 & -0.001 & \multicolumn{1}{c|}{-0.001} & 0.072 & 0.044 & \multicolumn{1}{c|}{0.044} & 22.3        & 1.95        & 1.8        \\ \hline
		\multicolumn{10}{c}{$\alpha_1 = 0.95$}                                                                                                                          \\ \hline
		\multicolumn{1}{l|}{100}                & -0.099 & -0.088 & \multicolumn{1}{c|}{-0.090} & 0.156 & 0.125 & \multicolumn{1}{c|}{0.126} & 5.6         & 0.3         & 0.55       \\
		\multicolumn{1}{l|}{300}                & -0.046 & -0.025 & \multicolumn{1}{c|}{-0.026} & 0.083 & 0.045 & \multicolumn{1}{c|}{0.045} & 22.5        & 2.2         & 2.25       \\
		\multicolumn{1}{l|}{500}                & -0.030 & -0.006 & \multicolumn{1}{c|}{-0.006} & 0.061 & 0.026 & \multicolumn{1}{c|}{0.025} & 33.1        & 4.4         & 3.8        \\ \hline
		\multicolumn{10}{c}{$\alpha_1=1$}                                                                                                                               \\ \hline
		\multicolumn{1}{l|}{100}                & 0      & 0      & \multicolumn{1}{c|}{0}      & 0     & 0     & \multicolumn{1}{c|}{0}     & -           & -           & -          \\
		\multicolumn{1}{l|}{300}                & 0      & 0      & \multicolumn{1}{c|}{0}      & 0     & 0     & \multicolumn{1}{c|}{0}     & -           & -           & -          \\
		\multicolumn{1}{l|}{500}                & 0      & 0      & \multicolumn{1}{c|}{0}      & 0     & 0     & \multicolumn{1}{c|}{0}     & -           & -           & -          \\ \hline
\hline
\end{tabular}
\bigskip\\
{\bf Notes:}
This table shows the result of  experiment 1.
Factors and error are generate from standard normal distribution.
Factor loadings come form uniform distribution $IIDU(\mu_{\beta} - 0.2, \mu_{\beta}+0.2)$, and $\mu_{\beta} = 0.71$.
We keep  $[n^{\alpha_{j}}]$ amount of loadings and assign the rest as zero.
For each different time-unit combinations, we replicate 2000 times.
For the size of the test, we use a two-tail test, under the hypothesis of $H_0, \hat{\alpha}_j = \alpha_j\; j=1,2$.
Cause under the scenarios of $\alpha = 1$, the size of the test will collapse, therefore the table does not report the sizes for $\alpha_1 = 1$.
\end{table}


\begin{table}[!hbt]
		\caption{Simulation result for double factors setting (no correlation)}\label{table:exp2}
	\centering
	\begin{tabular}{lccccccccc}
		\hline
		\hline
\multicolumn{1}{l|}{}                   & \multicolumn{9}{c}{Double Factors with correlation $\rho_{12} = 0$}                                                                \\ \hline
\multicolumn{1}{l|}{}                   & \multicolumn{3}{c|}{Bias $\times$ 100}        & \multicolumn{3}{c|}{RMSE $\times$ 100}      & \multicolumn{3}{c}{Size $\times$ 100} \\ \hline
\multicolumn{10}{c}{$\alpha_1 = 1, \alpha_2  = 0.7$}                                                                                                                           \\ \hline
\multicolumn{1}{l|}{n\textbackslash{}T} & 120    & 240    & \multicolumn{1}{c|}{360}    & 120    & 240   & \multicolumn{1}{c|}{360}   & 120         & 240        & 360         \\ \hline
\multicolumn{1}{l|}{100}                & 0.567  & 0.737  & \multicolumn{1}{c|}{0.628}  & 4.062  & 3.819 & \multicolumn{1}{c|}{3.799} & 2.95        & 1.45       & 1.85        \\
\multicolumn{1}{l|}{300}                & 0.512  & 0.611  & \multicolumn{1}{c|}{0.518}  & 2.398  & 2.103 & \multicolumn{1}{c|}{1.979} & 6.25        & 0.55       & 0.5         \\
\multicolumn{1}{l|}{500}                & -0.149 & 0.08   & \multicolumn{1}{c|}{-0.019} & 1.796  & 1.498 & \multicolumn{1}{c|}{1.443} & 8           & 0.2        & 0.1         \\ \hline
\multicolumn{10}{c}{$\alpha_1 = 1, \alpha_2 =0.75$}                                                                                                                              \\ \hline
\multicolumn{1}{c|}{100}                & -3.051 & -3.02  & \multicolumn{1}{c|}{-3.092} & 4.582  & 4.245 & \multicolumn{1}{c|}{4.248} & 2.45        & 0.1        & 0.10        \\
\multicolumn{1}{l|}{300}                & 0.491  & -1.035 & \multicolumn{1}{c|}{0.640}  & 1.843  & 1.460 & \multicolumn{1}{c|}{1.576} & 7.6         & 0.8        & 0.55        \\
\multicolumn{1}{l|}{500}                & -0.611 & -0.372 & \multicolumn{1}{c|}{-0.393} & 1.520  & 1.136 & \multicolumn{1}{c|}{1.125} & 11.35       & 0.15       & 0.1         \\ \hline
\multicolumn{10}{c}{$\alpha _1= 1, \alpha_2 = 0.8$}                                                                                                                              \\ \hline
\multicolumn{1}{l|}{100}                & -3.752 & -3.630 & \multicolumn{1}{c|}{-3.581} & 4.557  & 4.213 & \multicolumn{1}{c|}{4.210} & 84.65       & 85.9       & 85.25       \\
\multicolumn{1}{l|}{300}                & -1.218 & -0.331 & \multicolumn{1}{c|}{-1.021} & 1.812  & 0.792 & \multicolumn{1}{c|}{1.438} & 9.35        & 0.2        & 0.3         \\
\multicolumn{1}{l|}{500}                & -0.022 & 0.192  & \multicolumn{1}{c|}{0.147}  & 1.047  & 0.782 & \multicolumn{1}{c|}{0.742} & 15.35       & 1.1        & 1.1         \\ \hline
\multicolumn{10}{c}{$\alpha_1 = 1, \alpha_2 = 0.85$}                                                                                                                             \\ \hline
\multicolumn{1}{l|}{100}                & -0.075 & 0.127  & \multicolumn{1}{c|}{0.088}  & 1.996  & 1.697 & \multicolumn{1}{c|}{1.606} & 5.4         & 1.15       & 0.95        \\
\multicolumn{1}{l|}{300}                & -0.531 & -0.406 & \multicolumn{1}{c|}{-0.351} & 1.097  & 0.613 & \multicolumn{1}{c|}{0.777} & 10.8        & 0.15       & 0.2         \\
\multicolumn{1}{l|}{500}                & -0.647 & -0.391 & \multicolumn{1}{c|}{-0.391} & 1.020  & 0.643 & \multicolumn{1}{c|}{0.630} & 19.1        & 0.15       & 0           \\ \hline
\multicolumn{10}{c}{$\alpha_1 = 1, \alpha_2 = 0.9$}                                                                                                                              \\ \hline
\multicolumn{1}{l|}{100}                & -0.128 & 0.043  & \multicolumn{1}{c|}{0.025}  & 1.428  & 1.143 & \multicolumn{1}{c|}{1.118} & 4.9         & 0.65       & 0.7         \\
\multicolumn{1}{l|}{300}                & -0.651 & -0.334 & \multicolumn{1}{c|}{-0.394} & 1.002  & 0.435 & \multicolumn{1}{c|}{0.617} & 17.1        & 0.6        & 0.2         \\
\multicolumn{1}{l|}{500}                & -0.434 & -0.168 & \multicolumn{1}{c|}{-0.171} & 0.7435 & 0.367 & \multicolumn{1}{c|}{0.368} & 25.2        & 0.4        & 0.3         \\ \hline
\multicolumn{10}{c}{$\alpha_1 = 1, \alpha_2 = 0.95$}                                                                                                                             \\ \hline
\multicolumn{1}{l|}{100}                & -1.218 & -1.043 & \multicolumn{1}{c|}{-1.036} & 1.603  & 1.222 & \multicolumn{1}{c|}{1.212} & 6.65        & 0.25       & 0.05        \\
\multicolumn{1}{l|}{300}                & -0.611 & -0.344 & \multicolumn{1}{c|}{-0.356} & 0.881  & 0.435 & \multicolumn{1}{c|}{0.434} & 23.35       & 0.6        & 0.45        \\
\multicolumn{1}{l|}{500}                & -0.415 & -0.123 & \multicolumn{1}{c|}{-0.134} & 0.661  & 0.220 & \multicolumn{1}{c|}{0.216} & 36.75       & 1.35       & 1.1         \\ \hline
\multicolumn{10}{c}{$\alpha_1=1, \alpha_2 = 1$}                                                                                                                                  \\ \hline
\multicolumn{1}{l|}{100}                & 0      & 0      & \multicolumn{1}{c|}{0}      & 0      & 0     & \multicolumn{1}{c|}{0}     & -           & -          & -           \\
\multicolumn{1}{l|}{300}                & 0      & 0      & \multicolumn{1}{c|}{0}      & 0      & 0     & \multicolumn{1}{c|}{0}     & -           & -          & -           \\
\multicolumn{1}{l|}{500}                & 0      & 0      & \multicolumn{1}{c|}{0}      & 0      & 0     & \multicolumn{1}{c|}{0}     & -           & -          & -           \\ \hline\hline
	\end{tabular}
\bigskip\\
{\bf Notes:}
This table shows the result of  experiment 2.
Factors and errors are generate from standard normal distribution.
Between two factors, we assume they have no correlation.
Factor loadings come form uniform distribution $IIDU(\mu_{\beta} - 0.2, \mu_{\beta}+0.2)$, and $\mu_{\beta}$ is set to 0.71.
We keep  $[n^{\alpha_{j}}]$ amount of loadings and assign the rest as zero.
For each different time-unit combinations, we replicate 2000 times.
For the size of the test, we use a two-tail test, under the hypothesis of $H_0, \hat{\alpha}_j = \alpha_j\; j=1,2$.
Cause under the scenarios of $\alpha = 1$, the size of the test will collapse, therefore the table does not report the sizes for $\alpha_1 = \alpha_2 = 1$
\end{table}


	\begin{table}[]
		\caption{Simulation result for double factors setting (weak correlation)}\label{table:exp3}
		\centering
	\begin{tabular}{lccccccccc}
		\hline
		\hline
		\multicolumn{1}{l|}{}                   & \multicolumn{9}{c}{Double Factors with correlation $\rho_{12} = 0.3$}                                                               \\ \hline
		\multicolumn{1}{l|}{}                   & \multicolumn{3}{c|}{Bias $\times$ 100}        & \multicolumn{3}{c|}{RMSE $\times$ 100}     & \multicolumn{3}{c}{Size $\times$ 100} \\ \hline
		\multicolumn{10}{c}{$\alpha_1 = 1, \alpha_2 = 0.7$}                                                                                                                            \\ \hline
		\multicolumn{1}{l|}{n\textbackslash{}T} & 120    & 240    & \multicolumn{1}{c|}{360}    & 120   & 240   & \multicolumn{1}{c|}{360}   & 120          & 240        & 360        \\ \hline
		\multicolumn{1}{l|}{100}                & 0.038  & 0.064  & \multicolumn{1}{c|}{0.072}  & 0.421 & 0.382 & \multicolumn{1}{c|}{0.389} & 4.6          & 1.75       & 1.95       \\
		\multicolumn{1}{l|}{300}                & 0.021  & 0.058  & \multicolumn{1}{c|}{0.056}  & 0.253 & 0.206 & \multicolumn{1}{c|}{0.198} & 9.95         & 0.9        & 0.25       \\
		\multicolumn{1}{l|}{500}                & -0.032 & 0.006  & \multicolumn{1}{c|}{0}      & 0.201 & 0.153 & \multicolumn{1}{c|}{0}     & 12.20        & 0.1        & 0.05       \\ \hline
		\multicolumn{10}{c}{$\alpha_1 = 1, \alpha_2 = 0.75$}                                                                                                                          \\ \hline
		\multicolumn{1}{c|}{100}                & -0.325 & -0.313 & \multicolumn{1}{c|}{-0.310} & 0.488 & 0.419 & \multicolumn{1}{c|}{0.420} & 4.75         & 0.1        & 0          \\
		\multicolumn{1}{l|}{300}                & 0.028  & 0.063  & \multicolumn{1}{c|}{0.065}  & 0.253 & 0.157 & \multicolumn{1}{c|}{0.159} & 9.95         & 0.55       & 0.5        \\
		\multicolumn{1}{l|}{500}                & -0.082 & -0.037 & \multicolumn{1}{c|}{-0.039} & 0.175 & 0.114 & \multicolumn{1}{c|}{0.112} & 19.25        & 0.25       & 0.3        \\ \hline
		\multicolumn{10}{c}{$\alpha_1 = 1, \alpha_2 = 0.8$}                                                                                                                           \\ \hline
		\multicolumn{1}{l|}{100}                & -0.393 & -0.361 & \multicolumn{1}{c|}{-0.368} & 0.477 & 0.418 & \multicolumn{1}{c|}{0.421} & 85.45        & 85.2       & 86.4       \\
		\multicolumn{1}{l|}{300}                & 0.029  & -0.099 & \multicolumn{1}{c|}{-0.100} & 0.192 & 0.145 & \multicolumn{1}{c|}{0.145} & 12.2         & 0.65       & 0.5        \\
		\multicolumn{1}{l|}{500}                & -0.037 & -0.016 & \multicolumn{1}{c|}{0.016}  & 0.129 & 0.074 & \multicolumn{1}{c|}{0.074} & 27.8         & 0.25       & 1.2        \\ \hline
		\multicolumn{10}{c}{$\alpha_1 = 1, \alpha_2 = 0.85$}                                                                                                                          \\ \hline
		\multicolumn{1}{l|}{100}                & -0.027 & 0.008  & \multicolumn{1}{c|}{0.007}  & 0.234 & 0.160 & \multicolumn{1}{c|}{0.155} & 9.3          & 0.9        & 0.65       \\
		\multicolumn{1}{l|}{300}                & -0.147 & -0.031 & \multicolumn{1}{c|}{-0.037} & 0.219 & 0.079 & \multicolumn{1}{c|}{0.077} & 16.75        & 0.3        & 0.2        \\
		\multicolumn{1}{l|}{500}                & -0.088 & -0.039 & \multicolumn{1}{c|}{-0.039} & 0.136 & 0.063 & \multicolumn{1}{c|}{0.062} & 30.6         & 0.15       & 0          \\ \hline
		\multicolumn{10}{c}{$\alpha_1 = 1, \alpha_2 = 0.9$}                                                                                                                           \\ \hline
		\multicolumn{1}{l|}{100}                & -0.033 & 0.003  & \multicolumn{1}{c|}{0.002}  & 0.173 & 0.111 & \multicolumn{1}{c|}{0.110} & 9.4          & 0.6        & 0.55       \\
		\multicolumn{1}{l|}{300}                & -0.087 & -0.040 & \multicolumn{1}{c|}{-0.041} & 0.131 & 0.061 & \multicolumn{1}{c|}{0.061} & 27.8         & 0.1        & 0.05       \\
		\multicolumn{1}{l|}{500}                & -0.070 & -0.017 & \multicolumn{1}{c|}{-0.018} & 0.111 & 0.037 & \multicolumn{1}{c|}{0.037} & 41.15        & 0.6        & 0.35       \\ \hline
		\multicolumn{10}{c}{$\alpha_1 = 1, \alpha_2 = 0.95$}                                                                                                                          \\ \hline
		\multicolumn{1}{l|}{100}                & -0.134 & -0.101 & \multicolumn{1}{c|}{-0.104} & 0.185 & 0.122 & \multicolumn{1}{c|}{0.122} & 10.15        & 0.1        & 0.15       \\
		\multicolumn{1}{l|}{300}                & -0.083 & -0.034 & \multicolumn{1}{c|}{-0.034} & 0.118 & 0.043 & \multicolumn{1}{c|}{0.044} & 39.35        & 0.6        & 0.6        \\
		\multicolumn{1}{l|}{500}                & -0.062 & -0.013 & \multicolumn{1}{c|}{-0.012} & 0.937 & 0.022 & \multicolumn{1}{c|}{0.023} & 51.8         & 1.25       & 2.0        \\ \hline
		\multicolumn{10}{c}{$\alpha_1=1, \alpha_2 = 1$}                                                                                                                               \\ \hline
		\multicolumn{1}{l|}{100}                & 0      & 0      & \multicolumn{1}{c|}{0}      & 0     & 0     & \multicolumn{1}{c|}{0}     & -            & -          & -          \\
		\multicolumn{1}{l|}{300}                & 0      & 0      & \multicolumn{1}{c|}{0}      & 0     & 0     & \multicolumn{1}{c|}{0}     & -            & -          & -          \\
		\multicolumn{1}{l|}{500}                & 0      & 0      & \multicolumn{1}{c|}{0}      & 0     & 0     & \multicolumn{1}{c|}{0}     & -            & -          & -          \\ \hline\hline
	\end{tabular}
\bigskip \\
{\bf Notes:}
This table shows the result of  experiment 3.
Factors and errors are generate from standard normal distribution.
Between two factors, we assume they have correlation $\rho_{12} = 0.3$
Factor loadings come form uniform distribution $IIDU(\mu_{\beta} - 0.2, \mu_{\beta}+0.2)$, and $\mu_{\beta}$ is set to 0.71.
We keep  $[n^{\alpha_{j}}]$ amount of loadings and assign the rest as zero.
For each different time-unit combinations, we replicate 2000 times.
For the size of the test, we use a two-tail test, under the hypothesis of $H_0, \hat{\alpha}_j = \alpha_j\; j=1,2$.
Cause under the scenarios of $\alpha = 1$, the size of the test will collapse, therefore the table does not report the sizes when $\alpha_1 = \alpha_2 = 1$
\end{table}


\begin{table}[]
	\caption{Simulation result for double factors setting (strong correlation)}\label{table:exp4}
	\centering
	\begin{tabular}{lccccccccc}
		\hline
\hline
\multicolumn{1}{l|}{}    & \multicolumn{9}{c}{Double Factors with correlation $\rho_{12} = 0.7$}                                                                                                  \\ \hline
\multicolumn{1}{l|}{}    & \multicolumn{3}{c|}{Bias $\times$ 100}        & \multicolumn{3}{c|}{RMSE $\times$ 100}     & \multicolumn{3}{c}{Size $\times$ 100} \\ \hline
\multicolumn{10}{c}{$\alpha_1 = 0.7$}                                                                                                                         \\ \hline
\multicolumn{1}{l|}{n\textbackslash T}  & 120    & 240    & \multicolumn{1}{c|}{360}    & 120   & 240   & \multicolumn{1}{c|}{360}   & 120         & 240        & 360        \\ \hline
\multicolumn{1}{l|}{100} & -0.659 & 0.069  & \multicolumn{1}{c|}{0.059}  & 1.193 & 0.406 & \multicolumn{1}{c|}{0.372} & 46.95       & 2.65       & 1.25       \\
\multicolumn{1}{l|}{300} & -0.704 & 0.048  & \multicolumn{1}{c|}{0.055}  & 0.983 & 0.223 & \multicolumn{1}{c|}{0.203} & 75.55       & 3.7        & 0.9        \\
\multicolumn{1}{l|}{500} & -0.841 & -0.011 & \multicolumn{1}{c|}{-0.004} & 1.056 & 0.161 & \multicolumn{1}{c|}{0.144} & 86.4        & 5.1        & 0.3        \\ \hline
\multicolumn{10}{c}{$\alpha_1 = 0.75$}                                                                                                                        \\ \hline
\multicolumn{1}{c|}{100} & -1.03  & -0.315 & \multicolumn{1}{c|}{-0.323} & 1.360 & 0.443 & \multicolumn{1}{c|}{0.421} & 53.65       & 1.35       & 0.05       \\
\multicolumn{1}{l|}{300} & -0.724 & 0.049  & \multicolumn{1}{c|}{0.065}  & 0.947 & 0.166 & \multicolumn{1}{c|}{0.159} & 84.75       & 4.85       & 0.6        \\
\multicolumn{1}{l|}{500} & -0.877 & -0.055 & \multicolumn{1}{c|}{-0.04}  & 1.036 & 0.130 & \multicolumn{1}{c|}{0.111} & 93.5        & 6.5        & 0.15       \\ \hline
\multicolumn{10}{c}{$\alpha_1 = 0.8$}                                                                                                                         \\ \hline
\multicolumn{1}{l|}{100} & -1.099 & -0.374 & \multicolumn{1}{c|}{-0.362} & 1.36  & 0.436 & \multicolumn{1}{c|}{0.421} & 94.6        & 86.05      & 85.75      \\
\multicolumn{1}{l|}{300} & -0.9   & -0.114 & \multicolumn{1}{c|}{-0.103} & 1.071 & 0.165 & \multicolumn{1}{c|}{0.144} & 91.95       & 5.9        & 0.35       \\
\multicolumn{1}{l|}{500} & -0.822 & 0.005  & \multicolumn{1}{c|}{0.015}  & 0.968 & 0.086 & \multicolumn{1}{c|}{0.072} & 97.3        & 9.2        & 0.8        \\ \hline
\multicolumn{10}{c}{$\alpha_1 = 0.85$}                                                                                                                        \\ \hline
\multicolumn{1}{l|}{100} & -0.722 & -0.001 & \multicolumn{1}{c|}{0.009}  & 1.006 & 0.175 & \multicolumn{1}{c|}{0.157} & 71.75       & 3.00       & 0.65       \\
\multicolumn{1}{l|}{300} & -0.834 & -0.05  & \multicolumn{1}{c|}{-0.036} & 0.989 & 0.101 & \multicolumn{1}{c|}{0.078} & 95.25       & 8.65       & 0.3        \\
\multicolumn{1}{l|}{500} & -0.883 & -0.057 & \multicolumn{1}{c|}{-0.039} & 1.013 & 0.084 & \multicolumn{1}{c|}{0.062} & 98.95       & 13.3       & 0.15       \\ \hline
\multicolumn{10}{c}{$\alpha_1 = 0.9$}                                                                                                                         \\ \hline
\multicolumn{1}{l|}{100} & -0.723 & -0.004 & \multicolumn{1}{c|}{0.001}  & 0.972 & 0.125 & \multicolumn{1}{c|}{0.107} & 77.35       & 2.8        & 0.5        \\
\multicolumn{1}{l|}{300} & -0.872 & -0.055 & \multicolumn{1}{c|}{-0.04}  & 1.011 & 0.084 & \multicolumn{1}{c|}{0.062} & 98.2        & 11.1       & 0.25       \\
\multicolumn{1}{l|}{500} & -0.851 & -0.033 & \multicolumn{1}{c|}{-0.018} & 0.967 & 0.06  & \multicolumn{1}{c|}{0.037} & 99.75       & 17.7       & 0.75       \\ \hline
\multicolumn{10}{c}{$\alpha_1 = 0.95$}                                                                                                                        \\ \hline
\multicolumn{1}{l|}{100} & -0.879 & -0.116 & \multicolumn{1}{c|}{-0.103} & 1.083 & 0.143 & \multicolumn{1}{c|}{0.122} & 86.1        & 3.85       & 0.2        \\
\multicolumn{1}{l|}{300} & -0.853 & -0.049 & \multicolumn{1}{c|}{-0.035} & 0.977 & 0.066 & \multicolumn{1}{c|}{0.044} & 99.55       & 14.65      & 1.2        \\
\multicolumn{1}{l|}{500} & -0.875 & -0.029 & \multicolumn{1}{c|}{-0.014} & 0.987 & 0.046 & \multicolumn{1}{c|}{0.022} & 99.85       & 26.65      & 1.55       \\ \hline
\multicolumn{10}{c}{$\alpha_1=1$}                                                                                                                             \\ \hline
\multicolumn{1}{l|}{100} & -0.76  & -0.012 & \multicolumn{1}{c|}{0}      & 0.956 & 0054  & \multicolumn{1}{c|}{0.009} & -           & -          & -          \\
\multicolumn{1}{l|}{300} & -0.811 & -0.015 & \multicolumn{1}{c|}{0}      & 0.945 & 0.037 & \multicolumn{1}{c|}{0.004} & -           & -          & -          \\
\multicolumn{1}{l|}{500} & -0.848 & -0.017 & \multicolumn{1}{c|}{0}      & 0.96  & 0.033 & \multicolumn{1}{c|}{0.003} & -           & -          & -          \\ \hline
\hline
	\end{tabular}
	\bigskip \\
	{\bf Notes:}
	This table shows the result of  experiment 4.
	Factors and errors are generate from standard normal distribution.
	Between two factors, we assume they have correlation $\rho_{12} = 0.7$
	Factor loadings come form uniform distribution $IIDU(\mu_{\beta} - 0.2, \mu_{\beta}+0.2)$, and $\mu_{\beta}$ is set to 0.71.
	We keep  $[n^{\alpha_{j}}]$ amount of loadings and assign the rest as zero.
	For each different time-unit combinations, we replicate 2000 times.
	For the size of the test, we use a two-tail test, under the hypothesis of $H_0, \hat{\alpha}_j = \alpha_j\; j=1,2$.
	Cause under the scenarios of $\alpha = 1$, the size of the test will collapse, therefore the table does not report the sizes when $\alpha_1 = \alpha_2 = 1$
\end{table}


%\end{document}
\input{comparison_table.tex}

\end{document}
%\usepackage{graphicx}
%\usepackage{booktabs}
%\usepackage{subfigure}
%\usepackage[margin=2.5cm]{geometry}
%\usepackage[title]{appendix}
%\usepackage{bm}
%\usepackage{tcolorbox}
%\usepackage{apacite}
%\onehalfspacing

%

%\begin{document}

	
	\chapter{Simulation Results}\label{simulationtable}
\begin{table}[!hbt]
		\caption{Simulation result for single factor setting}\label{table:exp1}
	\centering

	\begin{tabular}{lccccccccc}
		\hline
		\hline
		\multicolumn{1}{l|}{}                   & \multicolumn{9}{c}{Single Factor}                                                               \\ \hline
		\multicolumn{1}{l|}{}                   & \multicolumn{3}{c|}{Bias $\times$ 100}        & \multicolumn{3}{c|}{RMSE $\times$ 100}     & \multicolumn{3}{c}{Size $\times$ 100} \\ \hline
		\multicolumn{10}{c}{$\alpha_1 = 0.7$}                                                                                                                            \\ \hline
		\multicolumn{1}{l|}{n\textbackslash{}T} & 120    & 240    & \multicolumn{1}{c|}{360}    & 120   & 240   & \multicolumn{1}{c|}{360}   & 120         & 240         & 360        \\ \hline
		\multicolumn{1}{l|}{100}                & 0.256  & 0.265  & \multicolumn{1}{c|}{0.227}  & 0.612 & 0.623 & \multicolumn{1}{c|}{0.560} & 7.85        & 7.7         & 5.55       \\
		\multicolumn{1}{l|}{300}                & 0.185  & 0.184  & \multicolumn{1}{c|}{0.184}  & 0.363 & 0.338 & \multicolumn{1}{c|}{0.335} & 8.9         & 4.45        & 4.5        \\
		\multicolumn{1}{l|}{500}                & 0.107  & 0.124  & \multicolumn{1}{c|}{0.109}  & 0.259 & 0.248 & \multicolumn{1}{c|}{0.234} & 6.9         & 2.5         & 1.6        \\ \hline
		\multicolumn{10}{c}{$\alpha_1 = 0.75$}                                                                                                                          \\ \hline
		\multicolumn{1}{c|}{100}                & -0.178 & -0.159 & \multicolumn{1}{c|}{-0.168} & 0.490 & 0.465 & \multicolumn{1}{c|}{0.450} & 2.5         & 0.85        & 0.4        \\
		\multicolumn{1}{l|}{300}                & 0.154  & 0.156  & \multicolumn{1}{c|}{0.143}  & 0.281 & 0.258 & \multicolumn{1}{c|}{0.234} & 9.4         & 3.7         & 3.35       \\
		\multicolumn{1}{l|}{500}                & 0.024  & 0.033  & \multicolumn{1}{c|}{0.263}  & 0.171 & 0.155 & \multicolumn{1}{c|}{0.148} & 7.8         & 2           & 1.25       \\ \hline
		\multicolumn{10}{c}{$\alpha_1 = 0.8$}                                                                                                                           \\ \hline
		\multicolumn{1}{l|}{100}                & -0.270 & -0.265 & \multicolumn{1}{c|}{-0.258} & 0.434 & 0.409 & \multicolumn{1}{c|}{0.411} & 71.4        & 72.05       & 71.45      \\
		\multicolumn{1}{l|}{300}                & -0.052 & -0.044 & \multicolumn{1}{c|}{-0.043} & 0.183 & 0.149 & \multicolumn{1}{c|}{0.150} & 10.15       & 2.45        & 2.9        \\
		\multicolumn{1}{l|}{500}                & 0.045  & 0.068  & \multicolumn{1}{c|}{0.067}  & 0.136 & 0.126 & \multicolumn{1}{c|}{0.121} & 16.6        & 6.4         & 5.9        \\ \hline
		\multicolumn{10}{c}{$\alpha_1 = 0.85$}                                                                                                                          \\ \hline
		\multicolumn{1}{l|}{100}                & 0.053  & 0.062  & \multicolumn{1}{c|}{0.058}  & 0.253 & 0.228 & \multicolumn{1}{c|}{0.221} & 6.05        & 2.95        & 2.5        \\
		\multicolumn{1}{l|}{300}                & -0.012 & 0.009  & \multicolumn{1}{c|}{-0.001} & 0.124 & 0.104 & \multicolumn{1}{c|}{0.095} & 10.55       & 1.8         & 1.15       \\
		\multicolumn{1}{l|}{500}                & -0.026 & -0.007 & \multicolumn{1}{c|}{-0.011} & 0.096 & 0.073 & \multicolumn{1}{c|}{0.069} & 13.25       & 0.9         & 0.7        \\ \hline
		\multicolumn{10}{c}{$\alpha_1 = 0.9$}                                                                                                                           \\ \hline
		\multicolumn{1}{l|}{100}                & 0.025  & 0.038  & \multicolumn{1}{c|}{0.360}  & 0.191 & 0.163 & \multicolumn{1}{c|}{0.157} & 6.85        & 2           & 1.65       \\
		\multicolumn{1}{l|}{300}                & -0.034 & -0.018 & \multicolumn{1}{c|}{-0.020} & 0.099 & 0.069 & \multicolumn{1}{c|}{0.068} & 13.2        & 0.8         & 0.9        \\
		\multicolumn{1}{l|}{500}                & -0.025 & -0.001 & \multicolumn{1}{c|}{-0.001} & 0.072 & 0.044 & \multicolumn{1}{c|}{0.044} & 22.3        & 1.95        & 1.8        \\ \hline
		\multicolumn{10}{c}{$\alpha_1 = 0.95$}                                                                                                                          \\ \hline
		\multicolumn{1}{l|}{100}                & -0.099 & -0.088 & \multicolumn{1}{c|}{-0.090} & 0.156 & 0.125 & \multicolumn{1}{c|}{0.126} & 5.6         & 0.3         & 0.55       \\
		\multicolumn{1}{l|}{300}                & -0.046 & -0.025 & \multicolumn{1}{c|}{-0.026} & 0.083 & 0.045 & \multicolumn{1}{c|}{0.045} & 22.5        & 2.2         & 2.25       \\
		\multicolumn{1}{l|}{500}                & -0.030 & -0.006 & \multicolumn{1}{c|}{-0.006} & 0.061 & 0.026 & \multicolumn{1}{c|}{0.025} & 33.1        & 4.4         & 3.8        \\ \hline
		\multicolumn{10}{c}{$\alpha_1=1$}                                                                                                                               \\ \hline
		\multicolumn{1}{l|}{100}                & 0      & 0      & \multicolumn{1}{c|}{0}      & 0     & 0     & \multicolumn{1}{c|}{0}     & -           & -           & -          \\
		\multicolumn{1}{l|}{300}                & 0      & 0      & \multicolumn{1}{c|}{0}      & 0     & 0     & \multicolumn{1}{c|}{0}     & -           & -           & -          \\
		\multicolumn{1}{l|}{500}                & 0      & 0      & \multicolumn{1}{c|}{0}      & 0     & 0     & \multicolumn{1}{c|}{0}     & -           & -           & -          \\ \hline
\hline
\end{tabular}
\bigskip\\
{\bf Notes:}
This table shows the result of  experiment 1.
Factors and error are generate from standard normal distribution.
Factor loadings come form uniform distribution $IIDU(\mu_{\beta} - 0.2, \mu_{\beta}+0.2)$, and $\mu_{\beta} = 0.71$.
We keep  $[n^{\alpha_{j}}]$ amount of loadings and assign the rest as zero.
For each different time-unit combinations, we replicate 2000 times.
For the size of the test, we use a two-tail test, under the hypothesis of $H_0, \hat{\alpha}_j = \alpha_j\; j=1,2$.
Cause under the scenarios of $\alpha = 1$, the size of the test will collapse, therefore the table does not report the sizes for $\alpha_1 = 1$.
\end{table}


\begin{table}[!hbt]
		\caption{Simulation result for double factors setting (no correlation)}\label{table:exp2}
	\centering
	\begin{tabular}{lccccccccc}
		\hline
		\hline
\multicolumn{1}{l|}{}                   & \multicolumn{9}{c}{Double Factors with correlation $\rho_{12} = 0$}                                                                \\ \hline
\multicolumn{1}{l|}{}                   & \multicolumn{3}{c|}{Bias $\times$ 100}        & \multicolumn{3}{c|}{RMSE $\times$ 100}      & \multicolumn{3}{c}{Size $\times$ 100} \\ \hline
\multicolumn{10}{c}{$\alpha_1 = 1, \alpha_2  = 0.7$}                                                                                                                           \\ \hline
\multicolumn{1}{l|}{n\textbackslash{}T} & 120    & 240    & \multicolumn{1}{c|}{360}    & 120    & 240   & \multicolumn{1}{c|}{360}   & 120         & 240        & 360         \\ \hline
\multicolumn{1}{l|}{100}                & 0.567  & 0.737  & \multicolumn{1}{c|}{0.628}  & 4.062  & 3.819 & \multicolumn{1}{c|}{3.799} & 2.95        & 1.45       & 1.85        \\
\multicolumn{1}{l|}{300}                & 0.512  & 0.611  & \multicolumn{1}{c|}{0.518}  & 2.398  & 2.103 & \multicolumn{1}{c|}{1.979} & 6.25        & 0.55       & 0.5         \\
\multicolumn{1}{l|}{500}                & -0.149 & 0.08   & \multicolumn{1}{c|}{-0.019} & 1.796  & 1.498 & \multicolumn{1}{c|}{1.443} & 8           & 0.2        & 0.1         \\ \hline
\multicolumn{10}{c}{$\alpha_1 = 1, \alpha_2 =0.75$}                                                                                                                              \\ \hline
\multicolumn{1}{c|}{100}                & -3.051 & -3.02  & \multicolumn{1}{c|}{-3.092} & 4.582  & 4.245 & \multicolumn{1}{c|}{4.248} & 2.45        & 0.1        & 0.10        \\
\multicolumn{1}{l|}{300}                & 0.491  & -1.035 & \multicolumn{1}{c|}{0.640}  & 1.843  & 1.460 & \multicolumn{1}{c|}{1.576} & 7.6         & 0.8        & 0.55        \\
\multicolumn{1}{l|}{500}                & -0.611 & -0.372 & \multicolumn{1}{c|}{-0.393} & 1.520  & 1.136 & \multicolumn{1}{c|}{1.125} & 11.35       & 0.15       & 0.1         \\ \hline
\multicolumn{10}{c}{$\alpha _1= 1, \alpha_2 = 0.8$}                                                                                                                              \\ \hline
\multicolumn{1}{l|}{100}                & -3.752 & -3.630 & \multicolumn{1}{c|}{-3.581} & 4.557  & 4.213 & \multicolumn{1}{c|}{4.210} & 84.65       & 85.9       & 85.25       \\
\multicolumn{1}{l|}{300}                & -1.218 & -0.331 & \multicolumn{1}{c|}{-1.021} & 1.812  & 0.792 & \multicolumn{1}{c|}{1.438} & 9.35        & 0.2        & 0.3         \\
\multicolumn{1}{l|}{500}                & -0.022 & 0.192  & \multicolumn{1}{c|}{0.147}  & 1.047  & 0.782 & \multicolumn{1}{c|}{0.742} & 15.35       & 1.1        & 1.1         \\ \hline
\multicolumn{10}{c}{$\alpha_1 = 1, \alpha_2 = 0.85$}                                                                                                                             \\ \hline
\multicolumn{1}{l|}{100}                & -0.075 & 0.127  & \multicolumn{1}{c|}{0.088}  & 1.996  & 1.697 & \multicolumn{1}{c|}{1.606} & 5.4         & 1.15       & 0.95        \\
\multicolumn{1}{l|}{300}                & -0.531 & -0.406 & \multicolumn{1}{c|}{-0.351} & 1.097  & 0.613 & \multicolumn{1}{c|}{0.777} & 10.8        & 0.15       & 0.2         \\
\multicolumn{1}{l|}{500}                & -0.647 & -0.391 & \multicolumn{1}{c|}{-0.391} & 1.020  & 0.643 & \multicolumn{1}{c|}{0.630} & 19.1        & 0.15       & 0           \\ \hline
\multicolumn{10}{c}{$\alpha_1 = 1, \alpha_2 = 0.9$}                                                                                                                              \\ \hline
\multicolumn{1}{l|}{100}                & -0.128 & 0.043  & \multicolumn{1}{c|}{0.025}  & 1.428  & 1.143 & \multicolumn{1}{c|}{1.118} & 4.9         & 0.65       & 0.7         \\
\multicolumn{1}{l|}{300}                & -0.651 & -0.334 & \multicolumn{1}{c|}{-0.394} & 1.002  & 0.435 & \multicolumn{1}{c|}{0.617} & 17.1        & 0.6        & 0.2         \\
\multicolumn{1}{l|}{500}                & -0.434 & -0.168 & \multicolumn{1}{c|}{-0.171} & 0.7435 & 0.367 & \multicolumn{1}{c|}{0.368} & 25.2        & 0.4        & 0.3         \\ \hline
\multicolumn{10}{c}{$\alpha_1 = 1, \alpha_2 = 0.95$}                                                                                                                             \\ \hline
\multicolumn{1}{l|}{100}                & -1.218 & -1.043 & \multicolumn{1}{c|}{-1.036} & 1.603  & 1.222 & \multicolumn{1}{c|}{1.212} & 6.65        & 0.25       & 0.05        \\
\multicolumn{1}{l|}{300}                & -0.611 & -0.344 & \multicolumn{1}{c|}{-0.356} & 0.881  & 0.435 & \multicolumn{1}{c|}{0.434} & 23.35       & 0.6        & 0.45        \\
\multicolumn{1}{l|}{500}                & -0.415 & -0.123 & \multicolumn{1}{c|}{-0.134} & 0.661  & 0.220 & \multicolumn{1}{c|}{0.216} & 36.75       & 1.35       & 1.1         \\ \hline
\multicolumn{10}{c}{$\alpha_1=1, \alpha_2 = 1$}                                                                                                                                  \\ \hline
\multicolumn{1}{l|}{100}                & 0      & 0      & \multicolumn{1}{c|}{0}      & 0      & 0     & \multicolumn{1}{c|}{0}     & -           & -          & -           \\
\multicolumn{1}{l|}{300}                & 0      & 0      & \multicolumn{1}{c|}{0}      & 0      & 0     & \multicolumn{1}{c|}{0}     & -           & -          & -           \\
\multicolumn{1}{l|}{500}                & 0      & 0      & \multicolumn{1}{c|}{0}      & 0      & 0     & \multicolumn{1}{c|}{0}     & -           & -          & -           \\ \hline\hline
	\end{tabular}
\bigskip\\
{\bf Notes:}
This table shows the result of  experiment 2.
Factors and errors are generate from standard normal distribution.
Between two factors, we assume they have no correlation.
Factor loadings come form uniform distribution $IIDU(\mu_{\beta} - 0.2, \mu_{\beta}+0.2)$, and $\mu_{\beta}$ is set to 0.71.
We keep  $[n^{\alpha_{j}}]$ amount of loadings and assign the rest as zero.
For each different time-unit combinations, we replicate 2000 times.
For the size of the test, we use a two-tail test, under the hypothesis of $H_0, \hat{\alpha}_j = \alpha_j\; j=1,2$.
Cause under the scenarios of $\alpha = 1$, the size of the test will collapse, therefore the table does not report the sizes for $\alpha_1 = \alpha_2 = 1$
\end{table}


	\begin{table}[]
		\caption{Simulation result for double factors setting (weak correlation)}\label{table:exp3}
		\centering
	\begin{tabular}{lccccccccc}
		\hline
		\hline
		\multicolumn{1}{l|}{}                   & \multicolumn{9}{c}{Double Factors with correlation $\rho_{12} = 0.3$}                                                               \\ \hline
		\multicolumn{1}{l|}{}                   & \multicolumn{3}{c|}{Bias $\times$ 100}        & \multicolumn{3}{c|}{RMSE $\times$ 100}     & \multicolumn{3}{c}{Size $\times$ 100} \\ \hline
		\multicolumn{10}{c}{$\alpha_1 = 1, \alpha_2 = 0.7$}                                                                                                                            \\ \hline
		\multicolumn{1}{l|}{n\textbackslash{}T} & 120    & 240    & \multicolumn{1}{c|}{360}    & 120   & 240   & \multicolumn{1}{c|}{360}   & 120          & 240        & 360        \\ \hline
		\multicolumn{1}{l|}{100}                & 0.038  & 0.064  & \multicolumn{1}{c|}{0.072}  & 0.421 & 0.382 & \multicolumn{1}{c|}{0.389} & 4.6          & 1.75       & 1.95       \\
		\multicolumn{1}{l|}{300}                & 0.021  & 0.058  & \multicolumn{1}{c|}{0.056}  & 0.253 & 0.206 & \multicolumn{1}{c|}{0.198} & 9.95         & 0.9        & 0.25       \\
		\multicolumn{1}{l|}{500}                & -0.032 & 0.006  & \multicolumn{1}{c|}{0}      & 0.201 & 0.153 & \multicolumn{1}{c|}{0}     & 12.20        & 0.1        & 0.05       \\ \hline
		\multicolumn{10}{c}{$\alpha_1 = 1, \alpha_2 = 0.75$}                                                                                                                          \\ \hline
		\multicolumn{1}{c|}{100}                & -0.325 & -0.313 & \multicolumn{1}{c|}{-0.310} & 0.488 & 0.419 & \multicolumn{1}{c|}{0.420} & 4.75         & 0.1        & 0          \\
		\multicolumn{1}{l|}{300}                & 0.028  & 0.063  & \multicolumn{1}{c|}{0.065}  & 0.253 & 0.157 & \multicolumn{1}{c|}{0.159} & 9.95         & 0.55       & 0.5        \\
		\multicolumn{1}{l|}{500}                & -0.082 & -0.037 & \multicolumn{1}{c|}{-0.039} & 0.175 & 0.114 & \multicolumn{1}{c|}{0.112} & 19.25        & 0.25       & 0.3        \\ \hline
		\multicolumn{10}{c}{$\alpha_1 = 1, \alpha_2 = 0.8$}                                                                                                                           \\ \hline
		\multicolumn{1}{l|}{100}                & -0.393 & -0.361 & \multicolumn{1}{c|}{-0.368} & 0.477 & 0.418 & \multicolumn{1}{c|}{0.421} & 85.45        & 85.2       & 86.4       \\
		\multicolumn{1}{l|}{300}                & 0.029  & -0.099 & \multicolumn{1}{c|}{-0.100} & 0.192 & 0.145 & \multicolumn{1}{c|}{0.145} & 12.2         & 0.65       & 0.5        \\
		\multicolumn{1}{l|}{500}                & -0.037 & -0.016 & \multicolumn{1}{c|}{0.016}  & 0.129 & 0.074 & \multicolumn{1}{c|}{0.074} & 27.8         & 0.25       & 1.2        \\ \hline
		\multicolumn{10}{c}{$\alpha_1 = 1, \alpha_2 = 0.85$}                                                                                                                          \\ \hline
		\multicolumn{1}{l|}{100}                & -0.027 & 0.008  & \multicolumn{1}{c|}{0.007}  & 0.234 & 0.160 & \multicolumn{1}{c|}{0.155} & 9.3          & 0.9        & 0.65       \\
		\multicolumn{1}{l|}{300}                & -0.147 & -0.031 & \multicolumn{1}{c|}{-0.037} & 0.219 & 0.079 & \multicolumn{1}{c|}{0.077} & 16.75        & 0.3        & 0.2        \\
		\multicolumn{1}{l|}{500}                & -0.088 & -0.039 & \multicolumn{1}{c|}{-0.039} & 0.136 & 0.063 & \multicolumn{1}{c|}{0.062} & 30.6         & 0.15       & 0          \\ \hline
		\multicolumn{10}{c}{$\alpha_1 = 1, \alpha_2 = 0.9$}                                                                                                                           \\ \hline
		\multicolumn{1}{l|}{100}                & -0.033 & 0.003  & \multicolumn{1}{c|}{0.002}  & 0.173 & 0.111 & \multicolumn{1}{c|}{0.110} & 9.4          & 0.6        & 0.55       \\
		\multicolumn{1}{l|}{300}                & -0.087 & -0.040 & \multicolumn{1}{c|}{-0.041} & 0.131 & 0.061 & \multicolumn{1}{c|}{0.061} & 27.8         & 0.1        & 0.05       \\
		\multicolumn{1}{l|}{500}                & -0.070 & -0.017 & \multicolumn{1}{c|}{-0.018} & 0.111 & 0.037 & \multicolumn{1}{c|}{0.037} & 41.15        & 0.6        & 0.35       \\ \hline
		\multicolumn{10}{c}{$\alpha_1 = 1, \alpha_2 = 0.95$}                                                                                                                          \\ \hline
		\multicolumn{1}{l|}{100}                & -0.134 & -0.101 & \multicolumn{1}{c|}{-0.104} & 0.185 & 0.122 & \multicolumn{1}{c|}{0.122} & 10.15        & 0.1        & 0.15       \\
		\multicolumn{1}{l|}{300}                & -0.083 & -0.034 & \multicolumn{1}{c|}{-0.034} & 0.118 & 0.043 & \multicolumn{1}{c|}{0.044} & 39.35        & 0.6        & 0.6        \\
		\multicolumn{1}{l|}{500}                & -0.062 & -0.013 & \multicolumn{1}{c|}{-0.012} & 0.937 & 0.022 & \multicolumn{1}{c|}{0.023} & 51.8         & 1.25       & 2.0        \\ \hline
		\multicolumn{10}{c}{$\alpha_1=1, \alpha_2 = 1$}                                                                                                                               \\ \hline
		\multicolumn{1}{l|}{100}                & 0      & 0      & \multicolumn{1}{c|}{0}      & 0     & 0     & \multicolumn{1}{c|}{0}     & -            & -          & -          \\
		\multicolumn{1}{l|}{300}                & 0      & 0      & \multicolumn{1}{c|}{0}      & 0     & 0     & \multicolumn{1}{c|}{0}     & -            & -          & -          \\
		\multicolumn{1}{l|}{500}                & 0      & 0      & \multicolumn{1}{c|}{0}      & 0     & 0     & \multicolumn{1}{c|}{0}     & -            & -          & -          \\ \hline\hline
	\end{tabular}
\bigskip \\
{\bf Notes:}
This table shows the result of  experiment 3.
Factors and errors are generate from standard normal distribution.
Between two factors, we assume they have correlation $\rho_{12} = 0.3$
Factor loadings come form uniform distribution $IIDU(\mu_{\beta} - 0.2, \mu_{\beta}+0.2)$, and $\mu_{\beta}$ is set to 0.71.
We keep  $[n^{\alpha_{j}}]$ amount of loadings and assign the rest as zero.
For each different time-unit combinations, we replicate 2000 times.
For the size of the test, we use a two-tail test, under the hypothesis of $H_0, \hat{\alpha}_j = \alpha_j\; j=1,2$.
Cause under the scenarios of $\alpha = 1$, the size of the test will collapse, therefore the table does not report the sizes when $\alpha_1 = \alpha_2 = 1$
\end{table}


\begin{table}[]
	\caption{Simulation result for double factors setting (strong correlation)}\label{table:exp4}
	\centering
	\begin{tabular}{lccccccccc}
		\hline
\hline
\multicolumn{1}{l|}{}    & \multicolumn{9}{c}{Double Factors with correlation $\rho_{12} = 0.7$}                                                                                                  \\ \hline
\multicolumn{1}{l|}{}    & \multicolumn{3}{c|}{Bias $\times$ 100}        & \multicolumn{3}{c|}{RMSE $\times$ 100}     & \multicolumn{3}{c}{Size $\times$ 100} \\ \hline
\multicolumn{10}{c}{$\alpha_1 = 0.7$}                                                                                                                         \\ \hline
\multicolumn{1}{l|}{n\textbackslash T}  & 120    & 240    & \multicolumn{1}{c|}{360}    & 120   & 240   & \multicolumn{1}{c|}{360}   & 120         & 240        & 360        \\ \hline
\multicolumn{1}{l|}{100} & -0.659 & 0.069  & \multicolumn{1}{c|}{0.059}  & 1.193 & 0.406 & \multicolumn{1}{c|}{0.372} & 46.95       & 2.65       & 1.25       \\
\multicolumn{1}{l|}{300} & -0.704 & 0.048  & \multicolumn{1}{c|}{0.055}  & 0.983 & 0.223 & \multicolumn{1}{c|}{0.203} & 75.55       & 3.7        & 0.9        \\
\multicolumn{1}{l|}{500} & -0.841 & -0.011 & \multicolumn{1}{c|}{-0.004} & 1.056 & 0.161 & \multicolumn{1}{c|}{0.144} & 86.4        & 5.1        & 0.3        \\ \hline
\multicolumn{10}{c}{$\alpha_1 = 0.75$}                                                                                                                        \\ \hline
\multicolumn{1}{c|}{100} & -1.03  & -0.315 & \multicolumn{1}{c|}{-0.323} & 1.360 & 0.443 & \multicolumn{1}{c|}{0.421} & 53.65       & 1.35       & 0.05       \\
\multicolumn{1}{l|}{300} & -0.724 & 0.049  & \multicolumn{1}{c|}{0.065}  & 0.947 & 0.166 & \multicolumn{1}{c|}{0.159} & 84.75       & 4.85       & 0.6        \\
\multicolumn{1}{l|}{500} & -0.877 & -0.055 & \multicolumn{1}{c|}{-0.04}  & 1.036 & 0.130 & \multicolumn{1}{c|}{0.111} & 93.5        & 6.5        & 0.15       \\ \hline
\multicolumn{10}{c}{$\alpha_1 = 0.8$}                                                                                                                         \\ \hline
\multicolumn{1}{l|}{100} & -1.099 & -0.374 & \multicolumn{1}{c|}{-0.362} & 1.36  & 0.436 & \multicolumn{1}{c|}{0.421} & 94.6        & 86.05      & 85.75      \\
\multicolumn{1}{l|}{300} & -0.9   & -0.114 & \multicolumn{1}{c|}{-0.103} & 1.071 & 0.165 & \multicolumn{1}{c|}{0.144} & 91.95       & 5.9        & 0.35       \\
\multicolumn{1}{l|}{500} & -0.822 & 0.005  & \multicolumn{1}{c|}{0.015}  & 0.968 & 0.086 & \multicolumn{1}{c|}{0.072} & 97.3        & 9.2        & 0.8        \\ \hline
\multicolumn{10}{c}{$\alpha_1 = 0.85$}                                                                                                                        \\ \hline
\multicolumn{1}{l|}{100} & -0.722 & -0.001 & \multicolumn{1}{c|}{0.009}  & 1.006 & 0.175 & \multicolumn{1}{c|}{0.157} & 71.75       & 3.00       & 0.65       \\
\multicolumn{1}{l|}{300} & -0.834 & -0.05  & \multicolumn{1}{c|}{-0.036} & 0.989 & 0.101 & \multicolumn{1}{c|}{0.078} & 95.25       & 8.65       & 0.3        \\
\multicolumn{1}{l|}{500} & -0.883 & -0.057 & \multicolumn{1}{c|}{-0.039} & 1.013 & 0.084 & \multicolumn{1}{c|}{0.062} & 98.95       & 13.3       & 0.15       \\ \hline
\multicolumn{10}{c}{$\alpha_1 = 0.9$}                                                                                                                         \\ \hline
\multicolumn{1}{l|}{100} & -0.723 & -0.004 & \multicolumn{1}{c|}{0.001}  & 0.972 & 0.125 & \multicolumn{1}{c|}{0.107} & 77.35       & 2.8        & 0.5        \\
\multicolumn{1}{l|}{300} & -0.872 & -0.055 & \multicolumn{1}{c|}{-0.04}  & 1.011 & 0.084 & \multicolumn{1}{c|}{0.062} & 98.2        & 11.1       & 0.25       \\
\multicolumn{1}{l|}{500} & -0.851 & -0.033 & \multicolumn{1}{c|}{-0.018} & 0.967 & 0.06  & \multicolumn{1}{c|}{0.037} & 99.75       & 17.7       & 0.75       \\ \hline
\multicolumn{10}{c}{$\alpha_1 = 0.95$}                                                                                                                        \\ \hline
\multicolumn{1}{l|}{100} & -0.879 & -0.116 & \multicolumn{1}{c|}{-0.103} & 1.083 & 0.143 & \multicolumn{1}{c|}{0.122} & 86.1        & 3.85       & 0.2        \\
\multicolumn{1}{l|}{300} & -0.853 & -0.049 & \multicolumn{1}{c|}{-0.035} & 0.977 & 0.066 & \multicolumn{1}{c|}{0.044} & 99.55       & 14.65      & 1.2        \\
\multicolumn{1}{l|}{500} & -0.875 & -0.029 & \multicolumn{1}{c|}{-0.014} & 0.987 & 0.046 & \multicolumn{1}{c|}{0.022} & 99.85       & 26.65      & 1.55       \\ \hline
\multicolumn{10}{c}{$\alpha_1=1$}                                                                                                                             \\ \hline
\multicolumn{1}{l|}{100} & -0.76  & -0.012 & \multicolumn{1}{c|}{0}      & 0.956 & 0054  & \multicolumn{1}{c|}{0.009} & -           & -          & -          \\
\multicolumn{1}{l|}{300} & -0.811 & -0.015 & \multicolumn{1}{c|}{0}      & 0.945 & 0.037 & \multicolumn{1}{c|}{0.004} & -           & -          & -          \\
\multicolumn{1}{l|}{500} & -0.848 & -0.017 & \multicolumn{1}{c|}{0}      & 0.96  & 0.033 & \multicolumn{1}{c|}{0.003} & -           & -          & -          \\ \hline
\hline
	\end{tabular}
	\bigskip \\
	{\bf Notes:}
	This table shows the result of  experiment 4.
	Factors and errors are generate from standard normal distribution.
	Between two factors, we assume they have correlation $\rho_{12} = 0.7$
	Factor loadings come form uniform distribution $IIDU(\mu_{\beta} - 0.2, \mu_{\beta}+0.2)$, and $\mu_{\beta}$ is set to 0.71.
	We keep  $[n^{\alpha_{j}}]$ amount of loadings and assign the rest as zero.
	For each different time-unit combinations, we replicate 2000 times.
	For the size of the test, we use a two-tail test, under the hypothesis of $H_0, \hat{\alpha}_j = \alpha_j\; j=1,2$.
	Cause under the scenarios of $\alpha = 1$, the size of the test will collapse, therefore the table does not report the sizes when $\alpha_1 = \alpha_2 = 1$
\end{table}


%\end{document}
\end{document}