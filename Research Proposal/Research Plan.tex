\documentclass[12pt]{article}
\usepackage{amsmath}
\usepackage{mathptmx}
\usepackage{setspace}
\usepackage{amssymb}
\usepackage{float}
\usepackage{graphicx}
\usepackage{booktabs}
\usepackage{subfigure}
\usepackage[margin=2.5cm]{geometry}
\usepackage[title]{appendix}
\usepackage{bm}
\usepackage{tcolorbox}
\usepackage{apacite}
\onehalfspacing
\usepackage{lineno}
\linenumbers
 \usepackage{diagbox}


\usepackage{fontspec}
\setmainfont{Times New Roman}
\addtolength{\jot}{0.5em}

\title{Factor Selection and Factor Strength\\
\large Base on the U.S. Stock Market data\\
Research Plan }

\author{Zhiyuan Jiang\\I.D:28710967}
\date{\today}

\begin{document}
	\maketitle	
	\newpage
	\tableofcontents
	\newpage
	\section{Introduction and Motivation} \label{Intro}
Capital Asset Pricing Model (CAPM), created by \citeA{Sharpe1964} and \citeA{Lintner1965} is one of the most famous and wildly used model on explaining the relationship between financial asset's risk and return.
Many scholars are trying to find and add new variables to the CAPM model to enhance its ability to capturing the dynamics  between asset return and return volatility. 
One of the most famous examples of new factors are the size factor (SMB) and book-to-market factor(HML) found by \citeA{Fama1992} 

A recent study by \citeA{Harvey2019}, revealed the fact that in recent years new factors and paper illustrate how those factors are helping explain the relationship between risk and return were in abundance.
In their 2015 papers, \citeauthor{Harvey2015}  coined the term "factor zoo" which precisely described the fact that the field of financial economics has too many factors. 
And some, if not most, of them, can provide seemly significant results purely because of luck.
\citeA{Harvey2017b} provides some insight and explanation about this phenomena from the prospect of critical value and the journal's preference. 

Inside the factor zoo, it is worth noticing that the ability of pricing risk is various among factors. 
Some researches (see \citeNP{Kleibergen2009}, and \citeNP{Gospodinov2017}) argues that for a weak factor which has small coefficient or even no coefficient, the statistic inference for a model containing such factors will be unreliable.
\citeA{Kan1999} warned that when including a factor has no correlation with the asset return in a two-pass method of testing the pricing model, the model will falsely identified that useless model as significant more usually than it should. 
 limited explaining power before contains it into the CAPM becomes crucial.
In their recent paper, \citeA{Bailey2020} introduce a new framework to measure a factor's ability or strength of pricing the risk of  assets.
We will briefly review their idea in the following section \ref{strength}

In this project, I adopt the framework provided by \citeA{Bailey2020} to exam the strength of factors from \citeauthor{Harvey2019}'s factor zoo, base on the securities return of Standard \& Poors (S\&P) 500 index companies. 
And then, by applying some machine learning techniques and algorithms,  I want to select strong factors which can most accurately pricing the asset's risk.



	\section{Methodology}\label{Method}
	In this section, I will briefly introduce the intuition behind \citeauthor{Bailey2020}'s factor strength, and then provides a framework about how the idea of factor strength implied on a Monte Carlo simulation.
	\subsection{Factor Strength: Intuition}\label{strength}
	Generally speaking, the strength of factor represents how pervasiveness the factor is.
	The stronger a factor is, the more stock it can significantly influence at a point of time. 
	Assume a single factor CAPM model:
	\[  y_{it} = \beta_i + \theta_{i}x_t + \epsilon_{it}  \tag{1}\label{simpleCAPM} \]
In  this model, we collect n different assets from T different time point.
The left hand  side $y_{it}$ is the excess return for asset $i, \{ i = 1, 2,\cdots n\}$ at time $t, \{t = 1, 2,\cdots, T\}$. $\theta_{i}$ here represents the factor loading of  factor $x_t$ at time $t$. 
After running the OLS regression between the asset's return and factor, we can collect a bunch of different loading $\theta_{i}$, we can state that:
\begin{align*}
|\theta_{i}| &> c_p(n)\quad i = 1, 2,  \dots, [n^{\alpha}]\\
|\theta_{i}| &= 0 \quad i = [n^{\alpha}] + 1, [n^{\alpha}] +2 ,\dots, n
\end{align*}
Here $c_p(n)$ denotes a multi-test corrected critical value for t-test. 
Those two equation clearly demonstrate how factor strength $\alpha$ is defined intuitively.
For the first $[n^{\alpha}]$ term, where $[\cdot]$ will take integer part,  the factor loading will statistically significant different from 0, and the rest loadings will not have any significant influence on pricing asset risk.
For a factor has strength $\alpha = 1$, that factor will be significant  for every assets at the same time.
The more observation the factor can significantly influence, the stronger the factor is, and vice versa.


	\subsection{Monte Carlo Design}\label{MC}
Before start using the real data, we want to study the property of $\alpha$ by running Monte Carlo simulation and in this section, I will introduce the basic simulation design.

Consider the following model with stochastic error:

\[  r_{it} = f_1(\bar{r_{t}} - r_f) + f_2( \theta_{i}x_{t}) +\epsilon_{it} \tag{2}\label{baselineequation} \]

In this Monte Carlo simulation, we consider a dataset has $i = 1, 2,\dots, n$ different assets, with $t= 1, 2,\dots, T$ different observations. 
$j = 1, 2, \dots, k$ different factors and one market factors are included in the simulation. 


$f_1(\cdot)$ and $f_2(\cdot)$ are two different functions represent the unknown mechanism of market factor and other factors in pricing asset risk.
$(\bar{r_{t}}- r_f) $ is the market return, calculated from market or index return $\bar{r_{t}}$ minus risk free return $r_f$. 
$r_{it}$ is the stock return, $\theta_{jt}$ denotes factors other than market factors and $\beta_{ij}$is the corresponding factor loading. 
$\epsilon_{it}$ is random error with structure can be defined in different designs.
Notice that the $\beta_{ij}$ will be influenced by each factor's strength $\alpha_j$, where we have $\alpha$ as defined in section $\ref{strength}$. 
And for each factor, we assume they follow a multinomial distribution with mean zero and a $k\times k$ variance-covariance matrix $\Sigma$. 
The diagonal of matrix $\Sigma$ indicates the variance of each factor, and the rest represent the correlation among all $k$ factors.
In this model, we can control several parts to investigates different scenarios of the simulation:
%\begin{enumerate}
%	\item The function $f_1(\cdot)$and $f_2(\cdot)$.
%	\item The strength $\alpha$ of each factor.
%	\item The correlation among each factors.
%	\item The stochastic error term $\epsilon_{it}$
%\end{enumerate}

	\section{Preliminary Analyse Result }
	In this section, I will introduce the baseline design setting of the Monte Carlo Simulation and provides a preliminary result of the simulation.
	 \subsection{Baseline Design}\label{base}
	Follow the model (\ref{baselineequation}), we assume both $f_1(a)$ and $f_2(a)$ are linear function:
	 \begin{align*}
	 f_1(a ) &= c_{i} +\beta_{} a\\
	 f_2(a) &=a
	 \end{align*}
	 Therefore, the model with single factor can be write as:
	 \[   r_{it} = c_i +   \theta_{i}x_{t} +\epsilon_{it}   \]

The constant$c_i$ is generated from a uniform distribution $\mathnormal{U}[-0.5, 0.5]$.
$\theta_{i}$ is the factor loading, and $x_t$ is factor with strength $\alpha_{x}$.
	 To generate factors loading, we employed a two steps strategy.
	 First we generate a whole factor loadings vector $\mathbf{\theta_i} = (\theta_{i1}, \theta_{i2} \cdots, \theta_{in})$,
	 All elements of the vector follows $\mathnormal{IIDU}(\mu_{\theta} -0.2, \mu_{\theta} + 0.2)$. 
	 The $\mu_{\theta}$ has been equalled to 0.71 to ensure all values apart from zero. 
	 After generating the vector, we randomly selected $[n^{\alpha_{x}}]$ elements from $\mathbf{\theta_i}$ to keep their value and set the other elements to zero. 
	 This step ensures the loading reflects the strength of each factor. 
	 For the stochastic error term, in this baseline design, we assume it follows a Standard Gaussian distribution, but we can easily extend it into a more complex form.
	 
	 Follow the same idea, we also construct a two factor model:
	 \[   r_{it} = c_i + \lambda x_m  + \theta_{i}x_{t} +\epsilon_{it}   \]
	 Here the $x_m$ is the market factor which assumably  has strength $\alpha_{m} = 1$. 
	 $\mathbf{\lambda}$ is the market factor loading as a vector with all elements different from zero. 
	 
	 For each of the those different models, we consider the $T = \{120, 240, 360\}$, $n =\{100, 300, 500\} $.
	  The market factor will have strength $\alpha_m = 1$ all the time, and the strength of the other factor in two factor model will be $\alpha_{x} = \{0.5, 0.7, 0.9,1\}$. For every setting, we will replicate 500 times independently, all the constant $c_i$ and loading $\theta_i$ will be re-generated for each replication.
	 \subsection{Simulation Result}
	The detailed simulation result table has been showing in the Appendix \ref{simulationtable}
	
To measure the goodness of simulation, we calculate the difference between the estimated factor strength and assigned factor strength as bias.
Base on the bias,, we also calculated the Mean Squared Error (MSE) for each setting.

From the result, we can easily find out that the error converge to zero when the strength $\alpha$ increases. 
When the $\alpha _x= 1$, we obtain the unbiased $\hat{\alpha}_{x}$

\newpage
\bibliographystyle{apacite}
\bibliography{proposal.bib}

\newpage
\appendix

%\documentclass[12pt]{article}
%\usepackage{amsmath}
%\usepackage{mathptmx}
%\usepackage{setspace}
%\usepackage{amssymb}
%\usepackage{float}
%\usepackage{graphicx}
%\usepackage{booktabs}
%\usepackage{subfigure}
%\usepackage[margin=2.5cm]{geometry}
%\usepackage[title]{appendix}
%\usepackage{bm}
%\usepackage{tcolorbox}
%\usepackage{apacite}
%\onehalfspacing

%

%\begin{document}

	
	\section{Simulation Result}\label{simulationtable}
\begin{table}[!hbt]
		\caption{Simulation result for single factor setting}\label{table:exp1}
	\centering

	\begin{tabular}{lccccccccc}
		\hline
		\hline
		\multicolumn{1}{l|}{}                   & \multicolumn{9}{c}{Single Factor}                                                               \\ \hline
		\multicolumn{1}{l|}{}                   & \multicolumn{3}{c|}{Bias $\times$ 100}        & \multicolumn{3}{c|}{RMSE $\times$ 100}     & \multicolumn{3}{c}{Size $\times$ 100} \\ \hline
		\multicolumn{10}{c}{$\alpha_1 = 0.7$}                                                                                                                            \\ \hline
		\multicolumn{1}{l|}{n\textbackslash{}T} & 120    & 240    & \multicolumn{1}{c|}{360}    & 120   & 240   & \multicolumn{1}{c|}{360}   & 120         & 240         & 360        \\ \hline
		\multicolumn{1}{l|}{100}                & 0.256  & 0.265  & \multicolumn{1}{c|}{0.227}  & 0.612 & 0.623 & \multicolumn{1}{c|}{0.560} & 7.85        & 7.7         & 5.55       \\
		\multicolumn{1}{l|}{300}                & 0.185  & 0.184  & \multicolumn{1}{c|}{0.184}  & 0.363 & 0.338 & \multicolumn{1}{c|}{0.335} & 8.9         & 4.45        & 4.5        \\
		\multicolumn{1}{l|}{500}                & 0.107  & 0.124  & \multicolumn{1}{c|}{0.109}  & 0.259 & 0.248 & \multicolumn{1}{c|}{0.234} & 6.9         & 2.5         & 1.6        \\ \hline
		\multicolumn{10}{c}{$\alpha_1 = 0.75$}                                                                                                                          \\ \hline
		\multicolumn{1}{c|}{100}                & -0.178 & -0.159 & \multicolumn{1}{c|}{-0.168} & 0.490 & 0.465 & \multicolumn{1}{c|}{0.450} & 2.5         & 0.85        & 0.4        \\
		\multicolumn{1}{l|}{300}                & 0.154  & 0.156  & \multicolumn{1}{c|}{0.143}  & 0.281 & 0.258 & \multicolumn{1}{c|}{0.234} & 9.4         & 3.7         & 3.35       \\
		\multicolumn{1}{l|}{500}                & 0.024  & 0.033  & \multicolumn{1}{c|}{0.263}  & 0.171 & 0.155 & \multicolumn{1}{c|}{0.148} & 7.8         & 2           & 1.25       \\ \hline
		\multicolumn{10}{c}{$\alpha_1 = 0.8$}                                                                                                                           \\ \hline
		\multicolumn{1}{l|}{100}                & -0.270 & -0.265 & \multicolumn{1}{c|}{-0.258} & 0.434 & 0.409 & \multicolumn{1}{c|}{0.411} & 71.4        & 72.05       & 71.45      \\
		\multicolumn{1}{l|}{300}                & -0.052 & -0.044 & \multicolumn{1}{c|}{-0.043} & 0.183 & 0.149 & \multicolumn{1}{c|}{0.150} & 10.15       & 2.45        & 2.9        \\
		\multicolumn{1}{l|}{500}                & 0.045  & 0.068  & \multicolumn{1}{c|}{0.067}  & 0.136 & 0.126 & \multicolumn{1}{c|}{0.121} & 16.6        & 6.4         & 5.9        \\ \hline
		\multicolumn{10}{c}{$\alpha_1 = 0.85$}                                                                                                                          \\ \hline
		\multicolumn{1}{l|}{100}                & 0.053  & 0.062  & \multicolumn{1}{c|}{0.058}  & 0.253 & 0.228 & \multicolumn{1}{c|}{0.221} & 6.05        & 2.95        & 2.5        \\
		\multicolumn{1}{l|}{300}                & -0.012 & 0.009  & \multicolumn{1}{c|}{-0.001} & 0.124 & 0.104 & \multicolumn{1}{c|}{0.095} & 10.55       & 1.8         & 1.15       \\
		\multicolumn{1}{l|}{500}                & -0.026 & -0.007 & \multicolumn{1}{c|}{-0.011} & 0.096 & 0.073 & \multicolumn{1}{c|}{0.069} & 13.25       & 0.9         & 0.7        \\ \hline
		\multicolumn{10}{c}{$\alpha_1 = 0.9$}                                                                                                                           \\ \hline
		\multicolumn{1}{l|}{100}                & 0.025  & 0.038  & \multicolumn{1}{c|}{0.360}  & 0.191 & 0.163 & \multicolumn{1}{c|}{0.157} & 6.85        & 2           & 1.65       \\
		\multicolumn{1}{l|}{300}                & -0.034 & -0.018 & \multicolumn{1}{c|}{-0.020} & 0.099 & 0.069 & \multicolumn{1}{c|}{0.068} & 13.2        & 0.8         & 0.9        \\
		\multicolumn{1}{l|}{500}                & -0.025 & -0.001 & \multicolumn{1}{c|}{-0.001} & 0.072 & 0.044 & \multicolumn{1}{c|}{0.044} & 22.3        & 1.95        & 1.8        \\ \hline
		\multicolumn{10}{c}{$\alpha_1 = 0.95$}                                                                                                                          \\ \hline
		\multicolumn{1}{l|}{100}                & -0.099 & -0.088 & \multicolumn{1}{c|}{-0.090} & 0.156 & 0.125 & \multicolumn{1}{c|}{0.126} & 5.6         & 0.3         & 0.55       \\
		\multicolumn{1}{l|}{300}                & -0.046 & -0.025 & \multicolumn{1}{c|}{-0.026} & 0.083 & 0.045 & \multicolumn{1}{c|}{0.045} & 22.5        & 2.2         & 2.25       \\
		\multicolumn{1}{l|}{500}                & -0.030 & -0.006 & \multicolumn{1}{c|}{-0.006} & 0.061 & 0.026 & \multicolumn{1}{c|}{0.025} & 33.1        & 4.4         & 3.8        \\ \hline
		\multicolumn{10}{c}{$\alpha_1=1$}                                                                                                                               \\ \hline
		\multicolumn{1}{l|}{100}                & 0      & 0      & \multicolumn{1}{c|}{0}      & 0     & 0     & \multicolumn{1}{c|}{0}     & -           & -           & -          \\
		\multicolumn{1}{l|}{300}                & 0      & 0      & \multicolumn{1}{c|}{0}      & 0     & 0     & \multicolumn{1}{c|}{0}     & -           & -           & -          \\
		\multicolumn{1}{l|}{500}                & 0      & 0      & \multicolumn{1}{c|}{0}      & 0     & 0     & \multicolumn{1}{c|}{0}     & -           & -           & -          \\ \hline
\hline
\end{tabular}
\bigskip\\
{\bf Notes:}
This table shows the result of  experiment 1.
Factors and error are generate from standard normal distribution.
Factor loadings come form uniform distribution $IIDU(\mu_{\beta} - 0.2, \mu_{\beta}+0.2)$, and $\mu_{\beta} = 0.71$.
We keep  $[n^{\alpha_{j}}]$ amount of loadings and assign the rest as zero.
For each different time-unit combinations, we replicate 2000 times.
For the size of the test, we use a two-tail test, under the hypothesis of $H_0, \hat{\alpha}_j = \alpha_j\; j=1,2$.
Cause under the scenarios of $\alpha = 1$, the size of the test will collapse, therefore the table does not report the sizes for $\alpha_1 = 1$.
\end{table}


\begin{table}[!hbt]
		\caption{Simulation result for double factors setting (no correlation)}\label{table:exp2}
	\centering
	\begin{tabular}{lccccccccc}
		\hline
		\hline
\multicolumn{1}{l|}{}                   & \multicolumn{9}{c}{Double Factor with correlation $\rho_{12} = 0$}                                                                \\ \hline
\multicolumn{1}{l|}{}                   & \multicolumn{3}{c|}{Bias $\times$ 100}        & \multicolumn{3}{c|}{RMSE $\times$ 100}      & \multicolumn{3}{c}{Size $\times$ 100} \\ \hline
\multicolumn{10}{c}{$\alpha_1 = 1, \alpha_2  = 0.7$}                                                                                                                           \\ \hline
\multicolumn{1}{l|}{n\textbackslash{}T} & 120    & 240    & \multicolumn{1}{c|}{360}    & 120    & 240   & \multicolumn{1}{c|}{360}   & 120         & 240        & 360         \\ \hline
\multicolumn{1}{l|}{100}                & 0.567  & 0.737  & \multicolumn{1}{c|}{0.628}  & 4.062  & 3.819 & \multicolumn{1}{c|}{3.799} & 2.95        & 1.45       & 1.85        \\
\multicolumn{1}{l|}{300}                & 0.512  & 0.611  & \multicolumn{1}{c|}{0.518}  & 2.398  & 2.103 & \multicolumn{1}{c|}{1.979} & 6.25        & 0.55       & 0.5         \\
\multicolumn{1}{l|}{500}                & -0.149 & 0.08   & \multicolumn{1}{c|}{-0.019} & 1.796  & 1.498 & \multicolumn{1}{c|}{1.443} & 8           & 0.2        & 0.1         \\ \hline
\multicolumn{10}{c}{$\alpha_1 = 1, \alpha_2 =0.75$}                                                                                                                              \\ \hline
\multicolumn{1}{c|}{100}                & -3.051 & -3.02  & \multicolumn{1}{c|}{-3.092} & 4.582  & 4.245 & \multicolumn{1}{c|}{4.248} & 2.45        & 0.1        & 0.10        \\
\multicolumn{1}{l|}{300}                & 0.491  & -1.035 & \multicolumn{1}{c|}{0.640}  & 1.843  & 1.460 & \multicolumn{1}{c|}{1.576} & 7.6         & 0.8        & 0.55        \\
\multicolumn{1}{l|}{500}                & -0.611 & -0.372 & \multicolumn{1}{c|}{-0.393} & 1.520  & 1.136 & \multicolumn{1}{c|}{1.125} & 11.35       & 0.15       & 0.1         \\ \hline
\multicolumn{10}{c}{$\alpha _1= 1, \alpha_2 = 0.8$}                                                                                                                              \\ \hline
\multicolumn{1}{l|}{100}                & -3.752 & -3.630 & \multicolumn{1}{c|}{-3.581} & 4.557  & 4.213 & \multicolumn{1}{c|}{4.210} & 84.65       & 85.9       & 85.25       \\
\multicolumn{1}{l|}{300}                & -1.218 & -0.331 & \multicolumn{1}{c|}{-1.021} & 1.812  & 0.792 & \multicolumn{1}{c|}{1.438} & 9.35        & 0.2        & 0.3         \\
\multicolumn{1}{l|}{500}                & -0.022 & 0.192  & \multicolumn{1}{c|}{0.147}  & 1.047  & 0.782 & \multicolumn{1}{c|}{0.742} & 15.35       & 1.1        & 1.1         \\ \hline
\multicolumn{10}{c}{$\alpha_1 = 1, \alpha_2 = 0.85$}                                                                                                                             \\ \hline
\multicolumn{1}{l|}{100}                & -0.075 & 0.127  & \multicolumn{1}{c|}{0.088}  & 1.996  & 1.697 & \multicolumn{1}{c|}{1.606} & 5.4         & 1.15       & 0.95        \\
\multicolumn{1}{l|}{300}                & -0.531 & -0.406 & \multicolumn{1}{c|}{-0.351} & 1.097  & 0.613 & \multicolumn{1}{c|}{0.777} & 10.8        & 0.15       & 0.2         \\
\multicolumn{1}{l|}{500}                & -0.647 & -0.391 & \multicolumn{1}{c|}{-0.391} & 1.020  & 0.643 & \multicolumn{1}{c|}{0.630} & 19.1        & 0.15       & 0           \\ \hline
\multicolumn{10}{c}{$\alpha_1 = 1, \alpha_2 = 0.9$}                                                                                                                              \\ \hline
\multicolumn{1}{l|}{100}                & -0.128 & 0.043  & \multicolumn{1}{c|}{0.025}  & 1.428  & 1.143 & \multicolumn{1}{c|}{1.118} & 4.9         & 0.65       & 0.7         \\
\multicolumn{1}{l|}{300}                & -0.651 & -0.334 & \multicolumn{1}{c|}{-0.394} & 1.002  & 0.435 & \multicolumn{1}{c|}{0.617} & 17.1        & 0.6        & 0.2         \\
\multicolumn{1}{l|}{500}                & -0.434 & -0.168 & \multicolumn{1}{c|}{-0.171} & 0.7435 & 0.367 & \multicolumn{1}{c|}{0.368} & 25.2        & 0.4        & 0.3         \\ \hline
\multicolumn{10}{c}{$\alpha_1 = 1, \alpha_2 = 0.95$}                                                                                                                             \\ \hline
\multicolumn{1}{l|}{100}                & -1.218 & -1.043 & \multicolumn{1}{c|}{-1.036} & 1.603  & 1.222 & \multicolumn{1}{c|}{1.212} & 6.65        & 0.25       & 0.05        \\
\multicolumn{1}{l|}{300}                & -0.611 & -0.344 & \multicolumn{1}{c|}{-0.356} & 0.881  & 0.435 & \multicolumn{1}{c|}{0.434} & 23.35       & 0.6        & 0.45        \\
\multicolumn{1}{l|}{500}                & -0.415 & -0.123 & \multicolumn{1}{c|}{-0.134} & 0.661  & 0.220 & \multicolumn{1}{c|}{0.216} & 36.75       & 1.35       & 1.1         \\ \hline
\multicolumn{10}{c}{$\alpha_1=1, \alpha_2 = 1$}                                                                                                                                  \\ \hline
\multicolumn{1}{l|}{100}                & 0      & 0      & \multicolumn{1}{c|}{0}      & 0      & 0     & \multicolumn{1}{c|}{0}     & -           & -          & -           \\
\multicolumn{1}{l|}{300}                & 0      & 0      & \multicolumn{1}{c|}{0}      & 0      & 0     & \multicolumn{1}{c|}{0}     & -           & -          & -           \\
\multicolumn{1}{l|}{500}                & 0      & 0      & \multicolumn{1}{c|}{0}      & 0      & 0     & \multicolumn{1}{c|}{0}     & -           & -          & -           \\ \hline\hline
	\end{tabular}
\bigskip\\
{\bf Notes:}
This table shows the result of  experiment 2.
Factors and errors are generate from standard normal distribution.
Between two factors, we assume they have no correlation.
Factor loadings come form uniform distribution $IIDU(\mu_{\beta} - 0.2, \mu_{\beta}+0.2)$, and $\mu_{\beta}$ is set to 0.71.
We keep  $[n^{\alpha_{j}}]$ amount of loadings and assign the rest as zero.
For each different time-unit combinations, we replicate 2000 times.
For the size of the test, we use a two-tail test, under the hypothesis of $H_0, \hat{\alpha}_j = \alpha_j\; j=1,2$.
Cause under the scenarios of $\alpha = 1$, the size of the test will collapse, therefore the table does not report the sizes for $\alpha_1 = \alpha_2 = 1$
\end{table}


	\begin{table}[]
		\caption{Simulation result for double factors setting (weak correlation)}\label{table:exp3}
		\centering
	\begin{tabular}{lccccccccc}
		\hline
		\hline
		\multicolumn{1}{l|}{}                   & \multicolumn{9}{c}{Double Factor with correlation $\rho_{12} = 0.3$}                                                               \\ \hline
		\multicolumn{1}{l|}{}                   & \multicolumn{3}{c|}{Bias $\times$ 100}        & \multicolumn{3}{c|}{RMSE $\times$ 100}     & \multicolumn{3}{c}{Size $\times$ 100} \\ \hline
		\multicolumn{10}{c}{$\alpha_1 = 1, \alpha_2 = 0.7$}                                                                                                                            \\ \hline
		\multicolumn{1}{l|}{n\textbackslash{}T} & 120    & 240    & \multicolumn{1}{c|}{360}    & 120   & 240   & \multicolumn{1}{c|}{360}   & 120          & 240        & 360        \\ \hline
		\multicolumn{1}{l|}{100}                & 0.038  & 0.064  & \multicolumn{1}{c|}{0.072}  & 0.421 & 0.382 & \multicolumn{1}{c|}{0.389} & 4.6          & 1.75       & 1.95       \\
		\multicolumn{1}{l|}{300}                & 0.021  & 0.058  & \multicolumn{1}{c|}{0.056}  & 0.253 & 0.206 & \multicolumn{1}{c|}{0.198} & 9.95         & 0.9        & 0.25       \\
		\multicolumn{1}{l|}{500}                & -0.032 & 0.006  & \multicolumn{1}{c|}{0}      & 0.201 & 0.153 & \multicolumn{1}{c|}{0}     & 12.20        & 0.1        & 0.05       \\ \hline
		\multicolumn{10}{c}{$\alpha_1 = 1, \alpha_2 = 0.75$}                                                                                                                          \\ \hline
		\multicolumn{1}{c|}{100}                & -0.325 & -0.313 & \multicolumn{1}{c|}{-0.310} & 0.488 & 0.419 & \multicolumn{1}{c|}{0.420} & 4.75         & 0.1        & 0          \\
		\multicolumn{1}{l|}{300}                & 0.028  & 0.063  & \multicolumn{1}{c|}{0.065}  & 0.253 & 0.157 & \multicolumn{1}{c|}{0.159} & 9.95         & 0.55       & 0.5        \\
		\multicolumn{1}{l|}{500}                & -0.082 & -0.037 & \multicolumn{1}{c|}{-0.039} & 0.175 & 0.114 & \multicolumn{1}{c|}{0.112} & 19.25        & 0.25       & 0.3        \\ \hline
		\multicolumn{10}{c}{$\alpha_1 = 1, \alpha_2 = 0.8$}                                                                                                                           \\ \hline
		\multicolumn{1}{l|}{100}                & -0.393 & -0.361 & \multicolumn{1}{c|}{-0.368} & 0.477 & 0.418 & \multicolumn{1}{c|}{0.421} & 85.45        & 85.2       & 86.4       \\
		\multicolumn{1}{l|}{300}                & 0.029  & -0.099 & \multicolumn{1}{c|}{-0.100} & 0.192 & 0.145 & \multicolumn{1}{c|}{0.145} & 12.2         & 0.65       & 0.5        \\
		\multicolumn{1}{l|}{500}                & -0.037 & -0.016 & \multicolumn{1}{c|}{0.016}  & 0.129 & 0.074 & \multicolumn{1}{c|}{0.074} & 27.8         & 0.25       & 1.2        \\ \hline
		\multicolumn{10}{c}{$\alpha_1 = 1, \alpha_2 = 0.85$}                                                                                                                          \\ \hline
		\multicolumn{1}{l|}{100}                & -0.027 & 0.008  & \multicolumn{1}{c|}{0.007}  & 0.234 & 0.160 & \multicolumn{1}{c|}{0.155} & 9.3          & 0.9        & 0.65       \\
		\multicolumn{1}{l|}{300}                & -0.147 & -0.031 & \multicolumn{1}{c|}{-0.037} & 0.219 & 0.079 & \multicolumn{1}{c|}{0.077} & 16.75        & 0.3        & 0.2        \\
		\multicolumn{1}{l|}{500}                & -0.088 & -0.039 & \multicolumn{1}{c|}{-0.039} & 0.136 & 0.063 & \multicolumn{1}{c|}{0.062} & 30.6         & 0.15       & 0          \\ \hline
		\multicolumn{10}{c}{$\alpha_1 = 1, \alpha_2 = 0.9$}                                                                                                                           \\ \hline
		\multicolumn{1}{l|}{100}                & -0.033 & 0.003  & \multicolumn{1}{c|}{0.002}  & 0.173 & 0.111 & \multicolumn{1}{c|}{0.110} & 9.4          & 0.6        & 0.55       \\
		\multicolumn{1}{l|}{300}                & -0.087 & -0.040 & \multicolumn{1}{c|}{-0.041} & 0.131 & 0.061 & \multicolumn{1}{c|}{0.061} & 27.8         & 0.1        & 0.05       \\
		\multicolumn{1}{l|}{500}                & -0.070 & -0.017 & \multicolumn{1}{c|}{-0.018} & 0.111 & 0.037 & \multicolumn{1}{c|}{0.037} & 41.15        & 0.6        & 0.35       \\ \hline
		\multicolumn{10}{c}{$\alpha_1 = 1, \alpha_2 = 0.95$}                                                                                                                          \\ \hline
		\multicolumn{1}{l|}{100}                & -0.134 & -0.101 & \multicolumn{1}{c|}{-0.104} & 0.185 & 0.122 & \multicolumn{1}{c|}{0.122} & 10.15        & 0.1        & 0.15       \\
		\multicolumn{1}{l|}{300}                & -0.083 & -0.034 & \multicolumn{1}{c|}{-0.034} & 0.118 & 0.043 & \multicolumn{1}{c|}{0.044} & 39.35        & 0.6        & 0.6        \\
		\multicolumn{1}{l|}{500}                & -0.062 & -0.013 & \multicolumn{1}{c|}{-0.012} & 0.937 & 0.022 & \multicolumn{1}{c|}{0.023} & 51.8         & 1.25       & 2.0        \\ \hline
		\multicolumn{10}{c}{$\alpha_1=1, \alpha_2 = 1$}                                                                                                                               \\ \hline
		\multicolumn{1}{l|}{100}                & 0      & 0      & \multicolumn{1}{c|}{0}      & 0     & 0     & \multicolumn{1}{c|}{0}     & -            & -          & -          \\
		\multicolumn{1}{l|}{300}                & 0      & 0      & \multicolumn{1}{c|}{0}      & 0     & 0     & \multicolumn{1}{c|}{0}     & -            & -          & -          \\
		\multicolumn{1}{l|}{500}                & 0      & 0      & \multicolumn{1}{c|}{0}      & 0     & 0     & \multicolumn{1}{c|}{0}     & -            & -          & -          \\ \hline\hline
	\end{tabular}
\bigskip \\
{\bf Notes:}
This table shows the result of  experiment 2.
Factors and errors are generate from standard normal distribution.
Between two factors, we assume they have correlation $\rho_{12} = 0.3$
Factor loadings come form uniform distribution $IIDU(\mu_{\beta} - 0.2, \mu_{\beta}+0.2)$, and $\mu_{\beta}$ is set to 0.71.
We keep  $[n^{\alpha_{j}}]$ amount of loadings and assign the rest as zero.
For each different time-unit combinations, we replicate 2000 times.
For the size of the test, we use a two-tail test, under the hypothesis of $H_0, \hat{\alpha}_j = \alpha_j\; j=1,2$.
Cause under the scenarios of $\alpha = 1$, the size of the test will collapse, therefore the table does not report the sizes when $\alpha_1 = \alpha_2 = 1$
\end{table}


%\end{document}



\end{document}