\documentclass[12pt]{beamer}
\usepackage{amsmath}
\usepackage[utf8]{inputenc}
\usepackage{apacite}
\usetheme{Singapore}

\usepackage{soul}


	\title{Factor Selection and Factor Strength\\
A study base on U.S. stock Market}
	\subtitle{Research Plan Presentation}

		\date{\today}
		\author[author]{Zhiyuan Jiang\\
			28710967\\[10mm]{\small Supervisors: Dr. Natalia Bailey \\ \hspace{18mm} Dr. David Frazier}}
		
		
\begin{document}
	
\frame[plain]{\titlepage}

%-------------------------------------------------------------------------------------------------------------------------------------------------------------------------------%
%-------------------------------------------------------------------------------------------------------------------------------------------------------------------------------%
\section{Introduction and Motivation}
\begin{frame}{Introduction and Motivation}
Capital Asset Pricing Model has shaped the way people evaluate the risk and return of assets.
  \[   r_{it} = c + \beta_0 x_m + \mathbf{\beta_j}^{\prime}\mathbf{x_j} + \varepsilon_{it}  \tag{1}\label{2CAPM} \]
By adding new factors, we can improve the model's ability of pricing the risk and predict the return.
 Scholars are always trying to find new factors to add into the $\mathbf{x_j}$  part.
   
   Now we have over 500 factors inside our factor zoo, and it is still growing. \cite{Harvey2019}
\end{frame}
\begin{frame}[plain]
	\begin{figure}
\includegraphics[scale = 0.5]{figure/factor_growth.png}
\caption{Factor amount growing through the year. }
	\cite{Harvey2017}
	\end{figure}
\end{frame}

\begin{frame}
\frametitle{Introduction and Motivation}
But some problem exist inside the factor zoo.\\

\begin{itemize}
\item Not all factors are good or useful
\begin{itemize}
\item Predict power drop once published \cite{McLean2016}
\item Can not been replicated \cite{Hou2018}
\item Significant because of luck \cite{Harvey2014}
\end{itemize}
\item Including weak or useless factors is terrible
\begin{itemize}
\item Useless factor in FM first-regression\cite{Fama1973} will yield misleading second regression result \cite{Kan1999}
\item If the loading is small, regression will report incorrect result \citeA{Kleibergen2009}
\end{itemize}
\end{itemize}
\end{frame}


%-------------------------------------------------------------------------------------------------------------------------------------------------------------------------------%
%-------------------------------------------------------------------------------------------------------------------------------------------------------------------------------%
	\section{Related Literature}
	\begin{frame}
\frametitle{Literature}
\begin{itemize}
	\item {\bf Identify factors from the zoo}\\
	\citeA{Harvey2015}, \citeA{McLean2016}, \citeA{Harvey2017}, \citeA{Barillas2018},\citeA{Pukthuanthong2019}
	\item {\bf Using machine learning method}\\
	\citeA{Rapach2013}, \citeA{Feng2019},\citeA{Gu2020}, \citeA{Lettau2020}, \citeA{Freyberger2020}, \citeA{Kozak2020}
	\end{itemize}
	\end{frame}
	
	
	
	
	
%	\begin{frame}
%\frametitle{Identify Useful Factors}
%There are a bunch of literatures and researches aiming identify useful factors.
%\begin{enumerate}
%\item \citeA{Harvey2015} exams over 300 factors, and suggest to arise the p-value threshold to 3 to eliminate some new factors turns to be significant purely by luck.
%\item \citeA{Harvey2017} introduce a bootstrap method 
%\item \citeA{Barillas2018} use Bayes procedure to find useful factor
%\item \citeA{Pukthuanthong2019} defined a protocol to identify priced factor.
%\end{enumerate}
	%\end{frame}

%\begin{frame}
%	\frametitle{Identify Useful Factor}
%A recent studies shows that using Machine Learning methods in asset pricing can bring huge benefits \cite{Gu2020}\\
%\begin{itemize}
%\item Better predict accuracy
%\item Improved efficiency
%\end{itemize}
%\end{frame}

%\begin{frame}
%	\frametitle{Identify Useful Factor}
%Some researchers already started using Machine Learning algorithms, especially Lasso, a factor selection algorithm.
%\begin{enumerate}
%\item \citeA{Rapach2013} using Lasso to find factor for predicting global stock market return.
%\item \citeA{Feng2019} used a improved double-selected method in factor selection
%\item Because Lasso has some limitation, \citeA{Huang2010} implied a group lasso method, trying to overcome such shortcoming.
%\end{enumerate}
%\end{frame}

%-------------------------------------------------------------------------------------------------------------------------------------------------------------------------------%
%-------------------------------------------------------------------------------------------------------------------------------------------------------------------------------%


	\section{Methodology}

\begin{frame}
\frametitle{Methodology}
This project facing two challenges:
\begin{enumerate}
\item	 A Big Zoo:  Too many candidates\\
Curse of Dimensionality 
\item 	Correlation among factors \\
Traditional factor selection algorithm can not handle this
\end{enumerate}\pause
Divided this project into two parts:
\begin{enumerate}
\item Calculate the factor strength, only focus on factor with certain strength
\item Use Elastic Net model, overcome the correlation problem 
\end{enumerate}
\end{frame}
	
\begin{frame}
\frametitle{Factor Strength}
Elaborated by  \citeA{Pesaran2019}, \citeA{Bailey2020}

Factor strength represents the factor's pervasiveness.

The stronger a factor is, the more assets it can price, i.e. has loading significantly \alert{different from zero.}

Recall the CAPM model introduced at the beginning, but ignore the market factor.

If a factor $x_j$ has strength $\alpha_j$, we let this factor regress with N different assets.
\begin{align*}
\beta_j \neq 0, j &= 1, 2, 3, \cdots, [N^{\alpha_j}]\\
\beta_j = 0, j &= [N^{\alpha_j} ]+1 ,[N^{\alpha_j}]  +2, [N^{\alpha_j}] +3, \cdots, N
\end{align*}
\end{frame}

\begin{frame}
\frametitle{Elastic Net}
Introduce by \citeA{Zou2005}, is a improved method to select factor.

Recall OLS estimation, for the factor model (\ref{2CAPM}), OLS will estimates the factor loading by minimise the residual sum of square:
\[  \hat{\beta}_j  = argmin\{\sum_{i = 1}^{N}(r_{it} - \mathbf{\beta_j }\mathbf{x_j})^2\} \]

For Elastic net, it add two penalty terms into the OLS regression:
	\[   \hat{\beta}_j  = argmin\{\sum_{i = 1}^{N}(r_{it} - \mathbf{\beta_j }\mathbf{x_j})^2 + \lambda_1\mathbf{\beta_j}^2  + \lambda_2|\mathbf{\beta_j}|  \label{ENcriterion} \tag{2}   \}    \]
	
	Advantage of Elastic Net: it can select factors from a group of correlated candidates.
\end{frame}


%-------------------------------------------------------------------------------------------------------------------------------------------------------------------------------%
%-------------------------------------------------------------------------------------------------------------------------------------------------------------------------------%


\section{Future Plan}


\begin{frame}
\frametitle{Future Plan}
For the next step, we will start the empirical analyses. 

We will collect and exam the data for the empirical research. 

Assets: {\bf Companies from Standard \& Poor (S\&P) 500 index}\\
We will use those securities return as the left hand side part of the CAPM.

Factor: {\bf Factors from \citeA{Harvey2019}'s factor list}\\
Using factor strength as the criterion to trim first, and then applied the elastic net method.

\end{frame}
%-------------------------------------------------------------------------------------------------------------------------------------------------------------------------------%
%-------------------------------------------------------------------------------------------------------------------------------------------------------------------------------%
\begin{frame}
	\centering	
\huge{ Thanks for listening\\}
\huge{ Question or Comment}
\end{frame}


%-------------------------------------------------------------------------------------------------------------------------------------------------------------------------------%
%-------------------------------------------------------------------------------------------------------------------------------------------------------------------------------%
\begin{frame}[allowframebreaks]
	\frametitle{Bibliography}
	\bibliographystyle{apacite}
	\bibliography{proposal.bib}
\end{frame}

\end{document}
